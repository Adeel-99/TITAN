%% Generated by Sphinx.
\def\sphinxdocclass{report}
\documentclass[letterpaper,10pt,english]{sphinxmanual}
\ifdefined\pdfpxdimen
   \let\sphinxpxdimen\pdfpxdimen\else\newdimen\sphinxpxdimen
\fi \sphinxpxdimen=.75bp\relax
\ifdefined\pdfimageresolution
    \pdfimageresolution= \numexpr \dimexpr1in\relax/\sphinxpxdimen\relax
\fi
%% let collapsible pdf bookmarks panel have high depth per default
\PassOptionsToPackage{bookmarksdepth=5}{hyperref}


\PassOptionsToPackage{warn}{textcomp}
\usepackage[utf8]{inputenc}
\ifdefined\DeclareUnicodeCharacter
% support both utf8 and utf8x syntaxes
  \ifdefined\DeclareUnicodeCharacterAsOptional
    \def\sphinxDUC#1{\DeclareUnicodeCharacter{"#1}}
  \else
    \let\sphinxDUC\DeclareUnicodeCharacter
  \fi
  \sphinxDUC{00A0}{\nobreakspace}
  \sphinxDUC{2500}{\sphinxunichar{2500}}
  \sphinxDUC{2502}{\sphinxunichar{2502}}
  \sphinxDUC{2514}{\sphinxunichar{2514}}
  \sphinxDUC{251C}{\sphinxunichar{251C}}
  \sphinxDUC{2572}{\textbackslash}
\fi
\usepackage{cmap}
\usepackage[T1]{fontenc}
\usepackage{amsmath,amssymb,amstext}
\usepackage{babel}



\usepackage{tgtermes}
\usepackage{tgheros}
\renewcommand{\ttdefault}{txtt}



\usepackage[Bjarne]{fncychap}
\usepackage{sphinx}

\fvset{fontsize=auto}
\usepackage{geometry}


% Include hyperref last.
\usepackage{hyperref}
% Fix anchor placement for figures with captions.
\usepackage{hypcap}% it must be loaded after hyperref.
% Set up styles of URL: it should be placed after hyperref.
\urlstyle{same}

\addto\captionsenglish{\renewcommand{\contentsname}{Contents:}}

\usepackage{sphinxmessages}
\setcounter{tocdepth}{1}



\title{TITAN}
\date{Jan 30, 2023}
\release{0.1}
\author{Fabio Morgado}
\newcommand{\sphinxlogo}{\vbox{}}
\renewcommand{\releasename}{Release}
\makeindex
\begin{document}

\ifdefined\shorthandoff
  \ifnum\catcode`\=\string=\active\shorthandoff{=}\fi
  \ifnum\catcode`\"=\active\shorthandoff{"}\fi
\fi

\pagestyle{empty}
\sphinxmaketitle
\pagestyle{plain}
\sphinxtableofcontents
\pagestyle{normal}
\phantomsection\label{\detokenize{index::doc}}


\sphinxAtStartPar
TransatmospherIc flighT simulAtioN \sphinxhyphen{} A python code for multi\sphinxhyphen{}fidelity and multi\sphinxhyphen{}physics simulations of access\sphinxhyphen{}to\sphinxhyphen{}space and re\sphinxhyphen{}entry

\sphinxAtStartPar
For further info, check {\hyperref[\detokenize{usage::doc}]{\sphinxcrossref{\DUrole{doc}{Usage}}}}. For installation, check {\hyperref[\detokenize{usage:installation}]{\sphinxcrossref{\DUrole{std,std-ref}{Installation}}}}.

\begin{sphinxadmonition}{warning}{Warning:}
\sphinxAtStartPar
This library is still under development.
\end{sphinxadmonition}

\sphinxstepscope


\chapter{Usage}
\label{\detokenize{usage:usage}}\label{\detokenize{usage::doc}}

\section{Installation}
\label{\detokenize{usage:installation}}
\sphinxAtStartPar
To install TITAN, it is required to use an Anaconda environment. The required libraries are listed in the requirements.txt file.
In order to install the required packages, the Anaconda environment can be created using

\begin{sphinxVerbatim}[commandchars=\\\{\}]
\PYG{g+gp}{\PYGZdl{} }conda create \PYGZhy{}\PYGZhy{}name myenv \PYGZhy{}\PYGZhy{}file requirements.txt
\end{sphinxVerbatim}

\sphinxAtStartPar
If the packages are not found, the user can append a conda channel to retrieve the packages, by running

\begin{sphinxVerbatim}[commandchars=\\\{\}]
\PYG{g+gp}{\PYGZdl{} }conda config \PYGZhy{}\PYGZhy{}append channels conda\PYGZhy{}forge
\end{sphinxVerbatim}

\sphinxAtStartPar
To activate the Conda environment:

\begin{sphinxVerbatim}[commandchars=\\\{\}]
\PYG{g+gp}{\PYGZdl{} }conda activate myenv
\end{sphinxVerbatim}

\sphinxAtStartPar
After activation, the user needs to install other packages that were not possible (as GMSH) using the conda requirements. To do so,
the user can use the pip manager to install the packages listed in the pip\_requirements.txt

\begin{sphinxVerbatim}[commandchars=\\\{\}]
\PYG{g+gp+gpVirtualEnv}{(.venv)} \PYG{g+gp}{\PYGZdl{} }pip install \PYGZhy{}r pip\PYGZus{}requirements.txt
\end{sphinxVerbatim}

\sphinxAtStartPar
To install pymap3D, you can clone the following github page \sphinxurl{https://github.com/geospace-code/pymap3d/} into the Executables foldar and install using

\begin{sphinxVerbatim}[commandchars=\\\{\}]
\PYG{g+gp+gpVirtualEnv}{(.venv)} \PYG{g+gp}{\PYGZdl{} }pip install \PYGZhy{}e pymap3d
\end{sphinxVerbatim}


\subsection{Optional}
\label{\detokenize{usage:optional}}

\subsubsection{Mutation++}
\label{\detokenize{usage:mutation}}
\sphinxAtStartPar
The mutation++ package is an optional method to compute the freestream conditions. It can be installed by following the instructions in \sphinxurl{https://github.com/mutationpp/Mutationpp}
Once the mutation++ has been compiled, you can install by:
\begin{enumerate}
\sphinxsetlistlabels{\arabic}{enumi}{enumii}{}{.}%
\item {} 
\sphinxAtStartPar
In the github link, go to the thirdpary folder and clone the Pybind repository into your thirdparty folder in mutationpp

\item {} 
\sphinxAtStartPar
In the Mutationpp root folder, run

\end{enumerate}

\begin{sphinxVerbatim}[commandchars=\\\{\}]
\PYG{g+gp+gpVirtualEnv}{(.venv)} \PYG{g+gp}{\PYGZdl{} }python setup.py build
\PYG{g+gp+gpVirtualEnv}{(.venv)} \PYG{g+gp}{\PYGZdl{} }python setup.py install
\end{sphinxVerbatim}


\subsubsection{AMGio}
\label{\detokenize{usage:amgio}}
\sphinxAtStartPar
AMGio is a library that is required to perform mesh adaptation when running high\sphinxhyphen{}fidelity simulations. To install the AMGio library, one must clone the following github page to the TITAN/Executables folder: \sphinxurl{https://github.com/bmunguia/amgio}. THe user can the proceed to the installation using

\begin{sphinxVerbatim}[commandchars=\\\{\}]
\PYG{g+gp+gpVirtualEnv}{(.venv)} \PYG{g+gp}{\PYGZdl{} }pip install \PYGZhy{}e amgio/su2gmf/
\end{sphinxVerbatim}


\subsection{GRAM model}
\label{\detokenize{usage:gram-model}}
\sphinxAtStartPar
TITAN has the capability to use the NASA\sphinxhyphen{}GRAM \sphinxurl{https://software.nasa.gov/software/MFS-33888-1} to retrieve the atmospheric properties of Earth, Neptune and Uranus. The user needs to request NASA to use the atmospheric model.

\sphinxAtStartPar
Once the GRAM tool is compiled, the user needs to link the binaries, and place them in the Executables folder


\subsection{Troubleshooting}
\label{\detokenize{usage:troubleshooting}}
\sphinxAtStartPar
If mpirun is not working, the user may require to reinstall openmpi and/or mpi4py using pip, by following the steps:

\begin{sphinxVerbatim}[commandchars=\\\{\}]
\PYG{g+gp+gpVirtualEnv}{(.venv)} \PYG{g+gp}{\PYGZdl{} }mamba uninstall mpi4py
\PYG{g+gp+gpVirtualEnv}{(.venv)} \PYG{g+gp}{\PYGZdl{} }mamba uninstall openmpi
\PYG{g+gp+gpVirtualEnv}{(.venv)} \PYG{g+gp}{\PYGZdl{} }pip install openmpi
\PYG{g+gp+gpVirtualEnv}{(.venv)} \PYG{g+gp}{\PYGZdl{} }pip install mpi4py
\end{sphinxVerbatim}


\section{Setting up the Configuration file}
\label{\detokenize{usage:setting-up-the-configuration-file}}
\sphinxAtStartPar
An explanation of the Configuration file can be found in the Config\_temmplate.cfg file, in the root folder.

\sphinxAtStartPar
TITAN will read the configuration file using the config parser package. The file is divided into several subsections:


\subsection{Options}
\label{\detokenize{usage:options}}\begin{itemize}
\item {} 
\sphinxAtStartPar
\sphinxstylestrong{Num\_iters} \sphinxhyphen{} Maximum number of iterations

\item {} 
\sphinxAtStartPar
\sphinxstylestrong{Load\_State} \sphinxhyphen{} Load the last simulation state

\item {} 
\sphinxAtStartPar
\sphinxstylestrong{Fidelity} \sphinxhyphen{} Select the level of the aerothermodynamics in the simulation (Low/High/Multi)

\item {} 
\sphinxAtStartPar
\sphinxstylestrong{Output\_folder} \sphinxhyphen{} Folder where the simulation solution is stored

\item {} 
\sphinxAtStartPar
\sphinxstylestrong{Load\_mesh} \sphinxhyphen{} Flag to indicate if the mesh should be loaded (if already pre\sphinxhyphen{}processed in previous simulation)

\item {} 
\sphinxAtStartPar
\sphinxstylestrong{Load\_state} \sphinxhyphen{} Flag to resume the simulation (overrules the flag Load\_mesh)

\end{itemize}


\subsection{Trajectory}
\label{\detokenize{usage:trajectory}}\begin{itemize}
\item {} 
\sphinxAtStartPar
\sphinxstylestrong{Altitude} \sphinxhyphen{} Initial altitude {[}meters{]}

\item {} 
\sphinxAtStartPar
\sphinxstylestrong{Velocity} \sphinxhyphen{} Initial Velocity {[}meters/second{]}

\item {} 
\sphinxAtStartPar
\sphinxstylestrong{Flight\_path\_angle} \sphinxhyphen{} Initial FLight Path Angle {[}degree{]}

\item {} 
\sphinxAtStartPar
\sphinxstylestrong{Heading\_angle} \sphinxhyphen{} Initial Heading Angle {[}degree{]}

\item {} 
\sphinxAtStartPar
\sphinxstylestrong{Latitude} \sphinxhyphen{} Initial Latitude {[}degree{]}

\item {} 
\sphinxAtStartPar
\sphinxstylestrong{Longitude} \sphinxhyphen{} Initial Longitude {[}degree{]}

\end{itemize}


\subsection{Model}
\label{\detokenize{usage:model}}\begin{itemize}
\item {} 
\sphinxAtStartPar
\sphinxstylestrong{Planet} \sphinxhyphen{} Name of the planel (Earth, Neptune, Uranus)

\item {} 
\sphinxAtStartPar
\sphinxstylestrong{Vehicle} \sphinxhyphen{} Flag for use of custom vehicle parameters (Mass, Nose radius, Area of reference)

\item {} 
\sphinxAtStartPar
\sphinxstylestrong{Drag} \sphinxhyphen{} Flag for use of drag model (if Vehicle = True)

\end{itemize}


\subsection{Vehicle}
\label{\detokenize{usage:vehicle}}\begin{itemize}
\item {} 
\sphinxAtStartPar
\sphinxstylestrong{Mass} \sphinxhyphen{} Mass of the vehicle {[}kg{]}

\item {} 
\sphinxAtStartPar
\sphinxstylestrong{Nose\_radius} \sphinxhyphen{} Nose radius of the vehicle {[}meters{]}

\item {} 
\sphinxAtStartPar
\sphinxstylestrong{Area\_reference} \sphinxhyphen{} Area of reference for coefficient computation  {[}meters\textasciicircum{}2{]}

\item {} 
\sphinxAtStartPar
\sphinxstylestrong{Drag\_file} \sphinxhyphen{} Name of the Drag model containing the Mach vs drag coefficient information in TITAN/Model/Drag

\end{itemize}


\subsection{Freestream}
\label{\detokenize{usage:freestream}}\begin{itemize}
\item {} 
\sphinxAtStartPar
\sphinxstylestrong{method} \sphinxhyphen{} Method used for the computation of the freestream (Standard, Mutationpp, GRAM)

\item {} 
\sphinxAtStartPar
\sphinxstylestrong{model} \sphinxhyphen{} Atmospheric model (Earth \sphinxhyphen{} NRLMSISE00,GRAM ; Neptune \sphinxhyphen{} GRAM; Uranus \sphinxhyphen{} GRAM)

\end{itemize}


\subsection{GRAM}
\label{\detokenize{usage:gram}}\begin{itemize}
\item {} 
\sphinxAtStartPar
\sphinxstylestrong{MinMaxFactor} \sphinxhyphen{} Value of the MinMaxFactor for the NeptuneGRAM

\item {} 
\sphinxAtStartPar
\sphinxstylestrong{ComputeMinMaxFactor} \sphinxhyphen{} Automatic computation of the MinMaxFactor for the NeptuneGRAM (see NeptuneGRAM manual. 0 = False, 1 = True)

\item {} 
\sphinxAtStartPar
\sphinxstylestrong{SPICE\_Path} \sphinxhyphen{} Path for the SPICE database

\item {} 
\sphinxAtStartPar
\sphinxstylestrong{GRAM\_Path} \sphinxhyphen{} Path for GRAM software (required for Earth GRAM)

\end{itemize}


\subsection{Time}
\label{\detokenize{usage:time}}\begin{itemize}
\item {} 
\sphinxAtStartPar
\sphinxstylestrong{Time\_step} \sphinxhyphen{} Value of the time step {[}second{]}

\end{itemize}


\subsection{SU2}
\label{\detokenize{usage:su2}}\begin{itemize}
\item {} 
\sphinxAtStartPar
\sphinxstylestrong{Solver} \sphinxhyphen{} Solver to be used in CFD simulation (EULER/NAVIER\_STOKES or NEMO\_EULER/NEMO\_NAVIER\_STOKES)

\item {} 
\sphinxAtStartPar
\sphinxstylestrong{Num\_iters} \sphinxhyphen{} Number of CFD iterations

\item {} 
\sphinxAtStartPar
\sphinxstylestrong{Conv\_method} \sphinxhyphen{} Convective scheme (Default = AUSM)

\item {} 
\sphinxAtStartPar
\sphinxstylestrong{Adapt\_iter} \sphinxhyphen{} Number of mesh adaptations

\item {} 
\sphinxAtStartPar
\sphinxstylestrong{Num\_cores} \sphinxhyphen{} Number of cores to run CFD simulation

\item {} 
\sphinxAtStartPar
\sphinxstylestrong{Muscl} \sphinxhyphen{} Flag for MUSCL reconstruction (Yes/No)

\item {} 
\sphinxAtStartPar
\sphinxstylestrong{Cfl} \sphinxhyphen{} CFL number

\end{itemize}


\subsection{Bloom}
\label{\detokenize{usage:bloom}}\begin{itemize}
\item {} 
\sphinxAtStartPar
\sphinxstylestrong{Flag} \sphinxhyphen{} Flag to activate Bloom (True/False)

\item {} 
\sphinxAtStartPar
\sphinxstylestrong{Layers} \sphinxhyphen{} Number of layers in the boundary layer

\item {} 
\sphinxAtStartPar
\sphinxstylestrong{Spacing} \sphinxhyphen{} Spacing of the initial layer

\item {} 
\sphinxAtStartPar
\sphinxstylestrong{Growth\_Rate} \sphinxhyphen{} Growth rate between layers

\end{itemize}


\subsection{AMG}
\label{\detokenize{usage:amg}}\begin{itemize}
\item {} 
\sphinxAtStartPar
\sphinxstylestrong{Flag} \sphinxhyphen{} Flag to activate AMG

\item {} 
\sphinxAtStartPar
\sphinxstylestrong{P} \sphinxhyphen{} Norm of the error estimate for the Hessian computation

\item {} 
\sphinxAtStartPar
\sphinxstylestrong{C} \sphinxhyphen{} Correction for metric complexity

\item {} 
\sphinxAtStartPar
\sphinxstylestrong{Sensor} \sphinxhyphen{} Name of the computational field used to compute the metric tensor for mesh adaptation

\end{itemize}


\subsection{Assembly}
\label{\detokenize{usage:assembly}}\begin{itemize}
\item {} 
\sphinxAtStartPar
\sphinxstylestrong{Path} \sphinxhyphen{} Path for the geometry files

\item {} 
\sphinxAtStartPar
\sphinxstylestrong{Connectivity} \sphinxhyphen{} Linkage information for the specified components in the Objects section

\item {} 
\sphinxAtStartPar
\sphinxstylestrong{Angle\_of\_attack} \sphinxhyphen{} Angle of attack of the assembly {[}degree{]}

\item {} 
\sphinxAtStartPar
\sphinxstylestrong{Sideslip} \sphinxhyphen{} Angle of sideslip of the assembly {[}degree{]}

\end{itemize}


\subsection{Objects}
\label{\detokenize{usage:objects}}\begin{itemize}
\item {} 
\sphinxAtStartPar
\sphinxstylestrong{Primitive used in the Assembly} \sphinxhyphen{} name\_Marker = (NAME, TYPE, MATERIAL)

\item {} 
\sphinxAtStartPar
\sphinxstylestrong{Joints used in the Assembly} \sphinxhyphen{} name\_Marker = (NAME, TYPE, MATERIAL, TRIGGER\_TYPE, TRIGGER\_VALUE)
\begin{itemize}
\item {} 
\sphinxAtStartPar
NAME \sphinxhyphen{}\textgreater{} Name of the geometry file in stl format

\item {} 
\sphinxAtStartPar
TYPE \sphinxhyphen{}\textgreater{} Type of the object (Primitive/Joint)

\item {} 
\sphinxAtStartPar
MATERIAL \sphinxhyphen{}\textgreater{} Material of the object, needs to one specified in the material database

\item {} 
\sphinxAtStartPar
TRIGGER\_TYPE  \sphinxhyphen{}\textgreater{} The criteria for the joint fragmentation (Altitude, time, iteration, Temperature)

\item {} 
\sphinxAtStartPar
TRIGGER\_VALUE \sphinxhyphen{}\textgreater{} The value to trigger the fragmentation

\end{itemize}

\end{itemize}


\section{Running a simulation}
\label{\detokenize{usage:running-a-simulation}}
\sphinxAtStartPar
TITAN is called in the conda environment using

\begin{sphinxVerbatim}[commandchars=\\\{\}]
\PYG{g+gp+gpVirtualEnv}{(.venv)} \PYG{g+gp}{\PYGZdl{} }python TITAN.py \PYGZhy{}c config.cfg
\end{sphinxVerbatim}

\sphinxAtStartPar
The solution is stored in the specifed output folder. The structure in the output folder is as \sphinxstylestrong{SPECIFY HERE}

\sphinxAtStartPar
After obtaining the solution of the simulation, the data can be postprocessed by introducing a new flag to the instruction, refering to the Postprocess method that can be \sphinxstylestrong{WIND} or \sphinxstylestrong{ECEF}. The following command does not run a new simulation, but it postprocess the already obtained solutions in the \sphinxstylestrong{Output\_folder} specified field.

\begin{sphinxVerbatim}[commandchars=\\\{\}]
\PYG{g+gp+gpVirtualEnv}{(.venv)} \PYG{g+gp}{\PYGZdl{} }python TITAN.py \PYGZhy{}c config.cfg \PYGZhy{}pp WIND
\end{sphinxVerbatim}


\section{Geometry modelling}
\label{\detokenize{usage:geometry-modelling}}
\sphinxAtStartPar
The frame convention in the geometry modelling are such that the X axis is the longitudinal axis pointing ahead, Z axis is the vertical axis pointing downwards, and the Y axis is the lateral one, pointing in such a way that the frame is right\sphinxhyphen{}handed.

\sphinxAtStartPar
In case of multiple components, if the components are in contact with each other, the respective meshes need to be identical in the interface (i.e. same node positioning and same facets).

\sphinxstepscope


\chapter{Modules}
\label{\detokenize{modules:modules}}\label{\detokenize{modules::doc}}

\section{TITAN}
\label{\detokenize{modules:titan}}\index{main() (in module TITAN)@\spxentry{main()}\spxextra{in module TITAN}}

\begin{fulllineitems}
\phantomsection\label{\detokenize{modules:TITAN.main}}
\pysigstartsignatures
\pysiglinewithargsret{\sphinxcode{\sphinxupquote{TITAN.}}\sphinxbfcode{\sphinxupquote{main}}}{\emph{\DUrole{n}{filename}\DUrole{o}{=}\DUrole{default_value}{\textquotesingle{}\textquotesingle{}}}, \emph{\DUrole{n}{postprocess}\DUrole{o}{=}\DUrole{default_value}{\textquotesingle{}\textquotesingle{}}}}{}
\pysigstopsignatures
\sphinxAtStartPar
TITAN main function
\begin{quote}\begin{description}
\sphinxlineitem{Parameters}\begin{itemize}
\item {} 
\sphinxAtStartPar
\sphinxstyleliteralstrong{\sphinxupquote{filename}} (\sphinxstyleliteralemphasis{\sphinxupquote{str}}) \textendash{} Name of the configuration file

\item {} 
\sphinxAtStartPar
\sphinxstyleliteralstrong{\sphinxupquote{postprocess}} (\sphinxstyleliteralemphasis{\sphinxupquote{str}}) \textendash{} Postprocess method. If specified, TITAN will only perform the postprocess of the already obtained solution in the specified output folder.
The config fille still needs to be specified.

\end{itemize}

\end{description}\end{quote}

\end{fulllineitems}

\index{loop() (in module TITAN)@\spxentry{loop()}\spxextra{in module TITAN}}

\begin{fulllineitems}
\phantomsection\label{\detokenize{modules:TITAN.loop}}
\pysigstartsignatures
\pysiglinewithargsret{\sphinxcode{\sphinxupquote{TITAN.}}\sphinxbfcode{\sphinxupquote{loop}}}{\emph{\DUrole{n}{options}\DUrole{o}{=}\DUrole{default_value}{{[}{]}}}, \emph{\DUrole{n}{titan}\DUrole{o}{=}\DUrole{default_value}{{[}{]}}}}{}
\pysigstopsignatures
\sphinxAtStartPar
Simulation loop for time propagation

\sphinxAtStartPar
The function calls the different modules to perform
dynamics propagation, thermal ablation, fragmentation
assessment and structural dynamics for each time iteration.
The loop finishes when the iteration number is higher than
the one the user specified.
\begin{quote}\begin{description}
\sphinxlineitem{Parameters}\begin{itemize}
\item {} 
\sphinxAtStartPar
\sphinxstyleliteralstrong{\sphinxupquote{options}} ({\hyperref[\detokenize{modules:configuration.Options}]{\sphinxcrossref{\sphinxstyleliteralemphasis{\sphinxupquote{Options}}}}}) \textendash{} object of class {\hyperref[\detokenize{modules:configuration.Options}]{\sphinxcrossref{\sphinxcode{\sphinxupquote{configuration.Options}}}}}

\item {} 
\sphinxAtStartPar
\sphinxstyleliteralstrong{\sphinxupquote{titan}} ({\hyperref[\detokenize{modules:assembly.Assembly_list}]{\sphinxcrossref{\sphinxstyleliteralemphasis{\sphinxupquote{Assembly\_list}}}}}) \textendash{} object of class Assembly\_list

\end{itemize}

\end{description}\end{quote}

\end{fulllineitems}



\section{Configuration}
\label{\detokenize{modules:configuration}}\index{Trajectory (class in configuration)@\spxentry{Trajectory}\spxextra{class in configuration}}

\begin{fulllineitems}
\phantomsection\label{\detokenize{modules:configuration.Trajectory}}
\pysigstartsignatures
\pysiglinewithargsret{\sphinxbfcode{\sphinxupquote{class\DUrole{w}{  }}}\sphinxcode{\sphinxupquote{configuration.}}\sphinxbfcode{\sphinxupquote{Trajectory}}}{\emph{\DUrole{n}{altitude}\DUrole{o}{=}\DUrole{default_value}{0}}, \emph{\DUrole{n}{gamma}\DUrole{o}{=}\DUrole{default_value}{0}}, \emph{\DUrole{n}{chi}\DUrole{o}{=}\DUrole{default_value}{0}}, \emph{\DUrole{n}{velocity}\DUrole{o}{=}\DUrole{default_value}{0}}, \emph{\DUrole{n}{latitude}\DUrole{o}{=}\DUrole{default_value}{0}}, \emph{\DUrole{n}{longitude}\DUrole{o}{=}\DUrole{default_value}{0}}}{}
\pysigstopsignatures
\sphinxAtStartPar
Class Trajectory

\sphinxAtStartPar
A class to store the user\sphinxhyphen{}defined trajectory information
\index{altitude (configuration.Trajectory attribute)@\spxentry{altitude}\spxextra{configuration.Trajectory attribute}}

\begin{fulllineitems}
\phantomsection\label{\detokenize{modules:configuration.Trajectory.altitude}}
\pysigstartsignatures
\pysigline{\sphinxbfcode{\sphinxupquote{altitude}}}
\pysigstopsignatures
\sphinxAtStartPar
{[}meters{]} Altitude value.

\end{fulllineitems}

\index{chi (configuration.Trajectory attribute)@\spxentry{chi}\spxextra{configuration.Trajectory attribute}}

\begin{fulllineitems}
\phantomsection\label{\detokenize{modules:configuration.Trajectory.chi}}
\pysigstartsignatures
\pysigline{\sphinxbfcode{\sphinxupquote{chi}}}
\pysigstopsignatures
\sphinxAtStartPar
{[}radians{]} Heading Angle value.

\end{fulllineitems}

\index{gamma (configuration.Trajectory attribute)@\spxentry{gamma}\spxextra{configuration.Trajectory attribute}}

\begin{fulllineitems}
\phantomsection\label{\detokenize{modules:configuration.Trajectory.gamma}}
\pysigstartsignatures
\pysigline{\sphinxbfcode{\sphinxupquote{gamma}}}
\pysigstopsignatures
\sphinxAtStartPar
{[}radians{]} Flight Path Angle value.

\end{fulllineitems}

\index{latitude (configuration.Trajectory attribute)@\spxentry{latitude}\spxextra{configuration.Trajectory attribute}}

\begin{fulllineitems}
\phantomsection\label{\detokenize{modules:configuration.Trajectory.latitude}}
\pysigstartsignatures
\pysigline{\sphinxbfcode{\sphinxupquote{latitude}}}
\pysigstopsignatures
\sphinxAtStartPar
{[}radians{]} Latitude value.

\end{fulllineitems}

\index{longitude (configuration.Trajectory attribute)@\spxentry{longitude}\spxextra{configuration.Trajectory attribute}}

\begin{fulllineitems}
\phantomsection\label{\detokenize{modules:configuration.Trajectory.longitude}}
\pysigstartsignatures
\pysigline{\sphinxbfcode{\sphinxupquote{longitude}}}
\pysigstopsignatures
\sphinxAtStartPar
{[}radians{]} Longitude value.

\end{fulllineitems}

\index{velocity (configuration.Trajectory attribute)@\spxentry{velocity}\spxextra{configuration.Trajectory attribute}}

\begin{fulllineitems}
\phantomsection\label{\detokenize{modules:configuration.Trajectory.velocity}}
\pysigstartsignatures
\pysigline{\sphinxbfcode{\sphinxupquote{velocity}}}
\pysigstopsignatures
\sphinxAtStartPar
{[}meters/second{]} Velocity value.

\end{fulllineitems}


\end{fulllineitems}

\index{Fenics (class in configuration)@\spxentry{Fenics}\spxextra{class in configuration}}

\begin{fulllineitems}
\phantomsection\label{\detokenize{modules:configuration.Fenics}}
\pysigstartsignatures
\pysiglinewithargsret{\sphinxbfcode{\sphinxupquote{class\DUrole{w}{  }}}\sphinxcode{\sphinxupquote{configuration.}}\sphinxbfcode{\sphinxupquote{Fenics}}}{\emph{\DUrole{n}{E}\DUrole{o}{=}\DUrole{default_value}{68000000000.0}}, \emph{\DUrole{n}{FENICS}\DUrole{o}{=}\DUrole{default_value}{False}}, \emph{\DUrole{n}{FE\_MPI}\DUrole{o}{=}\DUrole{default_value}{False}}, \emph{\DUrole{n}{FE\_MPI\_cores}\DUrole{o}{=}\DUrole{default_value}{12}}, \emph{\DUrole{n}{FE\_verbose}\DUrole{o}{=}\DUrole{default_value}{False}}}{}
\pysigstopsignatures
\sphinxAtStartPar
FEniCS class

\sphinxAtStartPar
Class to store the user\sphinxhyphen{}defined information for the structural dynamics simulation using FEniCS
\index{E (configuration.Fenics attribute)@\spxentry{E}\spxextra{configuration.Fenics attribute}}

\begin{fulllineitems}
\phantomsection\label{\detokenize{modules:configuration.Fenics.E}}
\pysigstartsignatures
\pysigline{\sphinxbfcode{\sphinxupquote{E}}}
\pysigstopsignatures
\sphinxAtStartPar
{[}Pa{]} Young Modulus

\end{fulllineitems}

\index{FE\_MPI (configuration.Fenics attribute)@\spxentry{FE\_MPI}\spxextra{configuration.Fenics attribute}}

\begin{fulllineitems}
\phantomsection\label{\detokenize{modules:configuration.Fenics.FE_MPI}}
\pysigstartsignatures
\pysigline{\sphinxbfcode{\sphinxupquote{FE\_MPI}}}
\pysigstopsignatures
\sphinxAtStartPar
{[}bool{]} Flag value indicating if MPI is to be used for FEniCS

\end{fulllineitems}

\index{FE\_MPI\_cores (configuration.Fenics attribute)@\spxentry{FE\_MPI\_cores}\spxextra{configuration.Fenics attribute}}

\begin{fulllineitems}
\phantomsection\label{\detokenize{modules:configuration.Fenics.FE_MPI_cores}}
\pysigstartsignatures
\pysigline{\sphinxbfcode{\sphinxupquote{FE\_MPI\_cores}}}
\pysigstopsignatures
\sphinxAtStartPar
{[}int{]} Flag value indicating the number of cores if \sphinxstylestrong{FE\_MPI=True}

\end{fulllineitems}

\index{FE\_verbose (configuration.Fenics attribute)@\spxentry{FE\_verbose}\spxextra{configuration.Fenics attribute}}

\begin{fulllineitems}
\phantomsection\label{\detokenize{modules:configuration.Fenics.FE_verbose}}
\pysigstartsignatures
\pysigline{\sphinxbfcode{\sphinxupquote{FE\_verbose}}}
\pysigstopsignatures
\sphinxAtStartPar
{[}bool{]} Flag value indicating the verbosity of FEniCS solver

\end{fulllineitems}


\end{fulllineitems}

\index{Dynamics (class in configuration)@\spxentry{Dynamics}\spxextra{class in configuration}}

\begin{fulllineitems}
\phantomsection\label{\detokenize{modules:configuration.Dynamics}}
\pysigstartsignatures
\pysiglinewithargsret{\sphinxbfcode{\sphinxupquote{class\DUrole{w}{  }}}\sphinxcode{\sphinxupquote{configuration.}}\sphinxbfcode{\sphinxupquote{Dynamics}}}{\emph{\DUrole{n}{time\_step}\DUrole{o}{=}\DUrole{default_value}{0}}, \emph{\DUrole{n}{time}\DUrole{o}{=}\DUrole{default_value}{0}}, \emph{\DUrole{n}{propagator}\DUrole{o}{=}\DUrole{default_value}{\textquotesingle{}EULER\textquotesingle{}}}, \emph{\DUrole{n}{adapt\_propagator}\DUrole{o}{=}\DUrole{default_value}{False}}, \emph{\DUrole{n}{manifold\_correction}\DUrole{o}{=}\DUrole{default_value}{True}}}{}
\pysigstopsignatures
\sphinxAtStartPar
Dynamics class

\sphinxAtStartPar
A class to store the user\sphinxhyphen{}defined dynamics options for the simulation
\index{adapt\_propagator (configuration.Dynamics attribute)@\spxentry{adapt\_propagator}\spxextra{configuration.Dynamics attribute}}

\begin{fulllineitems}
\phantomsection\label{\detokenize{modules:configuration.Dynamics.adapt_propagator}}
\pysigstartsignatures
\pysigline{\sphinxbfcode{\sphinxupquote{adapt\_propagator}}}
\pysigstopsignatures
\sphinxAtStartPar
{[}bool{]} Flag value indicating time\sphinxhyphen{}step adaptation

\end{fulllineitems}

\index{manifold\_correction (configuration.Dynamics attribute)@\spxentry{manifold\_correction}\spxextra{configuration.Dynamics attribute}}

\begin{fulllineitems}
\phantomsection\label{\detokenize{modules:configuration.Dynamics.manifold_correction}}
\pysigstartsignatures
\pysigline{\sphinxbfcode{\sphinxupquote{manifold\_correction}}}
\pysigstopsignatures
\sphinxAtStartPar
{[}bool{]} Flag value indicating manifold correction

\end{fulllineitems}

\index{propagator (configuration.Dynamics attribute)@\spxentry{propagator}\spxextra{configuration.Dynamics attribute}}

\begin{fulllineitems}
\phantomsection\label{\detokenize{modules:configuration.Dynamics.propagator}}
\pysigstartsignatures
\pysigline{\sphinxbfcode{\sphinxupquote{propagator}}}
\pysigstopsignatures
\sphinxAtStartPar
{[}str{]} Name of the propagator to be used in the dynamics (options \sphinxhyphen{} EULER).

\end{fulllineitems}

\index{time (configuration.Dynamics attribute)@\spxentry{time}\spxextra{configuration.Dynamics attribute}}

\begin{fulllineitems}
\phantomsection\label{\detokenize{modules:configuration.Dynamics.time}}
\pysigstartsignatures
\pysigline{\sphinxbfcode{\sphinxupquote{time}}}
\pysigstopsignatures
\sphinxAtStartPar
{[}seconds{]} Physical time of the simulation.

\end{fulllineitems}

\index{time\_step (configuration.Dynamics attribute)@\spxentry{time\_step}\spxextra{configuration.Dynamics attribute}}

\begin{fulllineitems}
\phantomsection\label{\detokenize{modules:configuration.Dynamics.time_step}}
\pysigstartsignatures
\pysigline{\sphinxbfcode{\sphinxupquote{time\_step}}}
\pysigstopsignatures
\sphinxAtStartPar
{[}seconds{]} Value of the time\sphinxhyphen{}step.

\end{fulllineitems}


\end{fulllineitems}

\index{Aerothermo (class in configuration)@\spxentry{Aerothermo}\spxextra{class in configuration}}

\begin{fulllineitems}
\phantomsection\label{\detokenize{modules:configuration.Aerothermo}}
\pysigstartsignatures
\pysiglinewithargsret{\sphinxbfcode{\sphinxupquote{class\DUrole{w}{  }}}\sphinxcode{\sphinxupquote{configuration.}}\sphinxbfcode{\sphinxupquote{Aerothermo}}}{\emph{\DUrole{n}{heat\_model}\DUrole{o}{=}\DUrole{default_value}{\textquotesingle{}vd\textquotesingle{}}}, \emph{\DUrole{n}{knc\_pressure}\DUrole{o}{=}\DUrole{default_value}{0.0001}}, \emph{\DUrole{n}{knc\_heatflux}\DUrole{o}{=}\DUrole{default_value}{0.005}}, \emph{\DUrole{n}{knf}\DUrole{o}{=}\DUrole{default_value}{100}}}{}
\pysigstopsignatures
\sphinxAtStartPar
Aerothermo class

\sphinxAtStartPar
A class to store the user\sphinxhyphen{}defined aerothemo model options
\index{heat\_model (configuration.Aerothermo attribute)@\spxentry{heat\_model}\spxextra{configuration.Aerothermo attribute}}

\begin{fulllineitems}
\phantomsection\label{\detokenize{modules:configuration.Aerothermo.heat_model}}
\pysigstartsignatures
\pysigline{\sphinxbfcode{\sphinxupquote{heat\_model}}}
\pysigstopsignatures
\sphinxAtStartPar
{[}str{]} Name of the heatflux model to be used

\end{fulllineitems}

\index{knc\_heatflux (configuration.Aerothermo attribute)@\spxentry{knc\_heatflux}\spxextra{configuration.Aerothermo attribute}}

\begin{fulllineitems}
\phantomsection\label{\detokenize{modules:configuration.Aerothermo.knc_heatflux}}
\pysigstartsignatures
\pysigline{\sphinxbfcode{\sphinxupquote{knc\_heatflux}}}
\pysigstopsignatures
\sphinxAtStartPar
{[}float{]} Value of the continuum knudsen for the heatflux computation

\end{fulllineitems}

\index{knc\_pressure (configuration.Aerothermo attribute)@\spxentry{knc\_pressure}\spxextra{configuration.Aerothermo attribute}}

\begin{fulllineitems}
\phantomsection\label{\detokenize{modules:configuration.Aerothermo.knc_pressure}}
\pysigstartsignatures
\pysigline{\sphinxbfcode{\sphinxupquote{knc\_pressure}}}
\pysigstopsignatures
\sphinxAtStartPar
{[}float{]} Value of the continuum knudsen for the pressure computation

\end{fulllineitems}

\index{knf (configuration.Aerothermo attribute)@\spxentry{knf}\spxextra{configuration.Aerothermo attribute}}

\begin{fulllineitems}
\phantomsection\label{\detokenize{modules:configuration.Aerothermo.knf}}
\pysigstartsignatures
\pysigline{\sphinxbfcode{\sphinxupquote{knf}}}
\pysigstopsignatures
\sphinxAtStartPar
{[}float{]} Value of the free\sphinxhyphen{}molecular knudsen

\end{fulllineitems}


\end{fulllineitems}

\index{Freestream (class in configuration)@\spxentry{Freestream}\spxextra{class in configuration}}

\begin{fulllineitems}
\phantomsection\label{\detokenize{modules:configuration.Freestream}}
\pysigstartsignatures
\pysigline{\sphinxbfcode{\sphinxupquote{class\DUrole{w}{  }}}\sphinxcode{\sphinxupquote{configuration.}}\sphinxbfcode{\sphinxupquote{Freestream}}}
\pysigstopsignatures
\sphinxAtStartPar
Freestream class

\sphinxAtStartPar
A class to store the user\sphinxhyphen{}defined freestream properties per time iteration
\index{R (configuration.Freestream attribute)@\spxentry{R}\spxextra{configuration.Freestream attribute}}

\begin{fulllineitems}
\phantomsection\label{\detokenize{modules:configuration.Freestream.R}}
\pysigstartsignatures
\pysigline{\sphinxbfcode{\sphinxupquote{R}}}
\pysigstopsignatures
\sphinxAtStartPar
{[} ?? {]} Value of the Gas constant

\end{fulllineitems}

\index{Temperature (configuration.Freestream attribute)@\spxentry{Temperature}\spxextra{configuration.Freestream attribute}}

\begin{fulllineitems}
\phantomsection\label{\detokenize{modules:configuration.Freestream.Temperature}}
\pysigstartsignatures
\pysigline{\sphinxbfcode{\sphinxupquote{Temperature}}}
\pysigstopsignatures
\sphinxAtStartPar
{[}kelvin{]} Value of the freestream temperature

\end{fulllineitems}

\index{Velocity (configuration.Freestream attribute)@\spxentry{Velocity}\spxextra{configuration.Freestream attribute}}

\begin{fulllineitems}
\phantomsection\label{\detokenize{modules:configuration.Freestream.Velocity}}
\pysigstartsignatures
\pysigline{\sphinxbfcode{\sphinxupquote{Velocity}}}
\pysigstopsignatures
\sphinxAtStartPar
{[}m/s{]} Value of the freestream velocity

\end{fulllineitems}

\index{gamma (configuration.Freestream attribute)@\spxentry{gamma}\spxextra{configuration.Freestream attribute}}

\begin{fulllineitems}
\phantomsection\label{\detokenize{modules:configuration.Freestream.gamma}}
\pysigstartsignatures
\pysigline{\sphinxbfcode{\sphinxupquote{gamma}}}
\pysigstopsignatures
\sphinxAtStartPar
{[}float{]} Value of the freestream specific heat ratio

\end{fulllineitems}

\index{knudsen (configuration.Freestream attribute)@\spxentry{knudsen}\spxextra{configuration.Freestream attribute}}

\begin{fulllineitems}
\phantomsection\label{\detokenize{modules:configuration.Freestream.knudsen}}
\pysigstartsignatures
\pysigline{\sphinxbfcode{\sphinxupquote{knudsen}}}
\pysigstopsignatures
\sphinxAtStartPar
{[}float{]} Value of the freestream knudsen

\end{fulllineitems}

\index{mach (configuration.Freestream attribute)@\spxentry{mach}\spxextra{configuration.Freestream attribute}}

\begin{fulllineitems}
\phantomsection\label{\detokenize{modules:configuration.Freestream.mach}}
\pysigstartsignatures
\pysigline{\sphinxbfcode{\sphinxupquote{mach}}}
\pysigstopsignatures
\sphinxAtStartPar
{[}float{]} Value of the freestream Mach number

\end{fulllineitems}

\index{method (configuration.Freestream attribute)@\spxentry{method}\spxextra{configuration.Freestream attribute}}

\begin{fulllineitems}
\phantomsection\label{\detokenize{modules:configuration.Freestream.method}}
\pysigstartsignatures
\pysigline{\sphinxbfcode{\sphinxupquote{method}}}
\pysigstopsignatures
\sphinxAtStartPar
Selection of freestream calculation method (Mutationpp, default = Standard)

\end{fulllineitems}

\index{mfp (configuration.Freestream attribute)@\spxentry{mfp}\spxextra{configuration.Freestream attribute}}

\begin{fulllineitems}
\phantomsection\label{\detokenize{modules:configuration.Freestream.mfp}}
\pysigstartsignatures
\pysigline{\sphinxbfcode{\sphinxupquote{mfp}}}
\pysigstopsignatures
\sphinxAtStartPar
{[} ?? {]}

\end{fulllineitems}

\index{model (configuration.Freestream attribute)@\spxentry{model}\spxextra{configuration.Freestream attribute}}

\begin{fulllineitems}
\phantomsection\label{\detokenize{modules:configuration.Freestream.model}}
\pysigstartsignatures
\pysigline{\sphinxbfcode{\sphinxupquote{model}}}
\pysigstopsignatures
\sphinxAtStartPar
{[} ?? {]}

\end{fulllineitems}

\index{muEC (configuration.Freestream attribute)@\spxentry{muEC}\spxextra{configuration.Freestream attribute}}

\begin{fulllineitems}
\phantomsection\label{\detokenize{modules:configuration.Freestream.muEC}}
\pysigstartsignatures
\pysigline{\sphinxbfcode{\sphinxupquote{muEC}}}
\pysigstopsignatures
\sphinxAtStartPar
{[} ?? {]}

\end{fulllineitems}

\index{muSu (configuration.Freestream attribute)@\spxentry{muSu}\spxextra{configuration.Freestream attribute}}

\begin{fulllineitems}
\phantomsection\label{\detokenize{modules:configuration.Freestream.muSu}}
\pysigstartsignatures
\pysigline{\sphinxbfcode{\sphinxupquote{muSu}}}
\pysigstopsignatures
\sphinxAtStartPar
{[} ?? {]}

\end{fulllineitems}

\index{ninf (configuration.Freestream attribute)@\spxentry{ninf}\spxextra{configuration.Freestream attribute}}

\begin{fulllineitems}
\phantomsection\label{\detokenize{modules:configuration.Freestream.ninf}}
\pysigstartsignatures
\pysigline{\sphinxbfcode{\sphinxupquote{ninf}}}
\pysigstopsignatures
\sphinxAtStartPar
{[} ?? {]}

\end{fulllineitems}

\index{omega (configuration.Freestream attribute)@\spxentry{omega}\spxextra{configuration.Freestream attribute}}

\begin{fulllineitems}
\phantomsection\label{\detokenize{modules:configuration.Freestream.omega}}
\pysigstartsignatures
\pysigline{\sphinxbfcode{\sphinxupquote{omega}}}
\pysigstopsignatures
\sphinxAtStartPar
{[} ?? {]}

\end{fulllineitems}

\index{percent\_gas (configuration.Freestream attribute)@\spxentry{percent\_gas}\spxextra{configuration.Freestream attribute}}

\begin{fulllineitems}
\phantomsection\label{\detokenize{modules:configuration.Freestream.percent_gas}}
\pysigstartsignatures
\pysigline{\sphinxbfcode{\sphinxupquote{percent\_gas}}}
\pysigstopsignatures
\sphinxAtStartPar
{[} ?? {]}

\end{fulllineitems}

\index{prandtl (configuration.Freestream attribute)@\spxentry{prandtl}\spxextra{configuration.Freestream attribute}}

\begin{fulllineitems}
\phantomsection\label{\detokenize{modules:configuration.Freestream.prandtl}}
\pysigstartsignatures
\pysigline{\sphinxbfcode{\sphinxupquote{prandtl}}}
\pysigstopsignatures
\sphinxAtStartPar
{[}float{]} Value of the freestream prandtl

\end{fulllineitems}

\index{pressure (configuration.Freestream attribute)@\spxentry{pressure}\spxextra{configuration.Freestream attribute}}

\begin{fulllineitems}
\phantomsection\label{\detokenize{modules:configuration.Freestream.pressure}}
\pysigstartsignatures
\pysigline{\sphinxbfcode{\sphinxupquote{pressure}}}
\pysigstopsignatures
\sphinxAtStartPar
{[}Pa{]} Value of the freestream Pressure

\end{fulllineitems}

\index{rho (configuration.Freestream attribute)@\spxentry{rho}\spxextra{configuration.Freestream attribute}}

\begin{fulllineitems}
\phantomsection\label{\detokenize{modules:configuration.Freestream.rho}}
\pysigstartsignatures
\pysigline{\sphinxbfcode{\sphinxupquote{rho}}}
\pysigstopsignatures
\sphinxAtStartPar
{[}kg/m\textasciicircum{}3{]} Value of the freestream density

\end{fulllineitems}


\end{fulllineitems}

\index{Options (class in configuration)@\spxentry{Options}\spxextra{class in configuration}}

\begin{fulllineitems}
\phantomsection\label{\detokenize{modules:configuration.Options}}
\pysigstartsignatures
\pysiglinewithargsret{\sphinxbfcode{\sphinxupquote{class\DUrole{w}{  }}}\sphinxcode{\sphinxupquote{configuration.}}\sphinxbfcode{\sphinxupquote{Options}}}{\emph{\DUrole{n}{iters}\DUrole{o}{=}\DUrole{default_value}{1}}, \emph{\DUrole{n}{time\_step}\DUrole{o}{=}\DUrole{default_value}{0.1}}, \emph{\DUrole{n}{fidelity}\DUrole{o}{=}\DUrole{default_value}{\textquotesingle{}Low\textquotesingle{}}}, \emph{\DUrole{n}{SPARTA}\DUrole{o}{=}\DUrole{default_value}{False}}, \emph{\DUrole{n}{SP\_NUM}\DUrole{o}{=}\DUrole{default_value}{1}}, \emph{\DUrole{n}{sp\_iters}\DUrole{o}{=}\DUrole{default_value}{0}}, \emph{\DUrole{n}{SPARTA\_MPI\_cores}\DUrole{o}{=}\DUrole{default_value}{4}}, \emph{\DUrole{n}{Opti\_Flag}\DUrole{o}{=}\DUrole{default_value}{\textquotesingle{}OFF\textquotesingle{}}}, \emph{\DUrole{n}{fenics}\DUrole{o}{=}\DUrole{default_value}{False}}, \emph{\DUrole{n}{FE\_MPI}\DUrole{o}{=}\DUrole{default_value}{False}}, \emph{\DUrole{n}{FE\_MPI\_cores}\DUrole{o}{=}\DUrole{default_value}{12}}, \emph{\DUrole{n}{FE\_verbose}\DUrole{o}{=}\DUrole{default_value}{False}}, \emph{\DUrole{n}{case}\DUrole{o}{=}\DUrole{default_value}{\textquotesingle{}benchmark\textquotesingle{}}}, \emph{\DUrole{n}{E}\DUrole{o}{=}\DUrole{default_value}{68000000000.0}}, \emph{\DUrole{n}{output\_folder}\DUrole{o}{=}\DUrole{default_value}{\textquotesingle{}TITAN\_sol\textquotesingle{}}}, \emph{\DUrole{n}{propagator}\DUrole{o}{=}\DUrole{default_value}{\textquotesingle{}Euler\textquotesingle{}}}, \emph{\DUrole{n}{adapt\_propagator}\DUrole{o}{=}\DUrole{default_value}{False}}, \emph{\DUrole{n}{assembly\_rotation}\DUrole{o}{=}\DUrole{default_value}{{[}{]}}}, \emph{\DUrole{n}{manifold\_correction}\DUrole{o}{=}\DUrole{default_value}{True}}, \emph{\DUrole{n}{adapt\_time\_step}\DUrole{o}{=}\DUrole{default_value}{False}}, \emph{\DUrole{n}{rerr\_tol}\DUrole{o}{=}\DUrole{default_value}{0.001}}, \emph{\DUrole{n}{num\_joints}\DUrole{o}{=}\DUrole{default_value}{0}}, \emph{\DUrole{n}{frame\_for\_writing}\DUrole{o}{=}\DUrole{default_value}{\textquotesingle{}W\textquotesingle{}}}, \emph{\DUrole{n}{max\_time\_step}\DUrole{o}{=}\DUrole{default_value}{0.5}}, \emph{\DUrole{n}{save\_displacement}\DUrole{o}{=}\DUrole{default_value}{False}}, \emph{\DUrole{n}{save\_vonMises}\DUrole{o}{=}\DUrole{default_value}{False}}}{}
\pysigstopsignatures
\sphinxAtStartPar
Options class

\sphinxAtStartPar
A class that keeps the information of the selected user\sphinxhyphen{}defined options for all the disciplinary
areas and methods required to run the simulation
\index{aerothermo (configuration.Options attribute)@\spxentry{aerothermo}\spxextra{configuration.Options attribute}}

\begin{fulllineitems}
\phantomsection\label{\detokenize{modules:configuration.Options.aerothermo}}
\pysigstartsignatures
\pysigline{\sphinxbfcode{\sphinxupquote{aerothermo}}}
\pysigstopsignatures
\sphinxAtStartPar
{[}{\hyperref[\detokenize{modules:configuration.Aerothermo}]{\sphinxcrossref{\sphinxcode{\sphinxupquote{Aerothermo}}}}}{]} Object of class Aerothermo

\end{fulllineitems}

\index{clean\_up\_folders() (configuration.Options method)@\spxentry{clean\_up\_folders()}\spxextra{configuration.Options method}}

\begin{fulllineitems}
\phantomsection\label{\detokenize{modules:configuration.Options.clean_up_folders}}
\pysigstartsignatures
\pysiglinewithargsret{\sphinxbfcode{\sphinxupquote{clean\_up\_folders}}}{}{}
\pysigstopsignatures
\sphinxAtStartPar
Cleans the simulation output folder specified in the configuration file

\end{fulllineitems}

\index{create\_output\_folders() (configuration.Options method)@\spxentry{create\_output\_folders()}\spxextra{configuration.Options method}}

\begin{fulllineitems}
\phantomsection\label{\detokenize{modules:configuration.Options.create_output_folders}}
\pysigstartsignatures
\pysiglinewithargsret{\sphinxbfcode{\sphinxupquote{create\_output\_folders}}}{}{}
\pysigstopsignatures
\sphinxAtStartPar
Creates the folder structure to save the soluions

\end{fulllineitems}

\index{current\_iter (configuration.Options attribute)@\spxentry{current\_iter}\spxextra{configuration.Options attribute}}

\begin{fulllineitems}
\phantomsection\label{\detokenize{modules:configuration.Options.current_iter}}
\pysigstartsignatures
\pysigline{\sphinxbfcode{\sphinxupquote{current\_iter}}}
\pysigstopsignatures
\sphinxAtStartPar
{[}int{]} Current iteration

\end{fulllineitems}

\index{dynamics (configuration.Options attribute)@\spxentry{dynamics}\spxextra{configuration.Options attribute}}

\begin{fulllineitems}
\phantomsection\label{\detokenize{modules:configuration.Options.dynamics}}
\pysigstartsignatures
\pysigline{\sphinxbfcode{\sphinxupquote{dynamics}}}
\pysigstopsignatures
\sphinxAtStartPar
{[}{\hyperref[\detokenize{modules:configuration.Dynamics}]{\sphinxcrossref{\sphinxcode{\sphinxupquote{Dynamics}}}}}{]} Object of class Dynamics

\end{fulllineitems}

\index{fenics (configuration.Options attribute)@\spxentry{fenics}\spxextra{configuration.Options attribute}}

\begin{fulllineitems}
\phantomsection\label{\detokenize{modules:configuration.Options.fenics}}
\pysigstartsignatures
\pysigline{\sphinxbfcode{\sphinxupquote{fenics}}}
\pysigstopsignatures
\sphinxAtStartPar
{[}{\hyperref[\detokenize{modules:configuration.Fenics}]{\sphinxcrossref{\sphinxcode{\sphinxupquote{Fenics}}}}}{]} Object of class Fenics

\end{fulllineitems}

\index{fidelity (configuration.Options attribute)@\spxentry{fidelity}\spxextra{configuration.Options attribute}}

\begin{fulllineitems}
\phantomsection\label{\detokenize{modules:configuration.Options.fidelity}}
\pysigstartsignatures
\pysigline{\sphinxbfcode{\sphinxupquote{fidelity}}}
\pysigstopsignatures
\sphinxAtStartPar
{[}str{]} Fidelity of the aerothermo calculation (Low \sphinxhyphen{} Default, High, Hybrid)

\end{fulllineitems}

\index{freestream (configuration.Options attribute)@\spxentry{freestream}\spxextra{configuration.Options attribute}}

\begin{fulllineitems}
\phantomsection\label{\detokenize{modules:configuration.Options.freestream}}
\pysigstartsignatures
\pysigline{\sphinxbfcode{\sphinxupquote{freestream}}}
\pysigstopsignatures
\sphinxAtStartPar
{[}{\hyperref[\detokenize{modules:configuration.Freestream}]{\sphinxcrossref{\sphinxcode{\sphinxupquote{Freestream}}}}}{]} Object of class Freestream

\end{fulllineitems}

\index{iters (configuration.Options attribute)@\spxentry{iters}\spxextra{configuration.Options attribute}}

\begin{fulllineitems}
\phantomsection\label{\detokenize{modules:configuration.Options.iters}}
\pysigstartsignatures
\pysigline{\sphinxbfcode{\sphinxupquote{iters}}}
\pysigstopsignatures
\sphinxAtStartPar
{[}int{]} Number of dynamic iterations

\end{fulllineitems}

\index{read\_state() (configuration.Options method)@\spxentry{read\_state()}\spxextra{configuration.Options method}}

\begin{fulllineitems}
\phantomsection\label{\detokenize{modules:configuration.Options.read_state}}
\pysigstartsignatures
\pysiglinewithargsret{\sphinxbfcode{\sphinxupquote{read\_state}}}{}{}
\pysigstopsignatures
\sphinxAtStartPar
Load last state of the TITAN object
\begin{quote}\begin{description}
\sphinxlineitem{Returns}
\sphinxAtStartPar
\sphinxstylestrong{titan} \textendash{} Object of class Assembly\_list

\sphinxlineitem{Return type}
\sphinxAtStartPar
{\hyperref[\detokenize{modules:assembly.Assembly_list}]{\sphinxcrossref{Assembly\_list}}}

\end{description}\end{quote}

\end{fulllineitems}

\index{save\_state() (configuration.Options method)@\spxentry{save\_state()}\spxextra{configuration.Options method}}

\begin{fulllineitems}
\phantomsection\label{\detokenize{modules:configuration.Options.save_state}}
\pysigstartsignatures
\pysiglinewithargsret{\sphinxbfcode{\sphinxupquote{save\_state}}}{\emph{\DUrole{n}{titan}}, \emph{\DUrole{n}{i}\DUrole{o}{=}\DUrole{default_value}{0}}}{}
\pysigstopsignatures
\sphinxAtStartPar
Saves the TITAN object state
\begin{quote}\begin{description}
\sphinxlineitem{Parameters}\begin{itemize}
\item {} 
\sphinxAtStartPar
\sphinxstyleliteralstrong{\sphinxupquote{titan}} ({\hyperref[\detokenize{modules:assembly.Assembly_list}]{\sphinxcrossref{\sphinxstyleliteralemphasis{\sphinxupquote{Assembly\_list}}}}}) \textendash{} Object of class Assebly\_list

\item {} 
\sphinxAtStartPar
\sphinxstyleliteralstrong{\sphinxupquote{i}} (\sphinxstyleliteralemphasis{\sphinxupquote{int}}) \textendash{} Iteration number.

\end{itemize}

\end{description}\end{quote}

\end{fulllineitems}

\index{structural\_dynamics (configuration.Options attribute)@\spxentry{structural\_dynamics}\spxextra{configuration.Options attribute}}

\begin{fulllineitems}
\phantomsection\label{\detokenize{modules:configuration.Options.structural_dynamics}}
\pysigstartsignatures
\pysigline{\sphinxbfcode{\sphinxupquote{structural\_dynamics}}}
\pysigstopsignatures
\sphinxAtStartPar
{[}boolean{]} Flag to perform structural dynamics

\end{fulllineitems}


\end{fulllineitems}

\index{read\_trajectory() (in module configuration)@\spxentry{read\_trajectory()}\spxextra{in module configuration}}

\begin{fulllineitems}
\phantomsection\label{\detokenize{modules:configuration.read_trajectory}}
\pysigstartsignatures
\pysiglinewithargsret{\sphinxcode{\sphinxupquote{configuration.}}\sphinxbfcode{\sphinxupquote{read\_trajectory}}}{\emph{\DUrole{n}{configParser}}}{}
\pysigstopsignatures
\sphinxAtStartPar
Read the Trajectory specified in the config file
\begin{quote}\begin{description}
\sphinxlineitem{Parameters}
\sphinxAtStartPar
\sphinxstyleliteralstrong{\sphinxupquote{configParser}} (\sphinxstyleliteralemphasis{\sphinxupquote{configParser}}) \textendash{} Object of Config Parser

\sphinxlineitem{Returns}
\sphinxAtStartPar
\sphinxstylestrong{trajectory} \textendash{} Object of class Trajectory

\sphinxlineitem{Return type}
\sphinxAtStartPar
{\hyperref[\detokenize{modules:configuration.Trajectory}]{\sphinxcrossref{Trajectory}}}

\end{description}\end{quote}

\end{fulllineitems}

\index{read\_geometry() (in module configuration)@\spxentry{read\_geometry()}\spxextra{in module configuration}}

\begin{fulllineitems}
\phantomsection\label{\detokenize{modules:configuration.read_geometry}}
\pysigstartsignatures
\pysiglinewithargsret{\sphinxcode{\sphinxupquote{configuration.}}\sphinxbfcode{\sphinxupquote{read\_geometry}}}{\emph{\DUrole{n}{configParser}}}{}
\pysigstopsignatures
\sphinxAtStartPar
Geometry pre\sphinxhyphen{}processing

\sphinxAtStartPar
Reads the specified configuration file and creates a list with the information of the objects and assemblies
\begin{quote}\begin{description}
\sphinxlineitem{Parameters}
\sphinxAtStartPar
\sphinxstyleliteralstrong{\sphinxupquote{configParser}} (\sphinxstyleliteralemphasis{\sphinxupquote{configParser}}) \textendash{} Object of Config Parser

\sphinxlineitem{Returns}
\sphinxAtStartPar
\sphinxstylestrong{titan} \textendash{} Object of class Assembly\_list

\sphinxlineitem{Return type}
\sphinxAtStartPar
{\hyperref[\detokenize{modules:assembly.Assembly_list}]{\sphinxcrossref{Assembly\_list}}}

\end{description}\end{quote}

\end{fulllineitems}

\index{read\_config\_file() (in module configuration)@\spxentry{read\_config\_file()}\spxextra{in module configuration}}

\begin{fulllineitems}
\phantomsection\label{\detokenize{modules:configuration.read_config_file}}
\pysigstartsignatures
\pysiglinewithargsret{\sphinxcode{\sphinxupquote{configuration.}}\sphinxbfcode{\sphinxupquote{read\_config\_file}}}{\emph{\DUrole{n}{configParser}}, \emph{\DUrole{n}{postprocess}\DUrole{o}{=}\DUrole{default_value}{\textquotesingle{}\textquotesingle{}}}}{}
\pysigstopsignatures
\sphinxAtStartPar
Read the config file
\begin{quote}\begin{description}
\sphinxlineitem{Parameters}\begin{itemize}
\item {} 
\sphinxAtStartPar
\sphinxstyleliteralstrong{\sphinxupquote{configParser}} (\sphinxstyleliteralemphasis{\sphinxupquote{configParser}}) \textendash{} Object of Config Parser

\item {} 
\sphinxAtStartPar
\sphinxstyleliteralstrong{\sphinxupquote{postprocess}} (\sphinxstyleliteralemphasis{\sphinxupquote{str}}) \textendash{} Postprocess method of the solution. If not None, only returns output\_folder

\end{itemize}

\sphinxlineitem{Returns}
\sphinxAtStartPar
\begin{itemize}
\item {} 
\sphinxAtStartPar
\sphinxstylestrong{options} (\sphinxstyleemphasis{Options}) \textendash{} Object of class Options

\item {} 
\sphinxAtStartPar
\sphinxstylestrong{titan} (\sphinxstyleemphasis{Assembly\_list}) \textendash{} List of objects of class Assembly\_list

\end{itemize}


\end{description}\end{quote}

\end{fulllineitems}



\section{Aerothermo}
\label{\detokenize{modules:aerothermo}}\index{compute\_aerothermo() (in module aerothermo)@\spxentry{compute\_aerothermo()}\spxextra{in module aerothermo}}

\begin{fulllineitems}
\phantomsection\label{\detokenize{modules:aerothermo.compute_aerothermo}}
\pysigstartsignatures
\pysiglinewithargsret{\sphinxcode{\sphinxupquote{aerothermo.}}\sphinxbfcode{\sphinxupquote{compute\_aerothermo}}}{\emph{\DUrole{n}{titan}}, \emph{\DUrole{n}{options}}}{}
\pysigstopsignatures
\sphinxAtStartPar
Fidelity selection for aerothermo computation
\begin{quote}\begin{description}
\sphinxlineitem{Parameters}\begin{itemize}
\item {} 
\sphinxAtStartPar
\sphinxstyleliteralstrong{\sphinxupquote{titan}} ({\hyperref[\detokenize{modules:assembly.Assembly_list}]{\sphinxcrossref{\sphinxstyleliteralemphasis{\sphinxupquote{Assembly\_list}}}}}) \textendash{} Object of class Assembly\_list

\item {} 
\sphinxAtStartPar
\sphinxstyleliteralstrong{\sphinxupquote{options}} ({\hyperref[\detokenize{modules:configuration.Options}]{\sphinxcrossref{\sphinxstyleliteralemphasis{\sphinxupquote{Options}}}}}) \textendash{} Object of class Options

\end{itemize}

\end{description}\end{quote}

\end{fulllineitems}

\index{compute\_low\_fidelity\_aerothermo() (in module aerothermo)@\spxentry{compute\_low\_fidelity\_aerothermo()}\spxextra{in module aerothermo}}

\begin{fulllineitems}
\phantomsection\label{\detokenize{modules:aerothermo.compute_low_fidelity_aerothermo}}
\pysigstartsignatures
\pysiglinewithargsret{\sphinxcode{\sphinxupquote{aerothermo.}}\sphinxbfcode{\sphinxupquote{compute\_low\_fidelity\_aerothermo}}}{\emph{\DUrole{n}{assembly}}, \emph{\DUrole{n}{options}}}{}
\pysigstopsignatures
\sphinxAtStartPar
Low\sphinxhyphen{}fidelity aerothermo computation

\sphinxAtStartPar
Function to compute the aerodynamic and aerothermodynamic using low\sphinxhyphen{}fidelity methods.
It can compute from free\sphinxhyphen{}molecular to continuum regime. For the transitional regime, it uses a bridging methodology.
\begin{quote}\begin{description}
\sphinxlineitem{Parameters}\begin{itemize}
\item {} 
\sphinxAtStartPar
\sphinxstyleliteralstrong{\sphinxupquote{assembly}} ({\hyperref[\detokenize{modules:assembly.Assembly_list}]{\sphinxcrossref{\sphinxstyleliteralemphasis{\sphinxupquote{Assembly\_list}}}}}) \textendash{} Object of class Assembly\_list

\item {} 
\sphinxAtStartPar
\sphinxstyleliteralstrong{\sphinxupquote{options}} ({\hyperref[\detokenize{modules:configuration.Options}]{\sphinxcrossref{\sphinxstyleliteralemphasis{\sphinxupquote{Options}}}}}) \textendash{} Object of class Options

\end{itemize}

\end{description}\end{quote}

\end{fulllineitems}

\index{backfaceculling() (in module aerothermo)@\spxentry{backfaceculling()}\spxextra{in module aerothermo}}

\begin{fulllineitems}
\phantomsection\label{\detokenize{modules:aerothermo.backfaceculling}}
\pysigstartsignatures
\pysiglinewithargsret{\sphinxcode{\sphinxupquote{aerothermo.}}\sphinxbfcode{\sphinxupquote{backfaceculling}}}{\emph{\DUrole{n}{body}}, \emph{\DUrole{n}{nodes}}, \emph{\DUrole{n}{nodes\_normal}}, \emph{\DUrole{n}{free\_vector}}, \emph{\DUrole{n}{npix}}}{}
\pysigstopsignatures
\sphinxAtStartPar
Backface culling function

\sphinxAtStartPar
This function detects the facets that are impinged by the flow
\begin{quote}\begin{description}
\sphinxlineitem{Parameters}\begin{itemize}
\item {} 
\sphinxAtStartPar
\sphinxstyleliteralstrong{\sphinxupquote{body}} ({\hyperref[\detokenize{modules:assembly.Assembly}]{\sphinxcrossref{\sphinxstyleliteralemphasis{\sphinxupquote{Assembly}}}}}) \textendash{} Object of Assembly class

\item {} 
\sphinxAtStartPar
\sphinxstyleliteralstrong{\sphinxupquote{free\_vector}} (\sphinxstyleliteralemphasis{\sphinxupquote{np.array}}) \textendash{} Array with the freestream direction with respect to the Body frame

\item {} 
\sphinxAtStartPar
\sphinxstyleliteralstrong{\sphinxupquote{npix}} (\sphinxstyleliteralemphasis{\sphinxupquote{int}}) \textendash{} Resolution of the matrix used for the facet projection methodology (pixels)

\end{itemize}

\sphinxlineitem{Returns}
\sphinxAtStartPar
\sphinxstylestrong{node\_points} \textendash{} Array of IDs of the visible nodes

\sphinxlineitem{Return type}
\sphinxAtStartPar
np.array

\end{description}\end{quote}

\end{fulllineitems}

\index{bridging() (in module aerothermo)@\spxentry{bridging()}\spxextra{in module aerothermo}}

\begin{fulllineitems}
\phantomsection\label{\detokenize{modules:aerothermo.bridging}}
\pysigstartsignatures
\pysiglinewithargsret{\sphinxcode{\sphinxupquote{aerothermo.}}\sphinxbfcode{\sphinxupquote{bridging}}}{\emph{\DUrole{n}{free}}, \emph{\DUrole{n}{Kn\_cont}}, \emph{\DUrole{n}{Kn\_free}}}{}
\pysigstopsignatures
\sphinxAtStartPar
Computation of the bridging factor for the aeordynamic computation
\begin{quote}\begin{description}
\sphinxlineitem{Parameters}\begin{itemize}
\item {} 
\sphinxAtStartPar
\sphinxstyleliteralstrong{\sphinxupquote{free}} (\sphinxstyleliteralemphasis{\sphinxupquote{Assembly.Freestream}}) \textendash{} Freestream object

\item {} 
\sphinxAtStartPar
\sphinxstyleliteralstrong{\sphinxupquote{Kn\_cont}} (\sphinxstyleliteralemphasis{\sphinxupquote{float}}) \textendash{} Knudsen limit for the continuum regime

\item {} 
\sphinxAtStartPar
\sphinxstyleliteralstrong{\sphinxupquote{Kn\_free}} (\sphinxstyleliteralemphasis{\sphinxupquote{float}}) \textendash{} Knudsen limit for the free\sphinxhyphen{}molecular regime

\end{itemize}

\sphinxlineitem{Returns}
\sphinxAtStartPar
\sphinxstylestrong{AeroBridge} \textendash{} Bridging factor

\sphinxlineitem{Return type}
\sphinxAtStartPar
float

\end{description}\end{quote}

\end{fulllineitems}

\index{compute\_aerodynamics() (in module aerothermo)@\spxentry{compute\_aerodynamics()}\spxextra{in module aerothermo}}

\begin{fulllineitems}
\phantomsection\label{\detokenize{modules:aerothermo.compute_aerodynamics}}
\pysigstartsignatures
\pysiglinewithargsret{\sphinxcode{\sphinxupquote{aerothermo.}}\sphinxbfcode{\sphinxupquote{compute\_aerodynamics}}}{\emph{\DUrole{n}{assembly}}, \emph{\DUrole{n}{obj}}, \emph{\DUrole{n}{index}}, \emph{\DUrole{n}{flow\_direction}}, \emph{\DUrole{n}{options}}}{}
\pysigstopsignatures
\sphinxAtStartPar
Low\sphinxhyphen{}fidelity computation of the aerodynamics (pressure, friction)
\begin{quote}\begin{description}
\sphinxlineitem{Parameters}\begin{itemize}
\item {} 
\sphinxAtStartPar
\sphinxstyleliteralstrong{\sphinxupquote{assembly}} ({\hyperref[\detokenize{modules:assembly.Assembly_list}]{\sphinxcrossref{\sphinxstyleliteralemphasis{\sphinxupquote{Assembly\_list}}}}}) \textendash{} Object of class Assembly\_list

\item {} 
\sphinxAtStartPar
\sphinxstyleliteralstrong{\sphinxupquote{obj}} ({\hyperref[\detokenize{modules:component.Component}]{\sphinxcrossref{\sphinxstyleliteralemphasis{\sphinxupquote{Component}}}}}) \textendash{} Object of class Component

\item {} 
\sphinxAtStartPar
\sphinxstyleliteralstrong{\sphinxupquote{index}} (\sphinxstyleliteralemphasis{\sphinxupquote{np.array}}\sphinxstyleliteralemphasis{\sphinxupquote{(}}\sphinxstyleliteralemphasis{\sphinxupquote{int}}\sphinxstyleliteralemphasis{\sphinxupquote{)}}) \textendash{} Indexing list indicating nodes facing the flow (backface culling)

\item {} 
\sphinxAtStartPar
\sphinxstyleliteralstrong{\sphinxupquote{flow\_direction}} (\sphinxstyleliteralemphasis{\sphinxupquote{np.array}}\sphinxstyleliteralemphasis{\sphinxupquote{(}}\sphinxstyleliteralemphasis{\sphinxupquote{float}}\sphinxstyleliteralemphasis{\sphinxupquote{)}}) \textendash{} Array indicating direction of the flow in the body frame

\item {} 
\sphinxAtStartPar
\sphinxstyleliteralstrong{\sphinxupquote{options}} ({\hyperref[\detokenize{modules:configuration.Options}]{\sphinxcrossref{\sphinxstyleliteralemphasis{\sphinxupquote{Options}}}}}) \textendash{} Object of class Options

\end{itemize}

\end{description}\end{quote}

\end{fulllineitems}

\index{compute\_aerothermodynamics() (in module aerothermo)@\spxentry{compute\_aerothermodynamics()}\spxextra{in module aerothermo}}

\begin{fulllineitems}
\phantomsection\label{\detokenize{modules:aerothermo.compute_aerothermodynamics}}
\pysigstartsignatures
\pysiglinewithargsret{\sphinxcode{\sphinxupquote{aerothermo.}}\sphinxbfcode{\sphinxupquote{compute\_aerothermodynamics}}}{\emph{\DUrole{n}{assembly}}, \emph{\DUrole{n}{obj}}, \emph{\DUrole{n}{index}}, \emph{\DUrole{n}{flow\_direction}}, \emph{\DUrole{n}{options}}}{}
\pysigstopsignatures
\sphinxAtStartPar
Low\sphinxhyphen{}fidelity computation of the aerothermodynamics (heat\sphinxhyphen{}flux)
\begin{quote}\begin{description}
\sphinxlineitem{Parameters}\begin{itemize}
\item {} 
\sphinxAtStartPar
\sphinxstyleliteralstrong{\sphinxupquote{assembly}} ({\hyperref[\detokenize{modules:assembly.Assembly_list}]{\sphinxcrossref{\sphinxstyleliteralemphasis{\sphinxupquote{Assembly\_list}}}}}) \textendash{} Object of class Assembly\_list

\item {} 
\sphinxAtStartPar
\sphinxstyleliteralstrong{\sphinxupquote{obj}} ({\hyperref[\detokenize{modules:component.Component}]{\sphinxcrossref{\sphinxstyleliteralemphasis{\sphinxupquote{Component}}}}}) \textendash{} Object of class Component

\item {} 
\sphinxAtStartPar
\sphinxstyleliteralstrong{\sphinxupquote{index}} (\sphinxstyleliteralemphasis{\sphinxupquote{np.array}}\sphinxstyleliteralemphasis{\sphinxupquote{(}}\sphinxstyleliteralemphasis{\sphinxupquote{int}}\sphinxstyleliteralemphasis{\sphinxupquote{)}}) \textendash{} Indexing list indicating nodes facing the flow (backface culling)

\item {} 
\sphinxAtStartPar
\sphinxstyleliteralstrong{\sphinxupquote{flow\_direction}} (\sphinxstyleliteralemphasis{\sphinxupquote{np.array}}\sphinxstyleliteralemphasis{\sphinxupquote{(}}\sphinxstyleliteralemphasis{\sphinxupquote{float}}\sphinxstyleliteralemphasis{\sphinxupquote{)}}) \textendash{} Array indicating direction of the flow in the body frame

\item {} 
\sphinxAtStartPar
\sphinxstyleliteralstrong{\sphinxupquote{options}} ({\hyperref[\detokenize{modules:configuration.Options}]{\sphinxcrossref{\sphinxstyleliteralemphasis{\sphinxupquote{Options}}}}}) \textendash{} Object of class Options

\end{itemize}

\end{description}\end{quote}

\end{fulllineitems}

\index{aerodynamics\_module\_continuum() (in module aerothermo)@\spxentry{aerodynamics\_module\_continuum()}\spxextra{in module aerothermo}}

\begin{fulllineitems}
\phantomsection\label{\detokenize{modules:aerothermo.aerodynamics_module_continuum}}
\pysigstartsignatures
\pysiglinewithargsret{\sphinxcode{\sphinxupquote{aerothermo.}}\sphinxbfcode{\sphinxupquote{aerodynamics\_module\_continuum}}}{\emph{\DUrole{n}{nodes\_normal}}, \emph{\DUrole{n}{free}}, \emph{\DUrole{n}{p}}, \emph{\DUrole{n}{flow\_direction}}}{}
\pysigstopsignatures
\sphinxAtStartPar
Pressure computation for continuum regime

\sphinxAtStartPar
Function uses the Modified Newtonian Theory
\begin{quote}\begin{description}
\sphinxlineitem{Parameters}\begin{itemize}
\item {} 
\sphinxAtStartPar
\sphinxstyleliteralstrong{\sphinxupquote{nodes\_normal}} (\sphinxstyleliteralemphasis{\sphinxupquote{np.array}}) \textendash{} List of the normals of each vertex on the surface

\item {} 
\sphinxAtStartPar
\sphinxstyleliteralstrong{\sphinxupquote{free}} (\sphinxstyleliteralemphasis{\sphinxupquote{Assembly.Freestream}}) \textendash{} Freestream object

\item {} 
\sphinxAtStartPar
\sphinxstyleliteralstrong{\sphinxupquote{p}} (\sphinxstyleliteralemphasis{\sphinxupquote{np.array}}) \textendash{} List of vertex IDs that are visible to the flow

\item {} 
\sphinxAtStartPar
\sphinxstyleliteralstrong{\sphinxupquote{flow\_direction}} (\sphinxstyleliteralemphasis{\sphinxupquote{np.array}}) \textendash{} Vector containing the flow\_direction in the Body frame

\end{itemize}

\sphinxlineitem{Returns}
\sphinxAtStartPar
\sphinxstylestrong{Pressure} \textendash{} Vector with pressure values

\sphinxlineitem{Return type}
\sphinxAtStartPar
np.array

\end{description}\end{quote}

\end{fulllineitems}

\index{aerodynamics\_module\_bridging() (in module aerothermo)@\spxentry{aerodynamics\_module\_bridging()}\spxextra{in module aerothermo}}

\begin{fulllineitems}
\phantomsection\label{\detokenize{modules:aerothermo.aerodynamics_module_bridging}}
\pysigstartsignatures
\pysiglinewithargsret{\sphinxcode{\sphinxupquote{aerothermo.}}\sphinxbfcode{\sphinxupquote{aerodynamics\_module\_bridging}}}{\emph{\DUrole{n}{nodes\_normal}}, \emph{\DUrole{n}{free}}, \emph{\DUrole{n}{p}}, \emph{\DUrole{n}{aerobridge}}, \emph{\DUrole{n}{flow\_direction}}, \emph{\DUrole{n}{wall\_temperature}}}{}
\pysigstopsignatures
\sphinxAtStartPar
Pressure computation for Transitional regime
\begin{quote}\begin{description}
\sphinxlineitem{Parameters}\begin{itemize}
\item {} 
\sphinxAtStartPar
\sphinxstyleliteralstrong{\sphinxupquote{nodes\_normal}} (\sphinxstyleliteralemphasis{\sphinxupquote{np.array}}) \textendash{} List of the normals of each vertex on the surface

\item {} 
\sphinxAtStartPar
\sphinxstyleliteralstrong{\sphinxupquote{free}} (\sphinxstyleliteralemphasis{\sphinxupquote{Assembly.Freestream}}) \textendash{} Freestream object

\item {} 
\sphinxAtStartPar
\sphinxstyleliteralstrong{\sphinxupquote{p}} (\sphinxstyleliteralemphasis{\sphinxupquote{np.array}}) \textendash{} List of vertex IDs that are visible to the flow

\item {} 
\sphinxAtStartPar
\sphinxstyleliteralstrong{\sphinxupquote{aerobridge}} (\sphinxstyleliteralemphasis{\sphinxupquote{float}}) \textendash{} Bridging value between 0 and 1

\item {} 
\sphinxAtStartPar
\sphinxstyleliteralstrong{\sphinxupquote{flow\_direction}} (\sphinxstyleliteralemphasis{\sphinxupquote{np.array}}) \textendash{} Vector containing the flow\_direction in the Body frame

\item {} 
\sphinxAtStartPar
\sphinxstyleliteralstrong{\sphinxupquote{body\_temperature}} (\sphinxstyleliteralemphasis{\sphinxupquote{float}}) \textendash{} Temperature of the body

\end{itemize}

\sphinxlineitem{Returns}
\sphinxAtStartPar
\begin{itemize}
\item {} 
\sphinxAtStartPar
\sphinxstylestrong{Pressure} (\sphinxstyleemphasis{np.array}) \textendash{} Vector with pressure values

\item {} 
\sphinxAtStartPar
\sphinxstylestrong{Shear} (\sphinxstyleemphasis{np.array}) \textendash{} Vector with skin friction values

\end{itemize}


\end{description}\end{quote}

\end{fulllineitems}

\index{aerodynamics\_module\_freemolecular() (in module aerothermo)@\spxentry{aerodynamics\_module\_freemolecular()}\spxextra{in module aerothermo}}

\begin{fulllineitems}
\phantomsection\label{\detokenize{modules:aerothermo.aerodynamics_module_freemolecular}}
\pysigstartsignatures
\pysiglinewithargsret{\sphinxcode{\sphinxupquote{aerothermo.}}\sphinxbfcode{\sphinxupquote{aerodynamics\_module\_freemolecular}}}{\emph{\DUrole{n}{nodes\_normal}}, \emph{\DUrole{n}{free}}, \emph{\DUrole{n}{p}}, \emph{\DUrole{n}{flow\_direction}}, \emph{\DUrole{n}{body\_temperature}}}{}
\pysigstopsignatures
\sphinxAtStartPar
Pressure computation for Free\sphinxhyphen{}molecular regime

\sphinxAtStartPar
Function uses the Schaaf and Chambre theory
\begin{quote}\begin{description}
\sphinxlineitem{Parameters}\begin{itemize}
\item {} 
\sphinxAtStartPar
\sphinxstyleliteralstrong{\sphinxupquote{nodes\_normal}} (\sphinxstyleliteralemphasis{\sphinxupquote{np.array}}) \textendash{} List of the normals of each vertex on the surface

\item {} 
\sphinxAtStartPar
\sphinxstyleliteralstrong{\sphinxupquote{free}} (\sphinxstyleliteralemphasis{\sphinxupquote{Assembly.Freestream}}) \textendash{} Freestream object

\item {} 
\sphinxAtStartPar
\sphinxstyleliteralstrong{\sphinxupquote{p}} (\sphinxstyleliteralemphasis{\sphinxupquote{np.array}}) \textendash{} List of vertex IDs that are visible to the flow

\item {} 
\sphinxAtStartPar
\sphinxstyleliteralstrong{\sphinxupquote{flow\_direction}} (\sphinxstyleliteralemphasis{\sphinxupquote{np.array}}) \textendash{} Vector containing the flow\_direction in the Body frame

\item {} 
\sphinxAtStartPar
\sphinxstyleliteralstrong{\sphinxupquote{body\_temperature}} (\sphinxstyleliteralemphasis{\sphinxupquote{float}}) \textendash{} Temperature of the body

\end{itemize}

\sphinxlineitem{Returns}
\sphinxAtStartPar
\begin{itemize}
\item {} 
\sphinxAtStartPar
\sphinxstylestrong{Pressure} (\sphinxstyleemphasis{np.array}) \textendash{} Vector with pressure values

\item {} 
\sphinxAtStartPar
\sphinxstylestrong{Shear} (\sphinxstyleemphasis{np.array}) \textendash{} Vector with skin friction values

\end{itemize}


\end{description}\end{quote}

\end{fulllineitems}

\index{aerothermodynamics\_module\_continuum() (in module aerothermo)@\spxentry{aerothermodynamics\_module\_continuum()}\spxextra{in module aerothermo}}

\begin{fulllineitems}
\phantomsection\label{\detokenize{modules:aerothermo.aerothermodynamics_module_continuum}}
\pysigstartsignatures
\pysiglinewithargsret{\sphinxcode{\sphinxupquote{aerothermo.}}\sphinxbfcode{\sphinxupquote{aerothermodynamics\_module\_continuum}}}{\emph{\DUrole{n}{nodes\_normal}}, \emph{\DUrole{n}{nodes\_radius}}, \emph{\DUrole{n}{free}}, \emph{\DUrole{n}{p}}, \emph{\DUrole{n}{body\_temperature}}, \emph{\DUrole{n}{flow\_direction}}, \emph{\DUrole{n}{hf\_model}}}{}
\pysigstopsignatures
\sphinxAtStartPar
Heatflux computation for continuum regime

\sphinxAtStartPar
Function uses the Scarab equation (sc) or the Van Driest equation (vd)
\begin{quote}\begin{description}
\sphinxlineitem{Parameters}\begin{itemize}
\item {} 
\sphinxAtStartPar
\sphinxstyleliteralstrong{\sphinxupquote{nodes\_normal}} (\sphinxstyleliteralemphasis{\sphinxupquote{np.array}}) \textendash{} List of the normals of each vertex on the surface

\item {} 
\sphinxAtStartPar
\sphinxstyleliteralstrong{\sphinxupquote{nodes\_radius}} (\sphinxstyleliteralemphasis{\sphinxupquote{np.array}}) \textendash{} Local radius of each vertex

\item {} 
\sphinxAtStartPar
\sphinxstyleliteralstrong{\sphinxupquote{free}} (\sphinxstyleliteralemphasis{\sphinxupquote{Assembly.Freestream}}) \textendash{} Freestream object

\item {} 
\sphinxAtStartPar
\sphinxstyleliteralstrong{\sphinxupquote{p}} (\sphinxstyleliteralemphasis{\sphinxupquote{np.array}}) \textendash{} List of vertex IDs that are visible to the flow

\item {} 
\sphinxAtStartPar
\sphinxstyleliteralstrong{\sphinxupquote{body\_temperature}} (\sphinxstyleliteralemphasis{\sphinxupquote{float}}) \textendash{} Temperature of the body

\item {} 
\sphinxAtStartPar
\sphinxstyleliteralstrong{\sphinxupquote{flow\_direction}} (\sphinxstyleliteralemphasis{\sphinxupquote{np.array}}) \textendash{} Vector containing the flow\_direction in the Body frame

\item {} 
\sphinxAtStartPar
\sphinxstyleliteralstrong{\sphinxupquote{hf\_model}} (\sphinxstyleliteralemphasis{\sphinxupquote{str}}) \textendash{} Heatflux model to be used (default = ??, sc = Scarab, vd = Van Driest)

\end{itemize}

\sphinxlineitem{Returns}
\sphinxAtStartPar
\sphinxstylestrong{Stc} \textendash{} Vector with Stanton number

\sphinxlineitem{Return type}
\sphinxAtStartPar
np.array

\end{description}\end{quote}

\end{fulllineitems}

\index{aerothermodynamics\_module\_bridging() (in module aerothermo)@\spxentry{aerothermodynamics\_module\_bridging()}\spxextra{in module aerothermo}}

\begin{fulllineitems}
\phantomsection\label{\detokenize{modules:aerothermo.aerothermodynamics_module_bridging}}
\pysigstartsignatures
\pysiglinewithargsret{\sphinxcode{\sphinxupquote{aerothermo.}}\sphinxbfcode{\sphinxupquote{aerothermodynamics\_module\_bridging}}}{\emph{\DUrole{n}{nodes\_normal}}, \emph{\DUrole{n}{nodes\_radius}}, \emph{\DUrole{n}{free}}, \emph{\DUrole{n}{p}}, \emph{\DUrole{n}{wall\_temperature}}, \emph{\DUrole{n}{flow\_direction}}, \emph{\DUrole{n}{atm\_data}}, \emph{\DUrole{n}{hf\_model}}, \emph{\DUrole{n}{Kn\_cont}}, \emph{\DUrole{n}{Kn\_free}}, \emph{\DUrole{n}{lref}}, \emph{\DUrole{n}{assembly}}, \emph{\DUrole{n}{options}}}{}
\pysigstopsignatures
\sphinxAtStartPar
Heatflux computation for the heat\sphinxhyphen{}flux regime
\begin{quote}\begin{description}
\sphinxlineitem{Parameters}\begin{itemize}
\item {} 
\sphinxAtStartPar
\sphinxstyleliteralstrong{\sphinxupquote{nodes\_normal}} (\sphinxstyleliteralemphasis{\sphinxupquote{np.array}}) \textendash{} List of the normals of each vertex on the surface

\item {} 
\sphinxAtStartPar
\sphinxstyleliteralstrong{\sphinxupquote{nodes\_radius}} (\sphinxstyleliteralemphasis{\sphinxupquote{np.array}}) \textendash{} Local radius of each vertex

\item {} 
\sphinxAtStartPar
\sphinxstyleliteralstrong{\sphinxupquote{free}} (\sphinxstyleliteralemphasis{\sphinxupquote{Assembly.Freestream}}) \textendash{} Freestream object

\item {} 
\sphinxAtStartPar
\sphinxstyleliteralstrong{\sphinxupquote{p}} (\sphinxstyleliteralemphasis{\sphinxupquote{np.array}}) \textendash{} List of vertex IDs that are visible to the flow

\item {} 
\sphinxAtStartPar
\sphinxstyleliteralstrong{\sphinxupquote{wall\_temperature}} (\sphinxstyleliteralemphasis{\sphinxupquote{float}}) \textendash{} Temperature of the body

\item {} 
\sphinxAtStartPar
\sphinxstyleliteralstrong{\sphinxupquote{flow\_direction}} (\sphinxstyleliteralemphasis{\sphinxupquote{np.array}}) \textendash{} Vector containing the flow\_direction in the Body frame

\item {} 
\sphinxAtStartPar
\sphinxstyleliteralstrong{\sphinxupquote{atm\_data}} (\sphinxstyleliteralemphasis{\sphinxupquote{str}}) \textendash{} Atmospheric model

\item {} 
\sphinxAtStartPar
\sphinxstyleliteralstrong{\sphinxupquote{hf\_model}} (\sphinxstyleliteralemphasis{\sphinxupquote{str}}) \textendash{} Heatflux model to be used (default = ??, sc = Scarab, vd = Van Driest)

\item {} 
\sphinxAtStartPar
\sphinxstyleliteralstrong{\sphinxupquote{Kn\_cont}} (\sphinxstyleliteralemphasis{\sphinxupquote{float}}) \textendash{} Knudsen limit for the continuum regime

\item {} 
\sphinxAtStartPar
\sphinxstyleliteralstrong{\sphinxupquote{Kn\_free}} (\sphinxstyleliteralemphasis{\sphinxupquote{float}}) \textendash{} Knudsen limit for the free\sphinxhyphen{}molecular regime

\item {} 
\sphinxAtStartPar
\sphinxstyleliteralstrong{\sphinxupquote{lref}} (\sphinxstyleliteralemphasis{\sphinxupquote{float}}) \textendash{} Reference length

\item {} 
\sphinxAtStartPar
\sphinxstyleliteralstrong{\sphinxupquote{options}} ({\hyperref[\detokenize{modules:configuration.Options}]{\sphinxcrossref{\sphinxstyleliteralemphasis{\sphinxupquote{Options}}}}}) \textendash{} Object of class Options

\end{itemize}

\sphinxlineitem{Returns}
\sphinxAtStartPar
\sphinxstylestrong{St} \textendash{} Vector with Stanton number

\sphinxlineitem{Return type}
\sphinxAtStartPar
np.array

\end{description}\end{quote}

\end{fulllineitems}

\index{aerothermodynamics\_module\_freemolecular() (in module aerothermo)@\spxentry{aerothermodynamics\_module\_freemolecular()}\spxextra{in module aerothermo}}

\begin{fulllineitems}
\phantomsection\label{\detokenize{modules:aerothermo.aerothermodynamics_module_freemolecular}}
\pysigstartsignatures
\pysiglinewithargsret{\sphinxcode{\sphinxupquote{aerothermo.}}\sphinxbfcode{\sphinxupquote{aerothermodynamics\_module\_freemolecular}}}{\emph{\DUrole{n}{nodes\_normal}}, \emph{\DUrole{n}{free}}, \emph{\DUrole{n}{p}}, \emph{\DUrole{n}{flow\_direction}}, \emph{\DUrole{n}{Wall\_Temperature}}}{}
\pysigstopsignatures
\sphinxAtStartPar
Heatflux computation for free\sphinxhyphen{}molecular regime

\sphinxAtStartPar
Function uses the Schaaf and Chambre Theory
Based on book of Wallace Hayes \sphinxhyphen{} Hypersonic Flow Theory
\begin{quote}\begin{description}
\sphinxlineitem{Parameters}\begin{itemize}
\item {} 
\sphinxAtStartPar
\sphinxstyleliteralstrong{\sphinxupquote{nodes\_normal}} (\sphinxstyleliteralemphasis{\sphinxupquote{np.array}}) \textendash{} List of the normals of each vertex on the surface

\item {} 
\sphinxAtStartPar
\sphinxstyleliteralstrong{\sphinxupquote{free}} (\sphinxstyleliteralemphasis{\sphinxupquote{Assembly.Freestream}}) \textendash{} Freestream object

\item {} 
\sphinxAtStartPar
\sphinxstyleliteralstrong{\sphinxupquote{p}} (\sphinxstyleliteralemphasis{\sphinxupquote{np.array}}) \textendash{} List of vertex IDs that are visible to the flow

\item {} 
\sphinxAtStartPar
\sphinxstyleliteralstrong{\sphinxupquote{Wall\_temperature}} (\sphinxstyleliteralemphasis{\sphinxupquote{float}}) \textendash{} Temperature of the body

\item {} 
\sphinxAtStartPar
\sphinxstyleliteralstrong{\sphinxupquote{flow\_direction}} (\sphinxstyleliteralemphasis{\sphinxupquote{np.array}}) \textendash{} Vector containing the flow\_direction in the Body frame

\end{itemize}

\sphinxlineitem{Returns}
\sphinxAtStartPar
\sphinxstylestrong{Stfm} \textendash{} Vector with Stanton number

\sphinxlineitem{Return type}
\sphinxAtStartPar
np.array

\end{description}\end{quote}

\end{fulllineitems}



\section{Dynamics}
\label{\detokenize{modules:dynamics}}\index{DerivativesAngle (class in dynamics)@\spxentry{DerivativesAngle}\spxextra{class in dynamics}}

\begin{fulllineitems}
\phantomsection\label{\detokenize{modules:dynamics.DerivativesAngle}}
\pysigstartsignatures
\pysiglinewithargsret{\sphinxbfcode{\sphinxupquote{class\DUrole{w}{  }}}\sphinxcode{\sphinxupquote{dynamics.}}\sphinxbfcode{\sphinxupquote{DerivativesAngle}}}{\emph{\DUrole{n}{droll}\DUrole{o}{=}\DUrole{default_value}{0}}, \emph{\DUrole{n}{dpitch}\DUrole{o}{=}\DUrole{default_value}{0}}, \emph{\DUrole{n}{dyaw}\DUrole{o}{=}\DUrole{default_value}{0}}, \emph{\DUrole{n}{ddroll}\DUrole{o}{=}\DUrole{default_value}{0}}, \emph{\DUrole{n}{ddpitch}\DUrole{o}{=}\DUrole{default_value}{0}}, \emph{\DUrole{n}{ddyaw}\DUrole{o}{=}\DUrole{default_value}{0}}}{}
\pysigstopsignatures
\sphinxAtStartPar
Class DerivativesAngle

\sphinxAtStartPar
A class to store the derivatives information regarding the angular dynamics in the body frame
\index{ddpitch (dynamics.DerivativesAngle attribute)@\spxentry{ddpitch}\spxextra{dynamics.DerivativesAngle attribute}}

\begin{fulllineitems}
\phantomsection\label{\detokenize{modules:dynamics.DerivativesAngle.ddpitch}}
\pysigstartsignatures
\pysigline{\sphinxbfcode{\sphinxupquote{ddpitch}}}
\pysigstopsignatures
\sphinxAtStartPar
{[}float{]} Second derivative of the pitch angle

\end{fulllineitems}

\index{ddroll (dynamics.DerivativesAngle attribute)@\spxentry{ddroll}\spxextra{dynamics.DerivativesAngle attribute}}

\begin{fulllineitems}
\phantomsection\label{\detokenize{modules:dynamics.DerivativesAngle.ddroll}}
\pysigstartsignatures
\pysigline{\sphinxbfcode{\sphinxupquote{ddroll}}}
\pysigstopsignatures
\sphinxAtStartPar
{[}float{]} Second derivative of the roll angle

\end{fulllineitems}

\index{ddyaw (dynamics.DerivativesAngle attribute)@\spxentry{ddyaw}\spxextra{dynamics.DerivativesAngle attribute}}

\begin{fulllineitems}
\phantomsection\label{\detokenize{modules:dynamics.DerivativesAngle.ddyaw}}
\pysigstartsignatures
\pysigline{\sphinxbfcode{\sphinxupquote{ddyaw}}}
\pysigstopsignatures
\sphinxAtStartPar
{[}float{]} Second derivative of the yaw angle

\end{fulllineitems}

\index{dpitch (dynamics.DerivativesAngle attribute)@\spxentry{dpitch}\spxextra{dynamics.DerivativesAngle attribute}}

\begin{fulllineitems}
\phantomsection\label{\detokenize{modules:dynamics.DerivativesAngle.dpitch}}
\pysigstartsignatures
\pysigline{\sphinxbfcode{\sphinxupquote{dpitch}}}
\pysigstopsignatures
\sphinxAtStartPar
{[}float{]} Derivative of the pitch angle

\end{fulllineitems}

\index{droll (dynamics.DerivativesAngle attribute)@\spxentry{droll}\spxextra{dynamics.DerivativesAngle attribute}}

\begin{fulllineitems}
\phantomsection\label{\detokenize{modules:dynamics.DerivativesAngle.droll}}
\pysigstartsignatures
\pysigline{\sphinxbfcode{\sphinxupquote{droll}}}
\pysigstopsignatures
\sphinxAtStartPar
{[}float{]} Derivative of the roll angle

\end{fulllineitems}

\index{dyaw (dynamics.DerivativesAngle attribute)@\spxentry{dyaw}\spxextra{dynamics.DerivativesAngle attribute}}

\begin{fulllineitems}
\phantomsection\label{\detokenize{modules:dynamics.DerivativesAngle.dyaw}}
\pysigstartsignatures
\pysigline{\sphinxbfcode{\sphinxupquote{dyaw}}}
\pysigstopsignatures
\sphinxAtStartPar
{[}float{]} Derivative of the yaw angle

\end{fulllineitems}


\end{fulllineitems}

\index{DerivativesCartesian (class in dynamics)@\spxentry{DerivativesCartesian}\spxextra{class in dynamics}}

\begin{fulllineitems}
\phantomsection\label{\detokenize{modules:dynamics.DerivativesCartesian}}
\pysigstartsignatures
\pysiglinewithargsret{\sphinxbfcode{\sphinxupquote{class\DUrole{w}{  }}}\sphinxcode{\sphinxupquote{dynamics.}}\sphinxbfcode{\sphinxupquote{DerivativesCartesian}}}{\emph{\DUrole{n}{dx}\DUrole{o}{=}\DUrole{default_value}{0}}, \emph{\DUrole{n}{dy}\DUrole{o}{=}\DUrole{default_value}{0}}, \emph{\DUrole{n}{dz}\DUrole{o}{=}\DUrole{default_value}{0}}, \emph{\DUrole{n}{du}\DUrole{o}{=}\DUrole{default_value}{0}}, \emph{\DUrole{n}{dv}\DUrole{o}{=}\DUrole{default_value}{0}}, \emph{\DUrole{n}{dw}\DUrole{o}{=}\DUrole{default_value}{0}}}{}
\pysigstopsignatures
\sphinxAtStartPar
Class DerivativesCartesian

\sphinxAtStartPar
A class to store the derivatives information of position and velocity in the cartesian (ECEF) frame
\index{du (dynamics.DerivativesCartesian attribute)@\spxentry{du}\spxextra{dynamics.DerivativesCartesian attribute}}

\begin{fulllineitems}
\phantomsection\label{\detokenize{modules:dynamics.DerivativesCartesian.du}}
\pysigstartsignatures
\pysigline{\sphinxbfcode{\sphinxupquote{du}}}
\pysigstopsignatures
\sphinxAtStartPar
{[}float{]} Derivative of the X\sphinxhyphen{}velocity

\end{fulllineitems}

\index{dv (dynamics.DerivativesCartesian attribute)@\spxentry{dv}\spxextra{dynamics.DerivativesCartesian attribute}}

\begin{fulllineitems}
\phantomsection\label{\detokenize{modules:dynamics.DerivativesCartesian.dv}}
\pysigstartsignatures
\pysigline{\sphinxbfcode{\sphinxupquote{dv}}}
\pysigstopsignatures
\sphinxAtStartPar
{[}float{]} Derivative of the Y\sphinxhyphen{}velocity

\end{fulllineitems}

\index{dw (dynamics.DerivativesCartesian attribute)@\spxentry{dw}\spxextra{dynamics.DerivativesCartesian attribute}}

\begin{fulllineitems}
\phantomsection\label{\detokenize{modules:dynamics.DerivativesCartesian.dw}}
\pysigstartsignatures
\pysigline{\sphinxbfcode{\sphinxupquote{dw}}}
\pysigstopsignatures
\sphinxAtStartPar
{[}float{]} Derivative of the Z\sphinxhyphen{}velocity

\end{fulllineitems}

\index{dx (dynamics.DerivativesCartesian attribute)@\spxentry{dx}\spxextra{dynamics.DerivativesCartesian attribute}}

\begin{fulllineitems}
\phantomsection\label{\detokenize{modules:dynamics.DerivativesCartesian.dx}}
\pysigstartsignatures
\pysigline{\sphinxbfcode{\sphinxupquote{dx}}}
\pysigstopsignatures
\sphinxAtStartPar
{[}float{]} Derivative of the X\sphinxhyphen{}position

\end{fulllineitems}

\index{dy (dynamics.DerivativesCartesian attribute)@\spxentry{dy}\spxextra{dynamics.DerivativesCartesian attribute}}

\begin{fulllineitems}
\phantomsection\label{\detokenize{modules:dynamics.DerivativesCartesian.dy}}
\pysigstartsignatures
\pysigline{\sphinxbfcode{\sphinxupquote{dy}}}
\pysigstopsignatures
\sphinxAtStartPar
{[}float{]} Derivative of the Y\sphinxhyphen{}position

\end{fulllineitems}

\index{dz (dynamics.DerivativesCartesian attribute)@\spxentry{dz}\spxextra{dynamics.DerivativesCartesian attribute}}

\begin{fulllineitems}
\phantomsection\label{\detokenize{modules:dynamics.DerivativesCartesian.dz}}
\pysigstartsignatures
\pysigline{\sphinxbfcode{\sphinxupquote{dz}}}
\pysigstopsignatures
\sphinxAtStartPar
{[}float{]} Derivative of the Z\sphinxhyphen{}position

\end{fulllineitems}


\end{fulllineitems}

\index{integrate() (in module dynamics)@\spxentry{integrate()}\spxextra{in module dynamics}}

\begin{fulllineitems}
\phantomsection\label{\detokenize{modules:dynamics.integrate}}
\pysigstartsignatures
\pysiglinewithargsret{\sphinxcode{\sphinxupquote{dynamics.}}\sphinxbfcode{\sphinxupquote{integrate}}}{\emph{\DUrole{n}{titan}}, \emph{\DUrole{n}{options}}}{}
\pysigstopsignatures
\sphinxAtStartPar
Time integration

\sphinxAtStartPar
This function calls a time integration scheme
\begin{quote}\begin{description}
\sphinxlineitem{Parameters}\begin{itemize}
\item {} 
\sphinxAtStartPar
\sphinxstyleliteralstrong{\sphinxupquote{titan}} ({\hyperref[\detokenize{modules:assembly.Assembly_list}]{\sphinxcrossref{\sphinxstyleliteralemphasis{\sphinxupquote{Assembly\_list}}}}}) \textendash{} Object of class Assembly\_list

\item {} 
\sphinxAtStartPar
\sphinxstyleliteralstrong{\sphinxupquote{options}} ({\hyperref[\detokenize{modules:configuration.Options}]{\sphinxcrossref{\sphinxstyleliteralemphasis{\sphinxupquote{Options}}}}}) \textendash{} Object of class Options

\end{itemize}

\end{description}\end{quote}

\end{fulllineitems}

\index{compute\_angular\_derivatives() (in module dynamics)@\spxentry{compute\_angular\_derivatives()}\spxextra{in module dynamics}}

\begin{fulllineitems}
\phantomsection\label{\detokenize{modules:dynamics.compute_angular_derivatives}}
\pysigstartsignatures
\pysiglinewithargsret{\sphinxcode{\sphinxupquote{dynamics.}}\sphinxbfcode{\sphinxupquote{compute\_angular\_derivatives}}}{\emph{\DUrole{n}{assembly}}}{}
\pysigstopsignatures
\sphinxAtStartPar
Computation of the angular derivatives in the Body frame

\sphinxAtStartPar
This function computes the angular dericatives taking into consideration the euler and aerodynamic moments
\begin{quote}\begin{description}
\sphinxlineitem{Parameters}
\sphinxAtStartPar
\sphinxstyleliteralstrong{\sphinxupquote{assembly}} ({\hyperref[\detokenize{modules:assembly.Assembly}]{\sphinxcrossref{\sphinxstyleliteralemphasis{\sphinxupquote{Assembly}}}}}) \textendash{} Object of Assembly class

\end{description}\end{quote}

\end{fulllineitems}

\index{compute\_cartesian\_derivatives() (in module dynamics)@\spxentry{compute\_cartesian\_derivatives()}\spxextra{in module dynamics}}

\begin{fulllineitems}
\phantomsection\label{\detokenize{modules:dynamics.compute_cartesian_derivatives}}
\pysigstartsignatures
\pysiglinewithargsret{\sphinxcode{\sphinxupquote{dynamics.}}\sphinxbfcode{\sphinxupquote{compute\_cartesian\_derivatives}}}{\emph{\DUrole{n}{assembly}}, \emph{\DUrole{n}{options}}}{}
\pysigstopsignatures
\sphinxAtStartPar
Computation of the cartesian derivatives

\sphinxAtStartPar
This function computes the cartesian derivatives of the position and velocity
It uses the gravity, aerodynamic, centrifugal and coriolis forces for the acceleration computation.
\begin{quote}\begin{description}
\sphinxlineitem{Parameters}\begin{itemize}
\item {} 
\sphinxAtStartPar
\sphinxstyleliteralstrong{\sphinxupquote{assembly}} ({\hyperref[\detokenize{modules:assembly.Assembly}]{\sphinxcrossref{\sphinxstyleliteralemphasis{\sphinxupquote{Assembly}}}}}) \textendash{} Object of class Assembly

\item {} 
\sphinxAtStartPar
\sphinxstyleliteralstrong{\sphinxupquote{options}} ({\hyperref[\detokenize{modules:configuration.Options}]{\sphinxcrossref{\sphinxstyleliteralemphasis{\sphinxupquote{Options}}}}}) \textendash{} Object of class Options

\end{itemize}

\end{description}\end{quote}

\end{fulllineitems}

\index{compute\_cartesian() (in module dynamics)@\spxentry{compute\_cartesian()}\spxextra{in module dynamics}}

\begin{fulllineitems}
\phantomsection\label{\detokenize{modules:dynamics.compute_cartesian}}
\pysigstartsignatures
\pysiglinewithargsret{\sphinxcode{\sphinxupquote{dynamics.}}\sphinxbfcode{\sphinxupquote{compute\_cartesian}}}{\emph{\DUrole{n}{assembly}}, \emph{\DUrole{n}{options}}}{}
\pysigstopsignatures
\sphinxAtStartPar
Computation of the cartesian dynamics

\sphinxAtStartPar
This function computes the cartesian position and velocity of the assembly
\begin{quote}\begin{description}
\sphinxlineitem{Parameters}\begin{itemize}
\item {} 
\sphinxAtStartPar
\sphinxstyleliteralstrong{\sphinxupquote{assembly}} ({\hyperref[\detokenize{modules:assembly.Assembly}]{\sphinxcrossref{\sphinxstyleliteralemphasis{\sphinxupquote{Assembly}}}}}) \textendash{} Object of class Assembly

\item {} 
\sphinxAtStartPar
\sphinxstyleliteralstrong{\sphinxupquote{options}} ({\hyperref[\detokenize{modules:configuration.Options}]{\sphinxcrossref{\sphinxstyleliteralemphasis{\sphinxupquote{Options}}}}}) \textendash{} Object of class Options

\end{itemize}

\end{description}\end{quote}

\end{fulllineitems}

\index{compute\_quaternion() (in module dynamics)@\spxentry{compute\_quaternion()}\spxextra{in module dynamics}}

\begin{fulllineitems}
\phantomsection\label{\detokenize{modules:dynamics.compute_quaternion}}
\pysigstartsignatures
\pysiglinewithargsret{\sphinxcode{\sphinxupquote{dynamics.}}\sphinxbfcode{\sphinxupquote{compute\_quaternion}}}{\emph{\DUrole{n}{assembly}}}{}
\pysigstopsignatures
\sphinxAtStartPar
Computation of the quaternion

\sphinxAtStartPar
This function computes the quaternion value of the body frame with respect to the ECEF frame
The quaternion will give the rotation matrix that will allow to pass from Body to ECEF
\begin{quote}\begin{description}
\sphinxlineitem{Parameters}
\sphinxAtStartPar
\sphinxstyleliteralstrong{\sphinxupquote{assembly}} ({\hyperref[\detokenize{modules:assembly.Assembly}]{\sphinxcrossref{\sphinxstyleliteralemphasis{\sphinxupquote{Assembly}}}}}) \textendash{} Object of Assembly class

\end{description}\end{quote}

\end{fulllineitems}

\index{compute\_Euler() (in module euler)@\spxentry{compute\_Euler()}\spxextra{in module euler}}

\begin{fulllineitems}
\phantomsection\label{\detokenize{modules:euler.compute_Euler}}
\pysigstartsignatures
\pysiglinewithargsret{\sphinxcode{\sphinxupquote{euler.}}\sphinxbfcode{\sphinxupquote{compute\_Euler}}}{\emph{\DUrole{n}{titan}}, \emph{\DUrole{n}{options}}}{}
\pysigstopsignatures
\sphinxAtStartPar
Euler integration
\begin{quote}\begin{description}
\sphinxlineitem{Parameters}\begin{itemize}
\item {} 
\sphinxAtStartPar
\sphinxstyleliteralstrong{\sphinxupquote{titan}} ({\hyperref[\detokenize{modules:assembly.Assembly_list}]{\sphinxcrossref{\sphinxstyleliteralemphasis{\sphinxupquote{Assembly\_list}}}}}) \textendash{} Object of class Assembly\_list

\item {} 
\sphinxAtStartPar
\sphinxstyleliteralstrong{\sphinxupquote{options}} ({\hyperref[\detokenize{modules:configuration.Options}]{\sphinxcrossref{\sphinxstyleliteralemphasis{\sphinxupquote{Options}}}}}) \textendash{} Object of class Options

\end{itemize}

\end{description}\end{quote}

\end{fulllineitems}

\index{update\_position\_cartesian() (in module euler)@\spxentry{update\_position\_cartesian()}\spxextra{in module euler}}

\begin{fulllineitems}
\phantomsection\label{\detokenize{modules:euler.update_position_cartesian}}
\pysigstartsignatures
\pysiglinewithargsret{\sphinxcode{\sphinxupquote{euler.}}\sphinxbfcode{\sphinxupquote{update\_position\_cartesian}}}{\emph{\DUrole{n}{assembly}}, \emph{\DUrole{n}{cartesianDerivatives}}, \emph{\DUrole{n}{angularDerivatives}}, \emph{\DUrole{n}{options}}}{}
\pysigstopsignatures
\sphinxAtStartPar
Update position and attitude of the assembly
\begin{quote}\begin{description}
\sphinxlineitem{Parameters}\begin{itemize}
\item {} 
\sphinxAtStartPar
\sphinxstyleliteralstrong{\sphinxupquote{assembly}} ({\hyperref[\detokenize{modules:assembly.Assembly}]{\sphinxcrossref{\sphinxstyleliteralemphasis{\sphinxupquote{Assembly}}}}}) \textendash{} Object of class Assembly

\item {} 
\sphinxAtStartPar
\sphinxstyleliteralstrong{\sphinxupquote{cartesianDerivatives}} ({\hyperref[\detokenize{modules:dynamics.DerivativesCartesian}]{\sphinxcrossref{\sphinxstyleliteralemphasis{\sphinxupquote{DerivativesCartesian}}}}}) \textendash{} Object of class DerivativesCartesian

\item {} 
\sphinxAtStartPar
\sphinxstyleliteralstrong{\sphinxupquote{angularDerivatives}} ({\hyperref[\detokenize{modules:dynamics.DerivativesAngle}]{\sphinxcrossref{\sphinxstyleliteralemphasis{\sphinxupquote{DerivativesAngle}}}}}) \textendash{} Object of class DerivativesAngle

\item {} 
\sphinxAtStartPar
\sphinxstyleliteralstrong{\sphinxupquote{options}} ({\hyperref[\detokenize{modules:configuration.Options}]{\sphinxcrossref{\sphinxstyleliteralemphasis{\sphinxupquote{Options}}}}}) \textendash{} Object of class Options

\end{itemize}

\end{description}\end{quote}

\end{fulllineitems}



\section{Forces}
\label{\detokenize{modules:forces}}\index{compute\_aerodynamic\_forces() (in module forces)@\spxentry{compute\_aerodynamic\_forces()}\spxextra{in module forces}}

\begin{fulllineitems}
\phantomsection\label{\detokenize{modules:forces.compute_aerodynamic_forces}}
\pysigstartsignatures
\pysiglinewithargsret{\sphinxcode{\sphinxupquote{forces.}}\sphinxbfcode{\sphinxupquote{compute\_aerodynamic\_forces}}}{\emph{\DUrole{n}{titan}}, \emph{\DUrole{n}{options}}}{}
\pysigstopsignatures
\sphinxAtStartPar
Computes the aerodynamic forces in the wind frame
\begin{quote}\begin{description}
\sphinxlineitem{Parameters}\begin{itemize}
\item {} 
\sphinxAtStartPar
\sphinxstyleliteralstrong{\sphinxupquote{titan}} ({\hyperref[\detokenize{modules:assembly.Assembly_list}]{\sphinxcrossref{\sphinxstyleliteralemphasis{\sphinxupquote{Assembly\_list}}}}}) \textendash{} Object of class Assembly\_list

\item {} 
\sphinxAtStartPar
\sphinxstyleliteralstrong{\sphinxupquote{options}} ({\hyperref[\detokenize{modules:configuration.Options}]{\sphinxcrossref{\sphinxstyleliteralemphasis{\sphinxupquote{Options}}}}}) \textendash{} Object of class Options

\end{itemize}

\end{description}\end{quote}

\end{fulllineitems}

\index{compute\_aerodynamic\_moments() (in module forces)@\spxentry{compute\_aerodynamic\_moments()}\spxextra{in module forces}}

\begin{fulllineitems}
\phantomsection\label{\detokenize{modules:forces.compute_aerodynamic_moments}}
\pysigstartsignatures
\pysiglinewithargsret{\sphinxcode{\sphinxupquote{forces.}}\sphinxbfcode{\sphinxupquote{compute\_aerodynamic\_moments}}}{\emph{\DUrole{n}{titan}}, \emph{\DUrole{n}{options}}}{}
\pysigstopsignatures
\sphinxAtStartPar
Computes the aerodynamic moments in the wind Body frame
\begin{quote}\begin{description}
\sphinxlineitem{Parameters}\begin{itemize}
\item {} 
\sphinxAtStartPar
\sphinxstyleliteralstrong{\sphinxupquote{titan}} ({\hyperref[\detokenize{modules:assembly.Assembly_list}]{\sphinxcrossref{\sphinxstyleliteralemphasis{\sphinxupquote{Assembly\_list}}}}}) \textendash{} Object of class Assembly\_list

\item {} 
\sphinxAtStartPar
\sphinxstyleliteralstrong{\sphinxupquote{options}} ({\hyperref[\detokenize{modules:configuration.Options}]{\sphinxcrossref{\sphinxstyleliteralemphasis{\sphinxupquote{Options}}}}}) \textendash{} Object of class Options

\end{itemize}

\end{description}\end{quote}

\end{fulllineitems}

\index{compute\_inertial\_forces() (in module forces)@\spxentry{compute\_inertial\_forces()}\spxextra{in module forces}}

\begin{fulllineitems}
\phantomsection\label{\detokenize{modules:forces.compute_inertial_forces}}
\pysigstartsignatures
\pysiglinewithargsret{\sphinxcode{\sphinxupquote{forces.}}\sphinxbfcode{\sphinxupquote{compute\_inertial\_forces}}}{\emph{\DUrole{n}{assembly}}, \emph{\DUrole{n}{options}}}{}
\pysigstopsignatures
\sphinxAtStartPar
Computes the inertial forces in the Body Frame

\sphinxAtStartPar
This functions computes the inertial forces that will be used for the Structurla dynamics
\begin{quote}\begin{description}
\sphinxlineitem{Parameters}\begin{itemize}
\item {} 
\sphinxAtStartPar
\sphinxstyleliteralstrong{\sphinxupquote{assembly}} ({\hyperref[\detokenize{modules:assembly.Assembly}]{\sphinxcrossref{\sphinxstyleliteralemphasis{\sphinxupquote{Assembly}}}}}) \textendash{} Object of class Assembly

\item {} 
\sphinxAtStartPar
\sphinxstyleliteralstrong{\sphinxupquote{options}} ({\hyperref[\detokenize{modules:configuration.Options}]{\sphinxcrossref{\sphinxstyleliteralemphasis{\sphinxupquote{Options}}}}}) \textendash{} Object of class Options

\end{itemize}

\end{description}\end{quote}

\end{fulllineitems}



\section{Fragmentation}
\label{\detokenize{modules:fragmentation}}\index{fragmentation() (in module fragmentation)@\spxentry{fragmentation()}\spxextra{in module fragmentation}}

\begin{fulllineitems}
\phantomsection\label{\detokenize{modules:fragmentation.fragmentation}}
\pysigstartsignatures
\pysiglinewithargsret{\sphinxcode{\sphinxupquote{fragmentation.}}\sphinxbfcode{\sphinxupquote{fragmentation}}}{\emph{\DUrole{n}{titan}}, \emph{\DUrole{n}{options}}}{}
\pysigstopsignatures
\sphinxAtStartPar
Check if components meet the specified criteria to be removed from the simulation.
At the moment, only altitude, iteration number, time and total ablation are specified.
\begin{quote}\begin{description}
\sphinxlineitem{Parameters}\begin{itemize}
\item {} 
\sphinxAtStartPar
\sphinxstyleliteralstrong{\sphinxupquote{titan}} ({\hyperref[\detokenize{modules:assembly.Assembly_list}]{\sphinxcrossref{\sphinxstyleliteralemphasis{\sphinxupquote{Assembly\_list}}}}}) \textendash{} Object of class Assembly\_list

\item {} 
\sphinxAtStartPar
\sphinxstyleliteralstrong{\sphinxupquote{options}} ({\hyperref[\detokenize{modules:configuration.Options}]{\sphinxcrossref{\sphinxstyleliteralemphasis{\sphinxupquote{Options}}}}}) \textendash{} Object of class Options

\end{itemize}

\end{description}\end{quote}

\end{fulllineitems}

\index{demise\_components() (in module fragmentation)@\spxentry{demise\_components()}\spxextra{in module fragmentation}}

\begin{fulllineitems}
\phantomsection\label{\detokenize{modules:fragmentation.demise_components}}
\pysigstartsignatures
\pysiglinewithargsret{\sphinxcode{\sphinxupquote{fragmentation.}}\sphinxbfcode{\sphinxupquote{demise\_components}}}{\emph{\DUrole{n}{titan}}, \emph{\DUrole{n}{assembly\_pos}}, \emph{\DUrole{n}{joints\_id}}, \emph{\DUrole{n}{options}}}{}
\pysigstopsignatures
\sphinxAtStartPar
Computes the inertial forces in the Body Frame

\sphinxAtStartPar
This functions computes the inertial forces that will be used for the Structurla dynamics
\begin{quote}\begin{description}
\sphinxlineitem{Parameters}\begin{itemize}
\item {} 
\sphinxAtStartPar
\sphinxstyleliteralstrong{\sphinxupquote{titan}} ({\hyperref[\detokenize{modules:assembly.Assembly_list}]{\sphinxcrossref{\sphinxstyleliteralemphasis{\sphinxupquote{Assembly\_list}}}}}) \textendash{} Object of class Assembly\_list

\item {} 
\sphinxAtStartPar
\sphinxstyleliteralstrong{\sphinxupquote{assembly\_pos}} (\sphinxstyleliteralemphasis{\sphinxupquote{array}}) \textendash{} Array containing the index position of the assemblies that will undergo fragmentation

\item {} 
\sphinxAtStartPar
\sphinxstyleliteralstrong{\sphinxupquote{joints\_id}} (\sphinxstyleliteralemphasis{\sphinxupquote{array}}) \textendash{} Array containing the index of the joints that demised (index in relation to each assembly that will undergo fragmentation), to be removed from the simulation

\item {} 
\sphinxAtStartPar
\sphinxstyleliteralstrong{\sphinxupquote{options}} ({\hyperref[\detokenize{modules:configuration.Options}]{\sphinxcrossref{\sphinxstyleliteralemphasis{\sphinxupquote{Options}}}}}) \textendash{} Object of class Options

\end{itemize}

\end{description}\end{quote}

\end{fulllineitems}



\section{Freestream}
\label{\detokenize{modules:freestream}}\index{load\_atmosphere() (in module atmosphere)@\spxentry{load\_atmosphere()}\spxextra{in module atmosphere}}

\begin{fulllineitems}
\phantomsection\label{\detokenize{modules:atmosphere.load_atmosphere}}
\pysigstartsignatures
\pysiglinewithargsret{\sphinxcode{\sphinxupquote{atmosphere.}}\sphinxbfcode{\sphinxupquote{load\_atmosphere}}}{\emph{\DUrole{n}{name}}}{}
\pysigstopsignatures
\sphinxAtStartPar
This function loads the atmosphere model with respect to the user specification
\begin{quote}\begin{description}
\sphinxlineitem{Parameters}
\sphinxAtStartPar
\sphinxstyleliteralstrong{\sphinxupquote{name}} (\sphinxstyleliteralemphasis{\sphinxupquote{str}}) \textendash{} Name of the atmospheric model

\sphinxlineitem{Returns}
\sphinxAtStartPar
\begin{itemize}
\item {} 
\sphinxAtStartPar
\sphinxstylestrong{f} (\sphinxstyleemphasis{scipy.interpolate.interp1d}) \textendash{} Function interpolation of the atmopshere atributes with respect to altitude

\item {} 
\sphinxAtStartPar
\sphinxstylestrong{spacies\_index} (\sphinxstyleemphasis{array}) \textendash{} Array with the species used in the model

\end{itemize}


\end{description}\end{quote}

\end{fulllineitems}

\index{mixture\_mpp() (in module mix\_mpp)@\spxentry{mixture\_mpp()}\spxextra{in module mix\_mpp}}

\begin{fulllineitems}
\phantomsection\label{\detokenize{modules:mix_mpp.mixture_mpp}}
\pysigstartsignatures
\pysiglinewithargsret{\sphinxcode{\sphinxupquote{mix\_mpp.}}\sphinxbfcode{\sphinxupquote{mixture\_mpp}}}{\emph{\DUrole{n}{species}}, \emph{\DUrole{n}{density}}, \emph{\DUrole{n}{temperature}}}{}
\pysigstopsignatures
\sphinxAtStartPar
Retrieve the mixture object of the Mutation++ library
\begin{quote}\begin{description}
\sphinxlineitem{Parameters}\begin{itemize}
\item {} 
\sphinxAtStartPar
\sphinxstyleliteralstrong{\sphinxupquote{species}} (\sphinxstyleliteralemphasis{\sphinxupquote{array}}) \textendash{} Species used for the mixture

\item {} 
\sphinxAtStartPar
\sphinxstyleliteralstrong{\sphinxupquote{density}} (\sphinxstyleliteralemphasis{\sphinxupquote{array}}) \textendash{} Density of each species

\item {} 
\sphinxAtStartPar
\sphinxstyleliteralstrong{\sphinxupquote{temperature}} (\sphinxstyleliteralemphasis{\sphinxupquote{float}}) \textendash{} Temperature of the mixture

\end{itemize}

\sphinxlineitem{Returns}
\sphinxAtStartPar
\sphinxstylestrong{mix} \textendash{} Object of the mpp Mixture

\sphinxlineitem{Return type}
\sphinxAtStartPar
mpp.Mixture()

\end{description}\end{quote}

\end{fulllineitems}

\index{compute\_freestream() (in module mix\_properties)@\spxentry{compute\_freestream()}\spxextra{in module mix\_properties}}

\begin{fulllineitems}
\phantomsection\label{\detokenize{modules:mix_properties.compute_freestream}}
\pysigstartsignatures
\pysiglinewithargsret{\sphinxcode{\sphinxupquote{mix\_properties.}}\sphinxbfcode{\sphinxupquote{compute\_freestream}}}{\emph{\DUrole{n}{model}}, \emph{\DUrole{n}{altitude}}, \emph{\DUrole{n}{velocity}}, \emph{\DUrole{n}{lref}}, \emph{\DUrole{n}{freestream}}, \emph{\DUrole{n}{assembly}}, \emph{\DUrole{n}{options}}}{}
\pysigstopsignatures
\sphinxAtStartPar
Compute the freestream properties

\sphinxAtStartPar
The user needs to specify the method for the freestream computation (Standard, Mutationpp)
\begin{quote}\begin{description}
\sphinxlineitem{Parameters}\begin{itemize}
\item {} 
\sphinxAtStartPar
\sphinxstyleliteralstrong{\sphinxupquote{model}} (\sphinxstyleliteralemphasis{\sphinxupquote{str}}) \textendash{} Name of the atmospheric model

\item {} 
\sphinxAtStartPar
\sphinxstyleliteralstrong{\sphinxupquote{altitude}} (\sphinxstyleliteralemphasis{\sphinxupquote{float}}) \textendash{} Altitude value in meters

\item {} 
\sphinxAtStartPar
\sphinxstyleliteralstrong{\sphinxupquote{velocity}} (\sphinxstyleliteralemphasis{\sphinxupquote{float}}) \textendash{} Velocity value in meters

\item {} 
\sphinxAtStartPar
\sphinxstyleliteralstrong{\sphinxupquote{lref}} (\sphinxstyleliteralemphasis{\sphinxupquote{float}}) \textendash{} Refence length in meters

\item {} 
\sphinxAtStartPar
\sphinxstyleliteralstrong{\sphinxupquote{freesteam}} ({\hyperref[\detokenize{modules:configuration.Freestream}]{\sphinxcrossref{\sphinxstyleliteralemphasis{\sphinxupquote{Freestream}}}}}) \textendash{} Object of class assembly.freestream

\item {} 
\sphinxAtStartPar
\sphinxstyleliteralstrong{\sphinxupquote{options}} ({\hyperref[\detokenize{modules:configuration.Options}]{\sphinxcrossref{\sphinxstyleliteralemphasis{\sphinxupquote{Options}}}}}) \textendash{} Object of class Options

\end{itemize}

\end{description}\end{quote}

\end{fulllineitems}

\index{compute\_stagnation() (in module mix\_properties)@\spxentry{compute\_stagnation()}\spxextra{in module mix\_properties}}

\begin{fulllineitems}
\phantomsection\label{\detokenize{modules:mix_properties.compute_stagnation}}
\pysigstartsignatures
\pysiglinewithargsret{\sphinxcode{\sphinxupquote{mix\_properties.}}\sphinxbfcode{\sphinxupquote{compute\_stagnation}}}{\emph{\DUrole{n}{free}}, \emph{\DUrole{n}{options}}}{}
\pysigstopsignatures
\sphinxAtStartPar
Compute the post\sphinxhyphen{}shock stagnation values
\begin{quote}\begin{description}
\sphinxlineitem{Parameters}\begin{itemize}
\item {} 
\sphinxAtStartPar
\sphinxstyleliteralstrong{\sphinxupquote{free}} ({\hyperref[\detokenize{modules:configuration.Freestream}]{\sphinxcrossref{\sphinxstyleliteralemphasis{\sphinxupquote{Freestream}}}}}) \textendash{} Object of class assembly.freestream

\item {} 
\sphinxAtStartPar
\sphinxstyleliteralstrong{\sphinxupquote{options}} ({\hyperref[\detokenize{modules:configuration.Options}]{\sphinxcrossref{\sphinxstyleliteralemphasis{\sphinxupquote{Options}}}}}) \textendash{} Object of class Options

\end{itemize}

\end{description}\end{quote}

\end{fulllineitems}



\section{Geometry}
\label{\detokenize{modules:geometry}}

\subsection{Assembly}
\label{\detokenize{modules:assembly}}\index{Assembly\_list (class in assembly)@\spxentry{Assembly\_list}\spxextra{class in assembly}}

\begin{fulllineitems}
\phantomsection\label{\detokenize{modules:assembly.Assembly_list}}
\pysigstartsignatures
\pysiglinewithargsret{\sphinxbfcode{\sphinxupquote{class\DUrole{w}{  }}}\sphinxcode{\sphinxupquote{assembly.}}\sphinxbfcode{\sphinxupquote{Assembly\_list}}}{\emph{\DUrole{n}{objects}}}{}
\pysigstopsignatures
\sphinxAtStartPar
Class Assembly list

\sphinxAtStartPar
A class to store a list of assemblies and respective information, as well as the number of iterations and simulation time
\index{assembly (assembly.Assembly\_list attribute)@\spxentry{assembly}\spxextra{assembly.Assembly\_list attribute}}

\begin{fulllineitems}
\phantomsection\label{\detokenize{modules:assembly.Assembly_list.assembly}}
\pysigstartsignatures
\pysigline{\sphinxbfcode{\sphinxupquote{assembly}}}
\pysigstopsignatures
\sphinxAtStartPar
{[}array{]} List of assemblies

\end{fulllineitems}

\index{connectivity (assembly.Assembly\_list attribute)@\spxentry{connectivity}\spxextra{assembly.Assembly\_list attribute}}

\begin{fulllineitems}
\phantomsection\label{\detokenize{modules:assembly.Assembly_list.connectivity}}
\pysigstartsignatures
\pysigline{\sphinxbfcode{\sphinxupquote{connectivity}}}
\pysigstopsignatures
\sphinxAtStartPar
{[}array{]} List of the linkage information between the different components

\end{fulllineitems}

\index{create\_assembly() (assembly.Assembly\_list method)@\spxentry{create\_assembly()}\spxextra{assembly.Assembly\_list method}}

\begin{fulllineitems}
\phantomsection\label{\detokenize{modules:assembly.Assembly_list.create_assembly}}
\pysigstartsignatures
\pysiglinewithargsret{\sphinxbfcode{\sphinxupquote{create\_assembly}}}{\emph{\DUrole{n}{connectivity}}, \emph{\DUrole{n}{aoa}\DUrole{o}{=}\DUrole{default_value}{0.0}}, \emph{\DUrole{n}{slip}\DUrole{o}{=}\DUrole{default_value}{0.0}}, \emph{\DUrole{n}{roll}\DUrole{o}{=}\DUrole{default_value}{0.0}}}{}
\pysigstopsignatures
\sphinxAtStartPar
Creates the assembly list
\begin{quote}\begin{description}
\sphinxlineitem{Parameters}\begin{itemize}
\item {} 
\sphinxAtStartPar
\sphinxstyleliteralstrong{\sphinxupquote{connectivty}} (\sphinxstyleliteralemphasis{\sphinxupquote{array}}) \textendash{} array containing the user\sphinxhyphen{}defined connectivity between the different components

\item {} 
\sphinxAtStartPar
\sphinxstyleliteralstrong{\sphinxupquote{aoa}} (\sphinxstyleliteralemphasis{\sphinxupquote{float}}) \textendash{} Angle of attack in degrees

\item {} 
\sphinxAtStartPar
\sphinxstyleliteralstrong{\sphinxupquote{slip}} (\sphinxstyleliteralemphasis{\sphinxupquote{float}}) \textendash{} Slip angle in degrees

\item {} 
\sphinxAtStartPar
\sphinxstyleliteralstrong{\sphinxupquote{roll}} (\sphinxstyleliteralemphasis{\sphinxupquote{float}}) \textendash{} Roll angle in degrees

\end{itemize}

\end{description}\end{quote}

\end{fulllineitems}

\index{id (assembly.Assembly\_list attribute)@\spxentry{id}\spxextra{assembly.Assembly\_list attribute}}

\begin{fulllineitems}
\phantomsection\label{\detokenize{modules:assembly.Assembly_list.id}}
\pysigstartsignatures
\pysigline{\sphinxbfcode{\sphinxupquote{id}}}
\pysigstopsignatures
\sphinxAtStartPar
{[}array{]} Number ID to identify the assembly. Whenever an assembly is generated (i.e. due to fragmentation/ablation or during the preprocessing phase), it will have this number ID.

\end{fulllineitems}

\index{iter (assembly.Assembly\_list attribute)@\spxentry{iter}\spxextra{assembly.Assembly\_list attribute}}

\begin{fulllineitems}
\phantomsection\label{\detokenize{modules:assembly.Assembly_list.iter}}
\pysigstartsignatures
\pysigline{\sphinxbfcode{\sphinxupquote{iter}}}
\pysigstopsignatures
\sphinxAtStartPar
{[}int{]} Iteration

\end{fulllineitems}

\index{objects (assembly.Assembly\_list attribute)@\spxentry{objects}\spxextra{assembly.Assembly\_list attribute}}

\begin{fulllineitems}
\phantomsection\label{\detokenize{modules:assembly.Assembly_list.objects}}
\pysigstartsignatures
\pysigline{\sphinxbfcode{\sphinxupquote{objects}}}
\pysigstopsignatures
\sphinxAtStartPar
{[}array{]} List of components

\end{fulllineitems}

\index{time (assembly.Assembly\_list attribute)@\spxentry{time}\spxextra{assembly.Assembly\_list attribute}}

\begin{fulllineitems}
\phantomsection\label{\detokenize{modules:assembly.Assembly_list.time}}
\pysigstartsignatures
\pysigline{\sphinxbfcode{\sphinxupquote{time}}}
\pysigstopsignatures
\sphinxAtStartPar
{[}float{]} simulation physical time

\end{fulllineitems}


\end{fulllineitems}

\index{Dynamics (class in assembly)@\spxentry{Dynamics}\spxextra{class in assembly}}

\begin{fulllineitems}
\phantomsection\label{\detokenize{modules:assembly.Dynamics}}
\pysigstartsignatures
\pysiglinewithargsret{\sphinxbfcode{\sphinxupquote{class\DUrole{w}{  }}}\sphinxcode{\sphinxupquote{assembly.}}\sphinxbfcode{\sphinxupquote{Dynamics}}}{\emph{\DUrole{n}{roll}\DUrole{o}{=}\DUrole{default_value}{0}}, \emph{\DUrole{n}{pitch}\DUrole{o}{=}\DUrole{default_value}{0}}, \emph{\DUrole{n}{yaw}\DUrole{o}{=}\DUrole{default_value}{0}}, \emph{\DUrole{n}{vel\_roll}\DUrole{o}{=}\DUrole{default_value}{0}}, \emph{\DUrole{n}{vel\_pitch}\DUrole{o}{=}\DUrole{default_value}{0}}, \emph{\DUrole{n}{vel\_yaw}\DUrole{o}{=}\DUrole{default_value}{0}}}{}
\pysigstopsignatures
\sphinxAtStartPar
Class Dynamics

\sphinxAtStartPar
A class to store the dynamics information of the assembly
\index{pitch (assembly.Dynamics attribute)@\spxentry{pitch}\spxextra{assembly.Dynamics attribute}}

\begin{fulllineitems}
\phantomsection\label{\detokenize{modules:assembly.Dynamics.pitch}}
\pysigstartsignatures
\pysigline{\sphinxbfcode{\sphinxupquote{pitch}}}
\pysigstopsignatures
\sphinxAtStartPar
{[}float{]} Pitch angle in radians

\end{fulllineitems}

\index{roll (assembly.Dynamics attribute)@\spxentry{roll}\spxextra{assembly.Dynamics attribute}}

\begin{fulllineitems}
\phantomsection\label{\detokenize{modules:assembly.Dynamics.roll}}
\pysigstartsignatures
\pysigline{\sphinxbfcode{\sphinxupquote{roll}}}
\pysigstopsignatures
\sphinxAtStartPar
{[}float{]} Roll angle in radians

\end{fulllineitems}

\index{vel\_pitch (assembly.Dynamics attribute)@\spxentry{vel\_pitch}\spxextra{assembly.Dynamics attribute}}

\begin{fulllineitems}
\phantomsection\label{\detokenize{modules:assembly.Dynamics.vel_pitch}}
\pysigstartsignatures
\pysigline{\sphinxbfcode{\sphinxupquote{vel\_pitch}}}
\pysigstopsignatures
\sphinxAtStartPar
{[}float{]} Pitch angular velocity in rad/s

\end{fulllineitems}

\index{vel\_roll (assembly.Dynamics attribute)@\spxentry{vel\_roll}\spxextra{assembly.Dynamics attribute}}

\begin{fulllineitems}
\phantomsection\label{\detokenize{modules:assembly.Dynamics.vel_roll}}
\pysigstartsignatures
\pysigline{\sphinxbfcode{\sphinxupquote{vel\_roll}}}
\pysigstopsignatures
\sphinxAtStartPar
{[}float{]} Roll angular velocity in rad/s

\end{fulllineitems}

\index{vel\_yaw (assembly.Dynamics attribute)@\spxentry{vel\_yaw}\spxextra{assembly.Dynamics attribute}}

\begin{fulllineitems}
\phantomsection\label{\detokenize{modules:assembly.Dynamics.vel_yaw}}
\pysigstartsignatures
\pysigline{\sphinxbfcode{\sphinxupquote{vel\_yaw}}}
\pysigstopsignatures
\sphinxAtStartPar
{[}float{]} Yaw angular velocity in rad/s

\end{fulllineitems}

\index{yaw (assembly.Dynamics attribute)@\spxentry{yaw}\spxextra{assembly.Dynamics attribute}}

\begin{fulllineitems}
\phantomsection\label{\detokenize{modules:assembly.Dynamics.yaw}}
\pysigstartsignatures
\pysigline{\sphinxbfcode{\sphinxupquote{yaw}}}
\pysigstopsignatures
\sphinxAtStartPar
{[}float{]} Yaw angle in radians

\end{fulllineitems}


\end{fulllineitems}

\index{Body\_force (class in assembly)@\spxentry{Body\_force}\spxextra{class in assembly}}

\begin{fulllineitems}
\phantomsection\label{\detokenize{modules:assembly.Body_force}}
\pysigstartsignatures
\pysiglinewithargsret{\sphinxbfcode{\sphinxupquote{class\DUrole{w}{  }}}\sphinxcode{\sphinxupquote{assembly.}}\sphinxbfcode{\sphinxupquote{Body\_force}}}{\emph{\DUrole{n}{force}\DUrole{o}{=}\DUrole{default_value}{array({[}{[}0.0{]}, {[}0.0{]}, {[}0.0{]}{]})}}, \emph{\DUrole{n}{moment}\DUrole{o}{=}\DUrole{default_value}{array({[}{[}0.0{]}, {[}0.0{]}, {[}0.0{]}{]})}}}{}
\pysigstopsignatures
\sphinxAtStartPar
Class Body\_force

\sphinxAtStartPar
A class to store the force and moment information that the assembly experiences at each iteration in the body frame
\index{force (assembly.Body\_force attribute)@\spxentry{force}\spxextra{assembly.Body\_force attribute}}

\begin{fulllineitems}
\phantomsection\label{\detokenize{modules:assembly.Body_force.force}}
\pysigstartsignatures
\pysigline{\sphinxbfcode{\sphinxupquote{force}}}
\pysigstopsignatures
\sphinxAtStartPar
{[}np.array{]} Force array (3x1)

\end{fulllineitems}

\index{moment (assembly.Body\_force attribute)@\spxentry{moment}\spxextra{assembly.Body\_force attribute}}

\begin{fulllineitems}
\phantomsection\label{\detokenize{modules:assembly.Body_force.moment}}
\pysigstartsignatures
\pysigline{\sphinxbfcode{\sphinxupquote{moment}}}
\pysigstopsignatures
\sphinxAtStartPar
{[}np.array{]} Moment array (3x1)

\end{fulllineitems}


\end{fulllineitems}

\index{Wind\_force (class in assembly)@\spxentry{Wind\_force}\spxextra{class in assembly}}

\begin{fulllineitems}
\phantomsection\label{\detokenize{modules:assembly.Wind_force}}
\pysigstartsignatures
\pysiglinewithargsret{\sphinxbfcode{\sphinxupquote{class\DUrole{w}{  }}}\sphinxcode{\sphinxupquote{assembly.}}\sphinxbfcode{\sphinxupquote{Wind\_force}}}{\emph{\DUrole{n}{lift}\DUrole{o}{=}\DUrole{default_value}{0}}, \emph{\DUrole{n}{drag}\DUrole{o}{=}\DUrole{default_value}{0}}, \emph{\DUrole{n}{crosswind}\DUrole{o}{=}\DUrole{default_value}{0}}}{}
\pysigstopsignatures
\sphinxAtStartPar
Class Wind\_force

\sphinxAtStartPar
A class to store the force information that the assembly experiences at each iteration in the wind frame
\index{crosswind (assembly.Wind\_force attribute)@\spxentry{crosswind}\spxextra{assembly.Wind\_force attribute}}

\begin{fulllineitems}
\phantomsection\label{\detokenize{modules:assembly.Wind_force.crosswind}}
\pysigstartsignatures
\pysigline{\sphinxbfcode{\sphinxupquote{crosswind}}}
\pysigstopsignatures
\sphinxAtStartPar
{[}float{]} Crosswind force

\end{fulllineitems}

\index{drag (assembly.Wind\_force attribute)@\spxentry{drag}\spxextra{assembly.Wind\_force attribute}}

\begin{fulllineitems}
\phantomsection\label{\detokenize{modules:assembly.Wind_force.drag}}
\pysigstartsignatures
\pysigline{\sphinxbfcode{\sphinxupquote{drag}}}
\pysigstopsignatures
\sphinxAtStartPar
{[}float{]} Drag force

\end{fulllineitems}

\index{lift (assembly.Wind\_force attribute)@\spxentry{lift}\spxextra{assembly.Wind\_force attribute}}

\begin{fulllineitems}
\phantomsection\label{\detokenize{modules:assembly.Wind_force.lift}}
\pysigstartsignatures
\pysigline{\sphinxbfcode{\sphinxupquote{lift}}}
\pysigstopsignatures
\sphinxAtStartPar
{[}float{]} Lift force

\end{fulllineitems}


\end{fulllineitems}

\index{Freestream (class in assembly)@\spxentry{Freestream}\spxextra{class in assembly}}

\begin{fulllineitems}
\phantomsection\label{\detokenize{modules:assembly.Freestream}}
\pysigstartsignatures
\pysiglinewithargsret{\sphinxbfcode{\sphinxupquote{class\DUrole{w}{  }}}\sphinxcode{\sphinxupquote{assembly.}}\sphinxbfcode{\sphinxupquote{Freestream}}}{\emph{\DUrole{n}{pressure}\DUrole{o}{=}\DUrole{default_value}{0}}, \emph{\DUrole{n}{mach}\DUrole{o}{=}\DUrole{default_value}{0}}, \emph{\DUrole{n}{gamma}\DUrole{o}{=}\DUrole{default_value}{0}}, \emph{\DUrole{n}{knudsen}\DUrole{o}{=}\DUrole{default_value}{0}}, \emph{\DUrole{n}{prandtl}\DUrole{o}{=}\DUrole{default_value}{0}}, \emph{\DUrole{n}{temperature}\DUrole{o}{=}\DUrole{default_value}{0}}, \emph{\DUrole{n}{density}\DUrole{o}{=}\DUrole{default_value}{0}}, \emph{\DUrole{n}{velocity}\DUrole{o}{=}\DUrole{default_value}{0}}, \emph{\DUrole{n}{R}\DUrole{o}{=}\DUrole{default_value}{0}}, \emph{\DUrole{n}{mfp}\DUrole{o}{=}\DUrole{default_value}{0}}, \emph{\DUrole{n}{omega}\DUrole{o}{=}\DUrole{default_value}{0}}, \emph{\DUrole{n}{diameter}\DUrole{o}{=}\DUrole{default_value}{0}}, \emph{\DUrole{n}{mu}\DUrole{o}{=}\DUrole{default_value}{0}}, \emph{\DUrole{n}{cp}\DUrole{o}{=}\DUrole{default_value}{0}}, \emph{\DUrole{n}{cv}\DUrole{o}{=}\DUrole{default_value}{0}}}{}
\pysigstopsignatures
\sphinxAtStartPar
Class Freestream

\sphinxAtStartPar
A class to store freestream information with respect to the position and velocity of each assembly
\index{P1\_s (assembly.Freestream attribute)@\spxentry{P1\_s}\spxextra{assembly.Freestream attribute}}

\begin{fulllineitems}
\phantomsection\label{\detokenize{modules:assembly.Freestream.P1_s}}
\pysigstartsignatures
\pysigline{\sphinxbfcode{\sphinxupquote{P1\_s}}}
\pysigstopsignatures
\sphinxAtStartPar
{[}float{]} Pressure at the stagnation point

\end{fulllineitems}

\index{R (assembly.Freestream attribute)@\spxentry{R}\spxextra{assembly.Freestream attribute}}

\begin{fulllineitems}
\phantomsection\label{\detokenize{modules:assembly.Freestream.R}}
\pysigstartsignatures
\pysigline{\sphinxbfcode{\sphinxupquote{R}}}
\pysigstopsignatures
\sphinxAtStartPar
{[}float{]} Gas constant

\end{fulllineitems}

\index{T1\_s (assembly.Freestream attribute)@\spxentry{T1\_s}\spxextra{assembly.Freestream attribute}}

\begin{fulllineitems}
\phantomsection\label{\detokenize{modules:assembly.Freestream.T1_s}}
\pysigstartsignatures
\pysigline{\sphinxbfcode{\sphinxupquote{T1\_s}}}
\pysigstopsignatures
\sphinxAtStartPar
{[}float{]} Temperature at the stagnation point

\end{fulllineitems}

\index{cp (assembly.Freestream attribute)@\spxentry{cp}\spxextra{assembly.Freestream attribute}}

\begin{fulllineitems}
\phantomsection\label{\detokenize{modules:assembly.Freestream.cp}}
\pysigstartsignatures
\pysigline{\sphinxbfcode{\sphinxupquote{cp}}}
\pysigstopsignatures
\sphinxAtStartPar
{[}float{]} Heat capacity at constant pressure

\end{fulllineitems}

\index{cv (assembly.Freestream attribute)@\spxentry{cv}\spxextra{assembly.Freestream attribute}}

\begin{fulllineitems}
\phantomsection\label{\detokenize{modules:assembly.Freestream.cv}}
\pysigstartsignatures
\pysigline{\sphinxbfcode{\sphinxupquote{cv}}}
\pysigstopsignatures
\sphinxAtStartPar
{[}float{]} Heat capacity at constant volume

\end{fulllineitems}

\index{density (assembly.Freestream attribute)@\spxentry{density}\spxextra{assembly.Freestream attribute}}

\begin{fulllineitems}
\phantomsection\label{\detokenize{modules:assembly.Freestream.density}}
\pysigstartsignatures
\pysigline{\sphinxbfcode{\sphinxupquote{density}}}
\pysigstopsignatures
\sphinxAtStartPar
{[}float{]} Freestream density {[}kg/m\textasciicircum{}3{]}

\end{fulllineitems}

\index{diameter (assembly.Freestream attribute)@\spxentry{diameter}\spxextra{assembly.Freestream attribute}}

\begin{fulllineitems}
\phantomsection\label{\detokenize{modules:assembly.Freestream.diameter}}
\pysigstartsignatures
\pysigline{\sphinxbfcode{\sphinxupquote{diameter}}}
\pysigstopsignatures
\sphinxAtStartPar
{[}float{]} Mean diameter

\end{fulllineitems}

\index{gamma (assembly.Freestream attribute)@\spxentry{gamma}\spxextra{assembly.Freestream attribute}}

\begin{fulllineitems}
\phantomsection\label{\detokenize{modules:assembly.Freestream.gamma}}
\pysigstartsignatures
\pysigline{\sphinxbfcode{\sphinxupquote{gamma}}}
\pysigstopsignatures
\sphinxAtStartPar
{[}float{]} Freestream specific heat ratio

\end{fulllineitems}

\index{h1\_s (assembly.Freestream attribute)@\spxentry{h1\_s}\spxextra{assembly.Freestream attribute}}

\begin{fulllineitems}
\phantomsection\label{\detokenize{modules:assembly.Freestream.h1_s}}
\pysigstartsignatures
\pysigline{\sphinxbfcode{\sphinxupquote{h1\_s}}}
\pysigstopsignatures
\sphinxAtStartPar
{[}?float?{]} Specific enthalpy at the stagnation point

\end{fulllineitems}

\index{knudsen (assembly.Freestream attribute)@\spxentry{knudsen}\spxextra{assembly.Freestream attribute}}

\begin{fulllineitems}
\phantomsection\label{\detokenize{modules:assembly.Freestream.knudsen}}
\pysigstartsignatures
\pysigline{\sphinxbfcode{\sphinxupquote{knudsen}}}
\pysigstopsignatures
\sphinxAtStartPar
{[}float{]} Freestream knudsen

\end{fulllineitems}

\index{mach (assembly.Freestream attribute)@\spxentry{mach}\spxextra{assembly.Freestream attribute}}

\begin{fulllineitems}
\phantomsection\label{\detokenize{modules:assembly.Freestream.mach}}
\pysigstartsignatures
\pysigline{\sphinxbfcode{\sphinxupquote{mach}}}
\pysigstopsignatures
\sphinxAtStartPar
{[}float{]} Freestream mach

\end{fulllineitems}

\index{mfp (assembly.Freestream attribute)@\spxentry{mfp}\spxextra{assembly.Freestream attribute}}

\begin{fulllineitems}
\phantomsection\label{\detokenize{modules:assembly.Freestream.mfp}}
\pysigstartsignatures
\pysigline{\sphinxbfcode{\sphinxupquote{mfp}}}
\pysigstopsignatures
\sphinxAtStartPar
{[}float{]} mean free path in meters

\end{fulllineitems}

\index{mu (assembly.Freestream attribute)@\spxentry{mu}\spxextra{assembly.Freestream attribute}}

\begin{fulllineitems}
\phantomsection\label{\detokenize{modules:assembly.Freestream.mu}}
\pysigstartsignatures
\pysigline{\sphinxbfcode{\sphinxupquote{mu}}}
\pysigstopsignatures
\sphinxAtStartPar
{[}float{]} Mean viscosity

\end{fulllineitems}

\index{mu\_s (assembly.Freestream attribute)@\spxentry{mu\_s}\spxextra{assembly.Freestream attribute}}

\begin{fulllineitems}
\phantomsection\label{\detokenize{modules:assembly.Freestream.mu_s}}
\pysigstartsignatures
\pysigline{\sphinxbfcode{\sphinxupquote{mu\_s}}}
\pysigstopsignatures
\sphinxAtStartPar
{[}float{]} Viscosity at the stagnation point

\end{fulllineitems}

\index{omega (assembly.Freestream attribute)@\spxentry{omega}\spxextra{assembly.Freestream attribute}}

\begin{fulllineitems}
\phantomsection\label{\detokenize{modules:assembly.Freestream.omega}}
\pysigstartsignatures
\pysigline{\sphinxbfcode{\sphinxupquote{omega}}}
\pysigstopsignatures
\sphinxAtStartPar
{[}float{]} Mean viscosity coefficient

\end{fulllineitems}

\index{percent\_gas (assembly.Freestream attribute)@\spxentry{percent\_gas}\spxextra{assembly.Freestream attribute}}

\begin{fulllineitems}
\phantomsection\label{\detokenize{modules:assembly.Freestream.percent_gas}}
\pysigstartsignatures
\pysigline{\sphinxbfcode{\sphinxupquote{percent\_gas}}}
\pysigstopsignatures
\sphinxAtStartPar
{[}array (float){]} percentage of species in the mixture with respect to moles

\end{fulllineitems}

\index{percent\_mass (assembly.Freestream attribute)@\spxentry{percent\_mass}\spxextra{assembly.Freestream attribute}}

\begin{fulllineitems}
\phantomsection\label{\detokenize{modules:assembly.Freestream.percent_mass}}
\pysigstartsignatures
\pysigline{\sphinxbfcode{\sphinxupquote{percent\_mass}}}
\pysigstopsignatures
\sphinxAtStartPar
{[}array (float){]} percentage of species in the mixture with respect to mass

\end{fulllineitems}

\index{prandtl (assembly.Freestream attribute)@\spxentry{prandtl}\spxextra{assembly.Freestream attribute}}

\begin{fulllineitems}
\phantomsection\label{\detokenize{modules:assembly.Freestream.prandtl}}
\pysigstartsignatures
\pysigline{\sphinxbfcode{\sphinxupquote{prandtl}}}
\pysigstopsignatures
\sphinxAtStartPar
{[}float{]} Freestream prandtl

\end{fulllineitems}

\index{pressure (assembly.Freestream attribute)@\spxentry{pressure}\spxextra{assembly.Freestream attribute}}

\begin{fulllineitems}
\phantomsection\label{\detokenize{modules:assembly.Freestream.pressure}}
\pysigstartsignatures
\pysigline{\sphinxbfcode{\sphinxupquote{pressure}}}
\pysigstopsignatures
\sphinxAtStartPar
{[}float{]} Freestream pressure {[}Pa{]}

\end{fulllineitems}

\index{rho\_s (assembly.Freestream attribute)@\spxentry{rho\_s}\spxextra{assembly.Freestream attribute}}

\begin{fulllineitems}
\phantomsection\label{\detokenize{modules:assembly.Freestream.rho_s}}
\pysigstartsignatures
\pysigline{\sphinxbfcode{\sphinxupquote{rho\_s}}}
\pysigstopsignatures
\sphinxAtStartPar
{[}float{]} Density at the stagnation point

\end{fulllineitems}

\index{species\_index (assembly.Freestream attribute)@\spxentry{species\_index}\spxextra{assembly.Freestream attribute}}

\begin{fulllineitems}
\phantomsection\label{\detokenize{modules:assembly.Freestream.species_index}}
\pysigstartsignatures
\pysigline{\sphinxbfcode{\sphinxupquote{species\_index}}}
\pysigstopsignatures
\sphinxAtStartPar
{[}array (str){]} list of species in the mixture

\end{fulllineitems}

\index{temperature (assembly.Freestream attribute)@\spxentry{temperature}\spxextra{assembly.Freestream attribute}}

\begin{fulllineitems}
\phantomsection\label{\detokenize{modules:assembly.Freestream.temperature}}
\pysigstartsignatures
\pysigline{\sphinxbfcode{\sphinxupquote{temperature}}}
\pysigstopsignatures
\sphinxAtStartPar
{[}float{]} Freestream temperature {[}K{]}

\end{fulllineitems}

\index{velocity (assembly.Freestream attribute)@\spxentry{velocity}\spxextra{assembly.Freestream attribute}}

\begin{fulllineitems}
\phantomsection\label{\detokenize{modules:assembly.Freestream.velocity}}
\pysigstartsignatures
\pysigline{\sphinxbfcode{\sphinxupquote{velocity}}}
\pysigstopsignatures
\sphinxAtStartPar
{[}float{]} Freestream velocity {[}m/s{]}

\end{fulllineitems}


\end{fulllineitems}

\index{Aerothermo (class in assembly)@\spxentry{Aerothermo}\spxextra{class in assembly}}

\begin{fulllineitems}
\phantomsection\label{\detokenize{modules:assembly.Aerothermo}}
\pysigstartsignatures
\pysiglinewithargsret{\sphinxbfcode{\sphinxupquote{class\DUrole{w}{  }}}\sphinxcode{\sphinxupquote{assembly.}}\sphinxbfcode{\sphinxupquote{Aerothermo}}}{\emph{\DUrole{n}{n\_points}}}{}
\pysigstopsignatures
\sphinxAtStartPar
Class Aerothermo

\sphinxAtStartPar
A class to store the surface quantities
\index{heatflux (assembly.Aerothermo attribute)@\spxentry{heatflux}\spxextra{assembly.Aerothermo attribute}}

\begin{fulllineitems}
\phantomsection\label{\detokenize{modules:assembly.Aerothermo.heatflux}}
\pysigstartsignatures
\pysigline{\sphinxbfcode{\sphinxupquote{heatflux}}}
\pysigstopsignatures
\sphinxAtStartPar
{[}np.array{]} Heatflux {[}W{]}

\end{fulllineitems}

\index{pressure (assembly.Aerothermo attribute)@\spxentry{pressure}\spxextra{assembly.Aerothermo attribute}}

\begin{fulllineitems}
\phantomsection\label{\detokenize{modules:assembly.Aerothermo.pressure}}
\pysigstartsignatures
\pysigline{\sphinxbfcode{\sphinxupquote{pressure}}}
\pysigstopsignatures
\sphinxAtStartPar
{[}np.array{]} Pressure {[}Pa{]}

\end{fulllineitems}

\index{shear (assembly.Aerothermo attribute)@\spxentry{shear}\spxextra{assembly.Aerothermo attribute}}

\begin{fulllineitems}
\phantomsection\label{\detokenize{modules:assembly.Aerothermo.shear}}
\pysigstartsignatures
\pysigline{\sphinxbfcode{\sphinxupquote{shear}}}
\pysigstopsignatures
\sphinxAtStartPar
{[}np.array{]} Skin friction {[}Pa{]}

\end{fulllineitems}


\end{fulllineitems}

\index{Assembly (class in assembly)@\spxentry{Assembly}\spxextra{class in assembly}}

\begin{fulllineitems}
\phantomsection\label{\detokenize{modules:assembly.Assembly}}
\pysigstartsignatures
\pysiglinewithargsret{\sphinxbfcode{\sphinxupquote{class\DUrole{w}{  }}}\sphinxcode{\sphinxupquote{assembly.}}\sphinxbfcode{\sphinxupquote{Assembly}}}{\emph{\DUrole{n}{objects}\DUrole{o}{=}\DUrole{default_value}{{[}{]}}}, \emph{\DUrole{n}{id}\DUrole{o}{=}\DUrole{default_value}{0}}, \emph{\DUrole{n}{aoa}\DUrole{o}{=}\DUrole{default_value}{0.0}}, \emph{\DUrole{n}{slip}\DUrole{o}{=}\DUrole{default_value}{0.0}}, \emph{\DUrole{n}{roll}\DUrole{o}{=}\DUrole{default_value}{0.0}}}{}
\pysigstopsignatures
\sphinxAtStartPar
Class Assembly

\sphinxAtStartPar
A class to store the information respective to each assemly at every time iteration
\index{Aref (assembly.Assembly attribute)@\spxentry{Aref}\spxextra{assembly.Assembly attribute}}

\begin{fulllineitems}
\phantomsection\label{\detokenize{modules:assembly.Assembly.Aref}}
\pysigstartsignatures
\pysigline{\sphinxbfcode{\sphinxupquote{Aref}}}
\pysigstopsignatures
\sphinxAtStartPar
{[}float{]} Area of reference {[}meters\textasciicircum{}2{]}

\end{fulllineitems}

\index{COG (assembly.Assembly attribute)@\spxentry{COG}\spxextra{assembly.Assembly attribute}}

\begin{fulllineitems}
\phantomsection\label{\detokenize{modules:assembly.Assembly.COG}}
\pysigstartsignatures
\pysigline{\sphinxbfcode{\sphinxupquote{COG}}}
\pysigstopsignatures
\sphinxAtStartPar
{[}array{]} CYZ coordinates of the center of mass in the body frame {[}meters{]}

\end{fulllineitems}

\index{Lref (assembly.Assembly attribute)@\spxentry{Lref}\spxextra{assembly.Assembly attribute}}

\begin{fulllineitems}
\phantomsection\label{\detokenize{modules:assembly.Assembly.Lref}}
\pysigstartsignatures
\pysigline{\sphinxbfcode{\sphinxupquote{Lref}}}
\pysigstopsignatures
\sphinxAtStartPar
{[}float{]} Length of reference {[}meters{]}

\end{fulllineitems}

\index{aerothermo (assembly.Assembly attribute)@\spxentry{aerothermo}\spxextra{assembly.Assembly attribute}}

\begin{fulllineitems}
\phantomsection\label{\detokenize{modules:assembly.Assembly.aerothermo}}
\pysigstartsignatures
\pysigline{\sphinxbfcode{\sphinxupquote{aerothermo}}}
\pysigstopsignatures
\sphinxAtStartPar
{[}Aerothermo{]} Object of class Aerothermo to store the surface quantities

\end{fulllineitems}

\index{aoa (assembly.Assembly attribute)@\spxentry{aoa}\spxextra{assembly.Assembly attribute}}

\begin{fulllineitems}
\phantomsection\label{\detokenize{modules:assembly.Assembly.aoa}}
\pysigstartsignatures
\pysigline{\sphinxbfcode{\sphinxupquote{aoa}}}
\pysigstopsignatures
\sphinxAtStartPar
{[}float{]} Angle of attack {[}radians{]}

\end{fulllineitems}

\index{body\_force (assembly.Assembly attribute)@\spxentry{body\_force}\spxextra{assembly.Assembly attribute}}

\begin{fulllineitems}
\phantomsection\label{\detokenize{modules:assembly.Assembly.body_force}}
\pysigstartsignatures
\pysigline{\sphinxbfcode{\sphinxupquote{body\_force}}}
\pysigstopsignatures
\sphinxAtStartPar
{[}Body\_force{]} Object of class Body\_force to store the force and moment information in the body frame

\end{fulllineitems}

\index{compute\_mass\_properties() (assembly.Assembly method)@\spxentry{compute\_mass\_properties()}\spxextra{assembly.Assembly method}}

\begin{fulllineitems}
\phantomsection\label{\detokenize{modules:assembly.Assembly.compute_mass_properties}}
\pysigstartsignatures
\pysiglinewithargsret{\sphinxbfcode{\sphinxupquote{compute\_mass\_properties}}}{}{}
\pysigstopsignatures
\sphinxAtStartPar
Computes the inertial properties

\sphinxAtStartPar
Function to compute the inertial properties using the 3D domain information.

\end{fulllineitems}

\index{dynamics (assembly.Assembly attribute)@\spxentry{dynamics}\spxextra{assembly.Assembly attribute}}

\begin{fulllineitems}
\phantomsection\label{\detokenize{modules:assembly.Assembly.dynamics}}
\pysigstartsignatures
\pysigline{\sphinxbfcode{\sphinxupquote{dynamics}}}
\pysigstopsignatures
\sphinxAtStartPar
{[}Dynamics{]} Object of class Dynamics to store the dynamics information

\end{fulllineitems}

\index{freestream (assembly.Assembly attribute)@\spxentry{freestream}\spxextra{assembly.Assembly attribute}}

\begin{fulllineitems}
\phantomsection\label{\detokenize{modules:assembly.Assembly.freestream}}
\pysigstartsignatures
\pysigline{\sphinxbfcode{\sphinxupquote{freestream}}}
\pysigstopsignatures
\sphinxAtStartPar
{[}Freestream{]} Object of class Freestream to store the freestream information

\end{fulllineitems}

\index{generate\_inner\_domain() (assembly.Assembly method)@\spxentry{generate\_inner\_domain()}\spxextra{assembly.Assembly method}}

\begin{fulllineitems}
\phantomsection\label{\detokenize{modules:assembly.Assembly.generate_inner_domain}}
\pysigstartsignatures
\pysiglinewithargsret{\sphinxbfcode{\sphinxupquote{generate\_inner\_domain}}}{\emph{\DUrole{n}{write}\DUrole{o}{=}\DUrole{default_value}{False}}, \emph{\DUrole{n}{output\_folder}\DUrole{o}{=}\DUrole{default_value}{\textquotesingle{}\textquotesingle{}}}, \emph{\DUrole{n}{output\_filename}\DUrole{o}{=}\DUrole{default_value}{\textquotesingle{}\textquotesingle{}}}, \emph{\DUrole{n}{bc\_ids}\DUrole{o}{=}\DUrole{default_value}{{[}{]}}}}{}
\pysigstopsignatures
\sphinxAtStartPar
Generates the 3D structural mesh

\sphinxAtStartPar
Generates the tetrahedral inner domain using the GMSH software
\begin{quote}\begin{description}
\sphinxlineitem{Parameters}\begin{itemize}
\item {} 
\sphinxAtStartPar
\sphinxstyleliteralstrong{\sphinxupquote{write}} (\sphinxstyleliteralemphasis{\sphinxupquote{bool}}) \textendash{} Flag to output the 3D domain

\item {} 
\sphinxAtStartPar
\sphinxstyleliteralstrong{\sphinxupquote{output\_folder}} (\sphinxstyleliteralemphasis{\sphinxupquote{str}}) \textendash{} Directory of the output folder when writing the 3D domain

\item {} 
\sphinxAtStartPar
\sphinxstyleliteralstrong{\sphinxupquote{output\_filename}} (\sphinxstyleliteralemphasis{\sphinxupquote{str}}) \textendash{} Name of the file

\end{itemize}

\end{description}\end{quote}

\end{fulllineitems}

\index{id (assembly.Assembly attribute)@\spxentry{id}\spxextra{assembly.Assembly attribute}}

\begin{fulllineitems}
\phantomsection\label{\detokenize{modules:assembly.Assembly.id}}
\pysigstartsignatures
\pysigline{\sphinxbfcode{\sphinxupquote{id}}}
\pysigstopsignatures
\sphinxAtStartPar
{[}int{]} ID of the assembly

\end{fulllineitems}

\index{inertia (assembly.Assembly attribute)@\spxentry{inertia}\spxextra{assembly.Assembly attribute}}

\begin{fulllineitems}
\phantomsection\label{\detokenize{modules:assembly.Assembly.inertia}}
\pysigstartsignatures
\pysigline{\sphinxbfcode{\sphinxupquote{inertia}}}
\pysigstopsignatures
\sphinxAtStartPar
{[}array{]} Inertia matrix in the body frame {[}kg/m\(\sp{\text{2}}\){]}

\end{fulllineitems}

\index{mass (assembly.Assembly attribute)@\spxentry{mass}\spxextra{assembly.Assembly attribute}}

\begin{fulllineitems}
\phantomsection\label{\detokenize{modules:assembly.Assembly.mass}}
\pysigstartsignatures
\pysigline{\sphinxbfcode{\sphinxupquote{mass}}}
\pysigstopsignatures
\sphinxAtStartPar
{[}float{]} Mass of the assembly {[}kg{]}

\end{fulllineitems}

\index{mesh (assembly.Assembly attribute)@\spxentry{mesh}\spxextra{assembly.Assembly attribute}}

\begin{fulllineitems}
\phantomsection\label{\detokenize{modules:assembly.Assembly.mesh}}
\pysigstartsignatures
\pysigline{\sphinxbfcode{\sphinxupquote{mesh}}}
\pysigstopsignatures
\sphinxAtStartPar
{[}Mesh{]} Object of class Mesh containing the grid information

\end{fulllineitems}

\index{objects (assembly.Assembly attribute)@\spxentry{objects}\spxextra{assembly.Assembly attribute}}

\begin{fulllineitems}
\phantomsection\label{\detokenize{modules:assembly.Assembly.objects}}
\pysigstartsignatures
\pysigline{\sphinxbfcode{\sphinxupquote{objects}}}
\pysigstopsignatures
\sphinxAtStartPar
{[}array{]} List of the components that are part of the assembly

\end{fulllineitems}

\index{slip (assembly.Assembly attribute)@\spxentry{slip}\spxextra{assembly.Assembly attribute}}

\begin{fulllineitems}
\phantomsection\label{\detokenize{modules:assembly.Assembly.slip}}
\pysigstartsignatures
\pysigline{\sphinxbfcode{\sphinxupquote{slip}}}
\pysigstopsignatures
\sphinxAtStartPar
{[}float{]} Slip angle {[}radians{]}

\end{fulllineitems}

\index{wind\_force (assembly.Assembly attribute)@\spxentry{wind\_force}\spxextra{assembly.Assembly attribute}}

\begin{fulllineitems}
\phantomsection\label{\detokenize{modules:assembly.Assembly.wind_force}}
\pysigstartsignatures
\pysigline{\sphinxbfcode{\sphinxupquote{wind\_force}}}
\pysigstopsignatures
\sphinxAtStartPar
{[}Body\_force{]} Object of class Wind\_force to store the force information in the wind frame

\end{fulllineitems}


\end{fulllineitems}

\index{create\_assembly\_flag() (in module assembly)@\spxentry{create\_assembly\_flag()}\spxextra{in module assembly}}

\begin{fulllineitems}
\phantomsection\label{\detokenize{modules:assembly.create_assembly_flag}}
\pysigstartsignatures
\pysiglinewithargsret{\sphinxcode{\sphinxupquote{assembly.}}\sphinxbfcode{\sphinxupquote{create\_assembly\_flag}}}{\emph{\DUrole{n}{list\_bodies}}, \emph{\DUrole{n}{Flags}}}{}
\pysigstopsignatures
\sphinxAtStartPar
Generates the assembly connectivity matrix

\sphinxAtStartPar
Creates a flag m*n where m is the number of assemblies and n is the sum of all components used in the simulation.
For every component belonging to a Body, the flag is True on that position.
The assemblies are created according to the generated matrix
\begin{quote}\begin{description}
\sphinxlineitem{Parameters}\begin{itemize}
\item {} 
\sphinxAtStartPar
\sphinxstyleliteralstrong{\sphinxupquote{list\_bodies}} (\sphinxstyleliteralemphasis{\sphinxupquote{array}}\sphinxstyleliteralemphasis{\sphinxupquote{ of }}\sphinxstyleliteralemphasis{\sphinxupquote{components}}) \textendash{} array containing the used\sphinxhyphen{}defined components

\item {} 
\sphinxAtStartPar
\sphinxstyleliteralstrong{\sphinxupquote{Flags}} (\sphinxstyleliteralemphasis{\sphinxupquote{np.array}}) \textendash{} numpy array containing the linkage information of each component

\end{itemize}

\sphinxlineitem{Returns}
\sphinxAtStartPar
\sphinxstylestrong{assembly\_flag} \textendash{} numpy array containing information on how to generate the assemblies with respect to the components introduced in the simulation

\sphinxlineitem{Return type}
\sphinxAtStartPar
np.array

\end{description}\end{quote}

\end{fulllineitems}



\subsection{Component}
\label{\detokenize{modules:component}}\index{Component (class in component)@\spxentry{Component}\spxextra{class in component}}

\begin{fulllineitems}
\phantomsection\label{\detokenize{modules:component.Component}}
\pysigstartsignatures
\pysiglinewithargsret{\sphinxbfcode{\sphinxupquote{class\DUrole{w}{  }}}\sphinxcode{\sphinxupquote{component.}}\sphinxbfcode{\sphinxupquote{Component}}}{\emph{\DUrole{n}{filename}}, \emph{\DUrole{n}{file\_type}}, \emph{\DUrole{n}{inner\_stl}\DUrole{o}{=}\DUrole{default_value}{\textquotesingle{}\textquotesingle{}}}, \emph{\DUrole{n}{id}\DUrole{o}{=}\DUrole{default_value}{0}}, \emph{\DUrole{n}{binary}\DUrole{o}{=}\DUrole{default_value}{True}}, \emph{\DUrole{n}{temperature}\DUrole{o}{=}\DUrole{default_value}{300}}, \emph{\DUrole{n}{trigger\_type}\DUrole{o}{=}\DUrole{default_value}{\textquotesingle{}Indestructible\textquotesingle{}}}, \emph{\DUrole{n}{trigger\_value}\DUrole{o}{=}\DUrole{default_value}{0}}, \emph{\DUrole{n}{fenics\_bc\_id}\DUrole{o}{=}\DUrole{default_value}{\sphinxhyphen{}1}}, \emph{\DUrole{n}{material}\DUrole{o}{=}\DUrole{default_value}{\textquotesingle{}Unittest\textquotesingle{}}}}{}
\pysigstopsignatures
\sphinxAtStartPar
Component class

\sphinxAtStartPar
Class to store the information of a singular component.
\index{COG (component.Component attribute)@\spxentry{COG}\spxextra{component.Component attribute}}

\begin{fulllineitems}
\phantomsection\label{\detokenize{modules:component.Component.COG}}
\pysigstartsignatures
\pysigline{\sphinxbfcode{\sphinxupquote{COG}}}
\pysigstopsignatures
\sphinxAtStartPar
{[}meters{]} Center of mass in XYZ coordinates

\end{fulllineitems}

\index{compute\_mass\_properties() (component.Component method)@\spxentry{compute\_mass\_properties()}\spxextra{component.Component method}}

\begin{fulllineitems}
\phantomsection\label{\detokenize{modules:component.Component.compute_mass_properties}}
\pysigstartsignatures
\pysiglinewithargsret{\sphinxbfcode{\sphinxupquote{compute\_mass\_properties}}}{\emph{\DUrole{n}{coords}}, \emph{\DUrole{n}{elements}}}{}
\pysigstopsignatures
\sphinxAtStartPar
Compute the inertia properties

\sphinxAtStartPar
Uses the volumetric grid information, along with the material density to compute the mass,
Center of mass and inertia matrix using tetras
\begin{quote}\begin{description}
\sphinxlineitem{Parameters}\begin{itemize}
\item {} 
\sphinxAtStartPar
\sphinxstyleliteralstrong{\sphinxupquote{coords}} (\sphinxstyleliteralemphasis{\sphinxupquote{np.array}}) \textendash{} numpy array containing the XYZ coordinates of the vertex of each tetrahedral element

\item {} 
\sphinxAtStartPar
\sphinxstyleliteralstrong{\sphinxupquote{elements}} (\sphinxstyleliteralemphasis{\sphinxupquote{np.array}}) \textendash{} numpy array containing the connectivity information of each tetrahedral element

\end{itemize}

\end{description}\end{quote}

\end{fulllineitems}

\index{id (component.Component attribute)@\spxentry{id}\spxextra{component.Component attribute}}

\begin{fulllineitems}
\phantomsection\label{\detokenize{modules:component.Component.id}}
\pysigstartsignatures
\pysigline{\sphinxbfcode{\sphinxupquote{id}}}
\pysigstopsignatures
\sphinxAtStartPar
{[}int{]} ID of the component

\end{fulllineitems}

\index{inertia (component.Component attribute)@\spxentry{inertia}\spxextra{component.Component attribute}}

\begin{fulllineitems}
\phantomsection\label{\detokenize{modules:component.Component.inertia}}
\pysigstartsignatures
\pysigline{\sphinxbfcode{\sphinxupquote{inertia}}}
\pysigstopsignatures
\sphinxAtStartPar
{[}kg/m\textasciicircum{}2{]} Inertia matrix

\end{fulllineitems}

\index{mass (component.Component attribute)@\spxentry{mass}\spxextra{component.Component attribute}}

\begin{fulllineitems}
\phantomsection\label{\detokenize{modules:component.Component.mass}}
\pysigstartsignatures
\pysigline{\sphinxbfcode{\sphinxupquote{mass}}}
\pysigstopsignatures
\sphinxAtStartPar
{[}kg{]} Mass of the component

\end{fulllineitems}

\index{material (component.Component attribute)@\spxentry{material}\spxextra{component.Component attribute}}

\begin{fulllineitems}
\phantomsection\label{\detokenize{modules:component.Component.material}}
\pysigstartsignatures
\pysigline{\sphinxbfcode{\sphinxupquote{material}}}
\pysigstopsignatures
\sphinxAtStartPar
{[}Material{]} Object of class Material to store the material properties

\end{fulllineitems}

\index{mesh (component.Component attribute)@\spxentry{mesh}\spxextra{component.Component attribute}}

\begin{fulllineitems}
\phantomsection\label{\detokenize{modules:component.Component.mesh}}
\pysigstartsignatures
\pysigline{\sphinxbfcode{\sphinxupquote{mesh}}}
\pysigstopsignatures
\sphinxAtStartPar
{[}Mesh{]} Object of class mesh that stores the mesh information

\end{fulllineitems}

\index{name (component.Component attribute)@\spxentry{name}\spxextra{component.Component attribute}}

\begin{fulllineitems}
\phantomsection\label{\detokenize{modules:component.Component.name}}
\pysigstartsignatures
\pysigline{\sphinxbfcode{\sphinxupquote{name}}}
\pysigstopsignatures
\sphinxAtStartPar
{[}str{]} Name of the file where the mesh is stores

\end{fulllineitems}

\index{temperature (component.Component attribute)@\spxentry{temperature}\spxextra{component.Component attribute}}

\begin{fulllineitems}
\phantomsection\label{\detokenize{modules:component.Component.temperature}}
\pysigstartsignatures
\pysigline{\sphinxbfcode{\sphinxupquote{temperature}}}
\pysigstopsignatures
\sphinxAtStartPar
{[}K{]} Temperature

\end{fulllineitems}

\index{trigger\_type (component.Component attribute)@\spxentry{trigger\_type}\spxextra{component.Component attribute}}

\begin{fulllineitems}
\phantomsection\label{\detokenize{modules:component.Component.trigger_type}}
\pysigstartsignatures
\pysigline{\sphinxbfcode{\sphinxupquote{trigger\_type}}}
\pysigstopsignatures
\sphinxAtStartPar
{[}str{]} Type of trigger for type joint (Altitude, Temperature, Stress)

\end{fulllineitems}

\index{trigger\_value (component.Component attribute)@\spxentry{trigger\_value}\spxextra{component.Component attribute}}

\begin{fulllineitems}
\phantomsection\label{\detokenize{modules:component.Component.trigger_value}}
\pysigstartsignatures
\pysigline{\sphinxbfcode{\sphinxupquote{trigger\_value}}}
\pysigstopsignatures
\sphinxAtStartPar
{[}float{]} Value of the trigger criteria

\end{fulllineitems}

\index{type (component.Component attribute)@\spxentry{type}\spxextra{component.Component attribute}}

\begin{fulllineitems}
\phantomsection\label{\detokenize{modules:component.Component.type}}
\pysigstartsignatures
\pysigline{\sphinxbfcode{\sphinxupquote{type}}}
\pysigstopsignatures
\sphinxAtStartPar
{[}str{]} Type of the component (joint, primitive). Several sub\sphinxhyphen{}components can be used to form a larger component

\end{fulllineitems}


\end{fulllineitems}



\section{Material}
\label{\detokenize{modules:material}}\index{Material (class in material)@\spxentry{Material}\spxextra{class in material}}

\begin{fulllineitems}
\phantomsection\label{\detokenize{modules:material.Material}}
\pysigstartsignatures
\pysiglinewithargsret{\sphinxbfcode{\sphinxupquote{class\DUrole{w}{  }}}\sphinxcode{\sphinxupquote{material.}}\sphinxbfcode{\sphinxupquote{Material}}}{\emph{\DUrole{n}{name}}}{}
\pysigstopsignatures
\sphinxAtStartPar
Class Material

\sphinxAtStartPar
A class to store the material properties for each user\sphinxhyphen{}defined component
\index{density (material.Material attribute)@\spxentry{density}\spxextra{material.Material attribute}}

\begin{fulllineitems}
\phantomsection\label{\detokenize{modules:material.Material.density}}
\pysigstartsignatures
\pysigline{\sphinxbfcode{\sphinxupquote{density}}}
\pysigstopsignatures
\sphinxAtStartPar
{[}float{]} Density of the material

\end{fulllineitems}

\index{emissivity (material.Material attribute)@\spxentry{emissivity}\spxextra{material.Material attribute}}

\begin{fulllineitems}
\phantomsection\label{\detokenize{modules:material.Material.emissivity}}
\pysigstartsignatures
\pysigline{\sphinxbfcode{\sphinxupquote{emissivity}}}
\pysigstopsignatures
\sphinxAtStartPar
{[}float{]} Emissivity value

\end{fulllineitems}

\index{heatConductivity (material.Material attribute)@\spxentry{heatConductivity}\spxextra{material.Material attribute}}

\begin{fulllineitems}
\phantomsection\label{\detokenize{modules:material.Material.heatConductivity}}
\pysigstartsignatures
\pysigline{\sphinxbfcode{\sphinxupquote{heatConductivity}}}
\pysigstopsignatures
\sphinxAtStartPar
{[}float{]} Heat conductivity value

\end{fulllineitems}

\index{material\_density() (material.Material method)@\spxentry{material\_density()}\spxextra{material.Material method}}

\begin{fulllineitems}
\phantomsection\label{\detokenize{modules:material.Material.material_density}}
\pysigstartsignatures
\pysiglinewithargsret{\sphinxbfcode{\sphinxupquote{material\_density}}}{\emph{\DUrole{n}{index}}}{}
\pysigstopsignatures
\sphinxAtStartPar
Function to retrieve the material density
\begin{quote}\begin{description}
\sphinxlineitem{Returns}
\sphinxAtStartPar
\sphinxstylestrong{density} \textendash{} Return material density

\sphinxlineitem{Return type}
\sphinxAtStartPar
float

\end{description}\end{quote}

\end{fulllineitems}

\index{material\_emissivity() (material.Material method)@\spxentry{material\_emissivity()}\spxextra{material.Material method}}

\begin{fulllineitems}
\phantomsection\label{\detokenize{modules:material.Material.material_emissivity}}
\pysigstartsignatures
\pysiglinewithargsret{\sphinxbfcode{\sphinxupquote{material\_emissivity}}}{\emph{\DUrole{n}{index}}}{}
\pysigstopsignatures
\sphinxAtStartPar
Function to retrieve the emissivity value
\begin{quote}\begin{description}
\sphinxlineitem{Returns}
\sphinxAtStartPar
\sphinxstylestrong{emissivity} \textendash{} Return emissivity value

\sphinxlineitem{Return type}
\sphinxAtStartPar
float

\end{description}\end{quote}

\end{fulllineitems}

\index{material\_heatConductivity() (material.Material method)@\spxentry{material\_heatConductivity()}\spxextra{material.Material method}}

\begin{fulllineitems}
\phantomsection\label{\detokenize{modules:material.Material.material_heatConductivity}}
\pysigstartsignatures
\pysiglinewithargsret{\sphinxbfcode{\sphinxupquote{material\_heatConductivity}}}{\emph{\DUrole{n}{index}}}{}
\pysigstopsignatures
\sphinxAtStartPar
Function to retrieve the material heat conductivity
\begin{quote}\begin{description}
\sphinxlineitem{Returns}
\sphinxAtStartPar
\sphinxstylestrong{heatConductivity} \textendash{} Return interpolation function for the heat conductivity

\sphinxlineitem{Return type}
\sphinxAtStartPar
scipy.interpolate.interp1d

\end{description}\end{quote}

\end{fulllineitems}

\index{material\_meltingHeat() (material.Material method)@\spxentry{material\_meltingHeat()}\spxextra{material.Material method}}

\begin{fulllineitems}
\phantomsection\label{\detokenize{modules:material.Material.material_meltingHeat}}
\pysigstartsignatures
\pysiglinewithargsret{\sphinxbfcode{\sphinxupquote{material\_meltingHeat}}}{\emph{\DUrole{n}{index}}}{}
\pysigstopsignatures
\sphinxAtStartPar
Function to retrieve the melting Heat value
\begin{quote}\begin{description}
\sphinxlineitem{Returns}
\sphinxAtStartPar
\sphinxstylestrong{meltingHeat} \textendash{} Return melting heat value

\sphinxlineitem{Return type}
\sphinxAtStartPar
float

\end{description}\end{quote}

\end{fulllineitems}

\index{material\_meltingTemperature() (material.Material method)@\spxentry{material\_meltingTemperature()}\spxextra{material.Material method}}

\begin{fulllineitems}
\phantomsection\label{\detokenize{modules:material.Material.material_meltingTemperature}}
\pysigstartsignatures
\pysiglinewithargsret{\sphinxbfcode{\sphinxupquote{material\_meltingTemperature}}}{\emph{\DUrole{n}{index}}}{}
\pysigstopsignatures
\sphinxAtStartPar
Function to retrieve the melting temperature value
\begin{quote}\begin{description}
\sphinxlineitem{Returns}
\sphinxAtStartPar
\sphinxstylestrong{meltingTemperature} \textendash{} Return melting temperature value

\sphinxlineitem{Return type}
\sphinxAtStartPar
float

\end{description}\end{quote}

\end{fulllineitems}

\index{material\_name() (material.Material method)@\spxentry{material\_name()}\spxextra{material.Material method}}

\begin{fulllineitems}
\phantomsection\label{\detokenize{modules:material.Material.material_name}}
\pysigstartsignatures
\pysiglinewithargsret{\sphinxbfcode{\sphinxupquote{material\_name}}}{\emph{\DUrole{n}{index}}}{}
\pysigstopsignatures
\sphinxAtStartPar
Function to retrieve the material name
\begin{quote}\begin{description}
\sphinxlineitem{Returns}
\sphinxAtStartPar
\sphinxstylestrong{name} \textendash{} Return material name

\sphinxlineitem{Return type}
\sphinxAtStartPar
str

\end{description}\end{quote}

\end{fulllineitems}

\index{material\_oxideActivationTemperature() (material.Material method)@\spxentry{material\_oxideActivationTemperature()}\spxextra{material.Material method}}

\begin{fulllineitems}
\phantomsection\label{\detokenize{modules:material.Material.material_oxideActivationTemperature}}
\pysigstartsignatures
\pysiglinewithargsret{\sphinxbfcode{\sphinxupquote{material\_oxideActivationTemperature}}}{\emph{\DUrole{n}{index}}}{}
\pysigstopsignatures
\sphinxAtStartPar
Function to retrieve the oxide activation Temperatire
\begin{quote}\begin{description}
\sphinxlineitem{Returns}
\sphinxAtStartPar
\sphinxstylestrong{oxideActivationTemperature} \textendash{} Return oxide activation temperature value

\sphinxlineitem{Return type}
\sphinxAtStartPar
float

\end{description}\end{quote}

\end{fulllineitems}

\index{material\_oxideEmissivity() (material.Material method)@\spxentry{material\_oxideEmissivity()}\spxextra{material.Material method}}

\begin{fulllineitems}
\phantomsection\label{\detokenize{modules:material.Material.material_oxideEmissivity}}
\pysigstartsignatures
\pysiglinewithargsret{\sphinxbfcode{\sphinxupquote{material\_oxideEmissivity}}}{\emph{\DUrole{n}{index}}}{}
\pysigstopsignatures
\sphinxAtStartPar
Function to retrieve the material oxide emissivity
\begin{quote}\begin{description}
\sphinxlineitem{Returns}
\sphinxAtStartPar
\sphinxstylestrong{oxideEmissivity} \textendash{} Return interpolation function for the oxide emissivity

\sphinxlineitem{Return type}
\sphinxAtStartPar
scipy.interpolate.interp1d

\end{description}\end{quote}

\end{fulllineitems}

\index{material\_oxideHeatOfFormation() (material.Material method)@\spxentry{material\_oxideHeatOfFormation()}\spxextra{material.Material method}}

\begin{fulllineitems}
\phantomsection\label{\detokenize{modules:material.Material.material_oxideHeatOfFormation}}
\pysigstartsignatures
\pysiglinewithargsret{\sphinxbfcode{\sphinxupquote{material\_oxideHeatOfFormation}}}{\emph{\DUrole{n}{index}}}{}
\pysigstopsignatures
\sphinxAtStartPar
Function to retrieve the oxide heat of formation
\begin{quote}\begin{description}
\sphinxlineitem{Returns}
\sphinxAtStartPar
\sphinxstylestrong{oxideHeatofFormation} \textendash{} Return oxide heat of formation value

\sphinxlineitem{Return type}
\sphinxAtStartPar
float

\end{description}\end{quote}

\end{fulllineitems}

\index{material\_oxideReactionProbability() (material.Material method)@\spxentry{material\_oxideReactionProbability()}\spxextra{material.Material method}}

\begin{fulllineitems}
\phantomsection\label{\detokenize{modules:material.Material.material_oxideReactionProbability}}
\pysigstartsignatures
\pysiglinewithargsret{\sphinxbfcode{\sphinxupquote{material\_oxideReactionProbability}}}{\emph{\DUrole{n}{index}}}{}
\pysigstopsignatures
\sphinxAtStartPar
Function to retrieve the oxide reaction probability
\begin{quote}\begin{description}
\sphinxlineitem{Returns}
\sphinxAtStartPar
\sphinxstylestrong{oxideReactionProbability} \textendash{} Return oxide reaction probability

\sphinxlineitem{Return type}
\sphinxAtStartPar
float

\end{description}\end{quote}

\end{fulllineitems}

\index{material\_specificHeatCapacity() (material.Material method)@\spxentry{material\_specificHeatCapacity()}\spxextra{material.Material method}}

\begin{fulllineitems}
\phantomsection\label{\detokenize{modules:material.Material.material_specificHeatCapacity}}
\pysigstartsignatures
\pysiglinewithargsret{\sphinxbfcode{\sphinxupquote{material\_specificHeatCapacity}}}{\emph{\DUrole{n}{index}}}{}
\pysigstopsignatures
\sphinxAtStartPar
Function to retrieve the material specific heat capacity
\begin{quote}\begin{description}
\sphinxlineitem{Returns}
\sphinxAtStartPar
\sphinxstylestrong{specificHeatCapacity} \textendash{} Return interpolation function for the specific heat capacity

\sphinxlineitem{Return type}
\sphinxAtStartPar
scipy.interpolate.interp1d

\end{description}\end{quote}

\end{fulllineitems}

\index{material\_yieldStress() (material.Material method)@\spxentry{material\_yieldStress()}\spxextra{material.Material method}}

\begin{fulllineitems}
\phantomsection\label{\detokenize{modules:material.Material.material_yieldStress}}
\pysigstartsignatures
\pysiglinewithargsret{\sphinxbfcode{\sphinxupquote{material\_yieldStress}}}{\emph{\DUrole{n}{index}}}{}
\pysigstopsignatures
\sphinxAtStartPar
Function to retrieve the material yield stress
\begin{quote}\begin{description}
\sphinxlineitem{Returns}
\sphinxAtStartPar
\sphinxstylestrong{yieldStress} \textendash{} Return interpolation function for the yield Stress

\sphinxlineitem{Return type}
\sphinxAtStartPar
scipy.interpolate.interp1d

\end{description}\end{quote}

\end{fulllineitems}

\index{material\_youngModulus() (material.Material method)@\spxentry{material\_youngModulus()}\spxextra{material.Material method}}

\begin{fulllineitems}
\phantomsection\label{\detokenize{modules:material.Material.material_youngModulus}}
\pysigstartsignatures
\pysiglinewithargsret{\sphinxbfcode{\sphinxupquote{material\_youngModulus}}}{\emph{\DUrole{n}{index}}}{}
\pysigstopsignatures
\sphinxAtStartPar
Function to retrieve the young Modulus
\begin{quote}\begin{description}
\sphinxlineitem{Returns}
\sphinxAtStartPar
\sphinxstylestrong{youngModulus} \textendash{} Return interpolation function for the young Modulus

\sphinxlineitem{Return type}
\sphinxAtStartPar
scipy.interpolate.interp1d

\end{description}\end{quote}

\end{fulllineitems}

\index{meltingHeat (material.Material attribute)@\spxentry{meltingHeat}\spxextra{material.Material attribute}}

\begin{fulllineitems}
\phantomsection\label{\detokenize{modules:material.Material.meltingHeat}}
\pysigstartsignatures
\pysigline{\sphinxbfcode{\sphinxupquote{meltingHeat}}}
\pysigstopsignatures
\sphinxAtStartPar
{[}float{]} Melting Heat value of the material

\end{fulllineitems}

\index{meltingTemperature (material.Material attribute)@\spxentry{meltingTemperature}\spxextra{material.Material attribute}}

\begin{fulllineitems}
\phantomsection\label{\detokenize{modules:material.Material.meltingTemperature}}
\pysigstartsignatures
\pysigline{\sphinxbfcode{\sphinxupquote{meltingTemperature}}}
\pysigstopsignatures
\sphinxAtStartPar
{[}float{]} Melting Temperature value

\end{fulllineitems}

\index{name (material.Material attribute)@\spxentry{name}\spxextra{material.Material attribute}}

\begin{fulllineitems}
\phantomsection\label{\detokenize{modules:material.Material.name}}
\pysigstartsignatures
\pysigline{\sphinxbfcode{\sphinxupquote{name}}}
\pysigstopsignatures
\sphinxAtStartPar
{[}str{]} Name of the material

\end{fulllineitems}

\index{oxideActivationTemperature (material.Material attribute)@\spxentry{oxideActivationTemperature}\spxextra{material.Material attribute}}

\begin{fulllineitems}
\phantomsection\label{\detokenize{modules:material.Material.oxideActivationTemperature}}
\pysigstartsignatures
\pysigline{\sphinxbfcode{\sphinxupquote{oxideActivationTemperature}}}
\pysigstopsignatures
\sphinxAtStartPar
{[}float{]} Oxidation activation temperature value

\end{fulllineitems}

\index{oxideEmissivity (material.Material attribute)@\spxentry{oxideEmissivity}\spxextra{material.Material attribute}}

\begin{fulllineitems}
\phantomsection\label{\detokenize{modules:material.Material.oxideEmissivity}}
\pysigstartsignatures
\pysigline{\sphinxbfcode{\sphinxupquote{oxideEmissivity}}}
\pysigstopsignatures
\sphinxAtStartPar
{[}float{]} Oxidation emissivity value

\end{fulllineitems}

\index{oxideHeatOfFormation (material.Material attribute)@\spxentry{oxideHeatOfFormation}\spxextra{material.Material attribute}}

\begin{fulllineitems}
\phantomsection\label{\detokenize{modules:material.Material.oxideHeatOfFormation}}
\pysigstartsignatures
\pysigline{\sphinxbfcode{\sphinxupquote{oxideHeatOfFormation}}}
\pysigstopsignatures
\sphinxAtStartPar
{[}float{]} Oxidation Formation Heat value

\end{fulllineitems}

\index{oxideReactionProbability (material.Material attribute)@\spxentry{oxideReactionProbability}\spxextra{material.Material attribute}}

\begin{fulllineitems}
\phantomsection\label{\detokenize{modules:material.Material.oxideReactionProbability}}
\pysigstartsignatures
\pysigline{\sphinxbfcode{\sphinxupquote{oxideReactionProbability}}}
\pysigstopsignatures
\sphinxAtStartPar
{[}float{]} Oxidation reaction probability value

\end{fulllineitems}

\index{specificHeatCapacity (material.Material attribute)@\spxentry{specificHeatCapacity}\spxextra{material.Material attribute}}

\begin{fulllineitems}
\phantomsection\label{\detokenize{modules:material.Material.specificHeatCapacity}}
\pysigstartsignatures
\pysigline{\sphinxbfcode{\sphinxupquote{specificHeatCapacity}}}
\pysigstopsignatures
\sphinxAtStartPar
{[}float{]} Specific Heat Capacity value of the material

\end{fulllineitems}

\index{yieldStress (material.Material attribute)@\spxentry{yieldStress}\spxextra{material.Material attribute}}

\begin{fulllineitems}
\phantomsection\label{\detokenize{modules:material.Material.yieldStress}}
\pysigstartsignatures
\pysigline{\sphinxbfcode{\sphinxupquote{yieldStress}}}
\pysigstopsignatures
\sphinxAtStartPar
{[}float{]} Yield Stress value

\end{fulllineitems}

\index{youngModulus (material.Material attribute)@\spxentry{youngModulus}\spxextra{material.Material attribute}}

\begin{fulllineitems}
\phantomsection\label{\detokenize{modules:material.Material.youngModulus}}
\pysigstartsignatures
\pysigline{\sphinxbfcode{\sphinxupquote{youngModulus}}}
\pysigstopsignatures
\sphinxAtStartPar
{[}float{]} Young Modulus value

\end{fulllineitems}


\end{fulllineitems}



\section{SU2}
\label{\detokenize{modules:su2}}\index{Solver (class in su2)@\spxentry{Solver}\spxextra{class in su2}}

\begin{fulllineitems}
\phantomsection\label{\detokenize{modules:su2.Solver}}
\pysigstartsignatures
\pysiglinewithargsret{\sphinxbfcode{\sphinxupquote{class\DUrole{w}{  }}}\sphinxcode{\sphinxupquote{su2.}}\sphinxbfcode{\sphinxupquote{Solver}}}{\emph{\DUrole{n}{restart}}, \emph{\DUrole{n}{su2}}, \emph{\DUrole{n}{freestream}}}{}
\pysigstopsignatures
\sphinxAtStartPar
Class Solver

\sphinxAtStartPar
A class to store the solver information.
The class in the su2.py file is hardcoded to work with SU2.
\index{fluid\_model (su2.Solver attribute)@\spxentry{fluid\_model}\spxextra{su2.Solver attribute}}

\begin{fulllineitems}
\phantomsection\label{\detokenize{modules:su2.Solver.fluid_model}}
\pysigstartsignatures
\pysigline{\sphinxbfcode{\sphinxupquote{fluid\_model}}}
\pysigstopsignatures
\sphinxAtStartPar
{[}str{]} Fluid Model (Fluid model = MUTATIONPP \sphinxhyphen{}\textgreater{} Uses the Mutationpp Library)

\end{fulllineitems}

\index{gas\_composition (su2.Solver attribute)@\spxentry{gas\_composition}\spxextra{su2.Solver attribute}}

\begin{fulllineitems}
\phantomsection\label{\detokenize{modules:su2.Solver.gas_composition}}
\pysigstartsignatures
\pysigline{\sphinxbfcode{\sphinxupquote{gas\_composition}}}
\pysigstopsignatures
\sphinxAtStartPar
{[}str{]} Gas Composition

\end{fulllineitems}

\index{gas\_model (su2.Solver attribute)@\spxentry{gas\_model}\spxextra{su2.Solver attribute}}

\begin{fulllineitems}
\phantomsection\label{\detokenize{modules:su2.Solver.gas_model}}
\pysigstartsignatures
\pysigline{\sphinxbfcode{\sphinxupquote{gas\_model}}}
\pysigstopsignatures
\sphinxAtStartPar
{[}str{]} Gas Model (Gas to be used in the simulation)

\end{fulllineitems}

\index{kind\_turb\_model (su2.Solver attribute)@\spxentry{kind\_turb\_model}\spxextra{su2.Solver attribute}}

\begin{fulllineitems}
\phantomsection\label{\detokenize{modules:su2.Solver.kind_turb_model}}
\pysigstartsignatures
\pysigline{\sphinxbfcode{\sphinxupquote{kind\_turb\_model}}}
\pysigstopsignatures
\sphinxAtStartPar
{[}str{]} Turbulence Model (Default = NONE)

\end{fulllineitems}

\index{restart (su2.Solver attribute)@\spxentry{restart}\spxextra{su2.Solver attribute}}

\begin{fulllineitems}
\phantomsection\label{\detokenize{modules:su2.Solver.restart}}
\pysigstartsignatures
\pysigline{\sphinxbfcode{\sphinxupquote{restart}}}
\pysigstopsignatures
\sphinxAtStartPar
{[}str{]} Restart boolean (if YES, the CFD simulation will restart from previous solution)

\end{fulllineitems}

\index{solver (su2.Solver attribute)@\spxentry{solver}\spxextra{su2.Solver attribute}}

\begin{fulllineitems}
\phantomsection\label{\detokenize{modules:su2.Solver.solver}}
\pysigstartsignatures
\pysigline{\sphinxbfcode{\sphinxupquote{solver}}}
\pysigstopsignatures
\sphinxAtStartPar
{[}str{]} Solver (EULER, NAVIER\sphinxhyphen{}STOKES, NEMO\_EULER, NEMO\_NAVIER\_STOKES)

\end{fulllineitems}

\index{transport\_coeff (su2.Solver attribute)@\spxentry{transport\_coeff}\spxextra{su2.Solver attribute}}

\begin{fulllineitems}
\phantomsection\label{\detokenize{modules:su2.Solver.transport_coeff}}
\pysigstartsignatures
\pysigline{\sphinxbfcode{\sphinxupquote{transport\_coeff}}}
\pysigstopsignatures
\sphinxAtStartPar
{[}str{]} Transport Coefficient

\end{fulllineitems}


\end{fulllineitems}

\index{Solver\_Freestream\_Conditions (class in su2)@\spxentry{Solver\_Freestream\_Conditions}\spxextra{class in su2}}

\begin{fulllineitems}
\phantomsection\label{\detokenize{modules:su2.Solver_Freestream_Conditions}}
\pysigstartsignatures
\pysiglinewithargsret{\sphinxbfcode{\sphinxupquote{class\DUrole{w}{  }}}\sphinxcode{\sphinxupquote{su2.}}\sphinxbfcode{\sphinxupquote{Solver\_Freestream\_Conditions}}}{\emph{\DUrole{n}{freestream}}}{}
\pysigstopsignatures
\sphinxAtStartPar
Class Solver Freestream Conditions

\sphinxAtStartPar
A class to store the freestream conditions used in the CFD simulation
The class in the su2.py file is hardcoded to work with SU2.
\index{aoa (su2.Solver\_Freestream\_Conditions attribute)@\spxentry{aoa}\spxextra{su2.Solver\_Freestream\_Conditions attribute}}

\begin{fulllineitems}
\phantomsection\label{\detokenize{modules:su2.Solver_Freestream_Conditions.aoa}}
\pysigstartsignatures
\pysigline{\sphinxbfcode{\sphinxupquote{aoa}}}
\pysigstopsignatures
\sphinxAtStartPar
{[}str{]} Angle of attack in deg

\end{fulllineitems}

\index{init\_option (su2.Solver\_Freestream\_Conditions attribute)@\spxentry{init\_option}\spxextra{su2.Solver\_Freestream\_Conditions attribute}}

\begin{fulllineitems}
\phantomsection\label{\detokenize{modules:su2.Solver_Freestream_Conditions.init_option}}
\pysigstartsignatures
\pysigline{\sphinxbfcode{\sphinxupquote{init\_option}}}
\pysigstopsignatures
\sphinxAtStartPar
{[}str{]} Initialization option to be used to compute the freestream (Default = TD\_CONDITIONS)

\end{fulllineitems}

\index{mach (su2.Solver\_Freestream\_Conditions attribute)@\spxentry{mach}\spxextra{su2.Solver\_Freestream\_Conditions attribute}}

\begin{fulllineitems}
\phantomsection\label{\detokenize{modules:su2.Solver_Freestream_Conditions.mach}}
\pysigstartsignatures
\pysigline{\sphinxbfcode{\sphinxupquote{mach}}}
\pysigstopsignatures
\sphinxAtStartPar
{[}str{]} Mach number

\end{fulllineitems}

\index{pressure (su2.Solver\_Freestream\_Conditions attribute)@\spxentry{pressure}\spxextra{su2.Solver\_Freestream\_Conditions attribute}}

\begin{fulllineitems}
\phantomsection\label{\detokenize{modules:su2.Solver_Freestream_Conditions.pressure}}
\pysigstartsignatures
\pysigline{\sphinxbfcode{\sphinxupquote{pressure}}}
\pysigstopsignatures
\sphinxAtStartPar
{[}str{]} Freestream Pressure in Pa

\end{fulllineitems}

\index{temperature (su2.Solver\_Freestream\_Conditions attribute)@\spxentry{temperature}\spxextra{su2.Solver\_Freestream\_Conditions attribute}}

\begin{fulllineitems}
\phantomsection\label{\detokenize{modules:su2.Solver_Freestream_Conditions.temperature}}
\pysigstartsignatures
\pysigline{\sphinxbfcode{\sphinxupquote{temperature}}}
\pysigstopsignatures
\sphinxAtStartPar
{[}str{]} Freestream Temperature in K

\end{fulllineitems}


\end{fulllineitems}

\index{Solver\_Reference\_Value (class in su2)@\spxentry{Solver\_Reference\_Value}\spxextra{class in su2}}

\begin{fulllineitems}
\phantomsection\label{\detokenize{modules:su2.Solver_Reference_Value}}
\pysigstartsignatures
\pysigline{\sphinxbfcode{\sphinxupquote{class\DUrole{w}{  }}}\sphinxcode{\sphinxupquote{su2.}}\sphinxbfcode{\sphinxupquote{Solver\_Reference\_Value}}}
\pysigstopsignatures
\sphinxAtStartPar
Class Solver Reference value

\sphinxAtStartPar
A class to store the reference values for the coefficient and moment computation
The class in the su2.py file is hardcoded to work with SU2.
\index{origin\_moment\_x (su2.Solver\_Reference\_Value attribute)@\spxentry{origin\_moment\_x}\spxextra{su2.Solver\_Reference\_Value attribute}}

\begin{fulllineitems}
\phantomsection\label{\detokenize{modules:su2.Solver_Reference_Value.origin_moment_x}}
\pysigstartsignatures
\pysigline{\sphinxbfcode{\sphinxupquote{origin\_moment\_x}}}
\pysigstopsignatures
\sphinxAtStartPar
{[}str{]} x\sphinxhyphen{}coordinate to which the moment is computed

\end{fulllineitems}

\index{origin\_moment\_y (su2.Solver\_Reference\_Value attribute)@\spxentry{origin\_moment\_y}\spxextra{su2.Solver\_Reference\_Value attribute}}

\begin{fulllineitems}
\phantomsection\label{\detokenize{modules:su2.Solver_Reference_Value.origin_moment_y}}
\pysigstartsignatures
\pysigline{\sphinxbfcode{\sphinxupquote{origin\_moment\_y}}}
\pysigstopsignatures
\sphinxAtStartPar
{[}str{]} y\sphinxhyphen{}coordinate to which the moment is computed

\end{fulllineitems}

\index{origin\_moment\_z (su2.Solver\_Reference\_Value attribute)@\spxentry{origin\_moment\_z}\spxextra{su2.Solver\_Reference\_Value attribute}}

\begin{fulllineitems}
\phantomsection\label{\detokenize{modules:su2.Solver_Reference_Value.origin_moment_z}}
\pysigstartsignatures
\pysigline{\sphinxbfcode{\sphinxupquote{origin\_moment\_z}}}
\pysigstopsignatures
\sphinxAtStartPar
{[}str{]} z\sphinxhyphen{}coordinate to which the moment is computed

\end{fulllineitems}

\index{ref\_area (su2.Solver\_Reference\_Value attribute)@\spxentry{ref\_area}\spxextra{su2.Solver\_Reference\_Value attribute}}

\begin{fulllineitems}
\phantomsection\label{\detokenize{modules:su2.Solver_Reference_Value.ref_area}}
\pysigstartsignatures
\pysigline{\sphinxbfcode{\sphinxupquote{ref\_area}}}
\pysigstopsignatures
\sphinxAtStartPar
{[}str{]} Area of reference

\end{fulllineitems}

\index{ref\_length (su2.Solver\_Reference\_Value attribute)@\spxentry{ref\_length}\spxextra{su2.Solver\_Reference\_Value attribute}}

\begin{fulllineitems}
\phantomsection\label{\detokenize{modules:su2.Solver_Reference_Value.ref_length}}
\pysigstartsignatures
\pysigline{\sphinxbfcode{\sphinxupquote{ref\_length}}}
\pysigstopsignatures
\sphinxAtStartPar
{[}str{]} Length of reference

\end{fulllineitems}


\end{fulllineitems}

\index{Solver\_BC (class in su2)@\spxentry{Solver\_BC}\spxextra{class in su2}}

\begin{fulllineitems}
\phantomsection\label{\detokenize{modules:su2.Solver_BC}}
\pysigstartsignatures
\pysiglinewithargsret{\sphinxbfcode{\sphinxupquote{class\DUrole{w}{  }}}\sphinxcode{\sphinxupquote{su2.}}\sphinxbfcode{\sphinxupquote{Solver\_BC}}}{\emph{\DUrole{n}{assembly}}, \emph{\DUrole{n}{su2}}}{}
\pysigstopsignatures
\sphinxAtStartPar
Class Solver Boundary conditions

\sphinxAtStartPar
A class to store the applied boundary conditions
The class in the su2.py file is hardcoded to work with SU2.
\index{euler (su2.Solver\_BC attribute)@\spxentry{euler}\spxextra{su2.Solver\_BC attribute}}

\begin{fulllineitems}
\phantomsection\label{\detokenize{modules:su2.Solver_BC.euler}}
\pysigstartsignatures
\pysigline{\sphinxbfcode{\sphinxupquote{euler}}}
\pysigstopsignatures
\sphinxAtStartPar
{[}str{]} Euler Marker

\end{fulllineitems}

\index{farfield (su2.Solver\_BC attribute)@\spxentry{farfield}\spxextra{su2.Solver\_BC attribute}}

\begin{fulllineitems}
\phantomsection\label{\detokenize{modules:su2.Solver_BC.farfield}}
\pysigstartsignatures
\pysigline{\sphinxbfcode{\sphinxupquote{farfield}}}
\pysigstopsignatures
\sphinxAtStartPar
{[}str{]} Farfield Marker

\end{fulllineitems}

\index{iso (su2.Solver\_BC attribute)@\spxentry{iso}\spxextra{su2.Solver\_BC attribute}}

\begin{fulllineitems}
\phantomsection\label{\detokenize{modules:su2.Solver_BC.iso}}
\pysigstartsignatures
\pysigline{\sphinxbfcode{\sphinxupquote{iso}}}
\pysigstopsignatures
\sphinxAtStartPar
{[}str{]} Isothermal Marker

\end{fulllineitems}

\index{monitor (su2.Solver\_BC attribute)@\spxentry{monitor}\spxextra{su2.Solver\_BC attribute}}

\begin{fulllineitems}
\phantomsection\label{\detokenize{modules:su2.Solver_BC.monitor}}
\pysigstartsignatures
\pysigline{\sphinxbfcode{\sphinxupquote{monitor}}}
\pysigstopsignatures
\sphinxAtStartPar
{[}str{]} Monitoring Marker

\end{fulllineitems}

\index{outlet (su2.Solver\_BC attribute)@\spxentry{outlet}\spxextra{su2.Solver\_BC attribute}}

\begin{fulllineitems}
\phantomsection\label{\detokenize{modules:su2.Solver_BC.outlet}}
\pysigstartsignatures
\pysigline{\sphinxbfcode{\sphinxupquote{outlet}}}
\pysigstopsignatures
\sphinxAtStartPar
{[}str{]} Outlet Marker

\end{fulllineitems}

\index{plot (su2.Solver\_BC attribute)@\spxentry{plot}\spxextra{su2.Solver\_BC attribute}}

\begin{fulllineitems}
\phantomsection\label{\detokenize{modules:su2.Solver_BC.plot}}
\pysigstartsignatures
\pysigline{\sphinxbfcode{\sphinxupquote{plot}}}
\pysigstopsignatures
\sphinxAtStartPar
{[}str{]} Plot Marker

\end{fulllineitems}


\end{fulllineitems}

\index{Solver\_Numerical\_Method (class in su2)@\spxentry{Solver\_Numerical\_Method}\spxextra{class in su2}}

\begin{fulllineitems}
\phantomsection\label{\detokenize{modules:su2.Solver_Numerical_Method}}
\pysigstartsignatures
\pysiglinewithargsret{\sphinxbfcode{\sphinxupquote{class\DUrole{w}{  }}}\sphinxcode{\sphinxupquote{su2.}}\sphinxbfcode{\sphinxupquote{Solver\_Numerical\_Method}}}{\emph{\DUrole{n}{su2}}}{}
\pysigstopsignatures
\sphinxAtStartPar
Class Solver Numerical Method

\sphinxAtStartPar
A class to store the solver numerical methods
The class in the su2.py file is hardcoded to work with SU2.

\end{fulllineitems}

\index{Flow\_Numerical\_Method (class in su2)@\spxentry{Flow\_Numerical\_Method}\spxextra{class in su2}}

\begin{fulllineitems}
\phantomsection\label{\detokenize{modules:su2.Flow_Numerical_Method}}
\pysigstartsignatures
\pysiglinewithargsret{\sphinxbfcode{\sphinxupquote{class\DUrole{w}{  }}}\sphinxcode{\sphinxupquote{su2.}}\sphinxbfcode{\sphinxupquote{Flow\_Numerical\_Method}}}{\emph{\DUrole{n}{su2}}}{}
\pysigstopsignatures
\sphinxAtStartPar
Class Flow Numerical Method

\sphinxAtStartPar
A class to store the flow numerical methods
The class in the su2.py file is hardcoded to work with SU2.
\index{conv\_method (su2.Flow\_Numerical\_Method attribute)@\spxentry{conv\_method}\spxextra{su2.Flow\_Numerical\_Method attribute}}

\begin{fulllineitems}
\phantomsection\label{\detokenize{modules:su2.Flow_Numerical_Method.conv_method}}
\pysigstartsignatures
\pysigline{\sphinxbfcode{\sphinxupquote{conv\_method}}}
\pysigstopsignatures
\sphinxAtStartPar
{[}str{]} Convective method (AUSM, AUSMPLUSUP2)

\end{fulllineitems}

\index{limiter (su2.Flow\_Numerical\_Method attribute)@\spxentry{limiter}\spxextra{su2.Flow\_Numerical\_Method attribute}}

\begin{fulllineitems}
\phantomsection\label{\detokenize{modules:su2.Flow_Numerical_Method.limiter}}
\pysigstartsignatures
\pysigline{\sphinxbfcode{\sphinxupquote{limiter}}}
\pysigstopsignatures
\sphinxAtStartPar
{[}str{]} Limiter method (Default = VENKATAKRISHNAN\_WANG)

\end{fulllineitems}

\index{limiter\_coeff (su2.Flow\_Numerical\_Method attribute)@\spxentry{limiter\_coeff}\spxextra{su2.Flow\_Numerical\_Method attribute}}

\begin{fulllineitems}
\phantomsection\label{\detokenize{modules:su2.Flow_Numerical_Method.limiter_coeff}}
\pysigstartsignatures
\pysigline{\sphinxbfcode{\sphinxupquote{limiter\_coeff}}}
\pysigstopsignatures
\sphinxAtStartPar
{[}str{]} Limiter coefficiet (Default = 0.01)

\end{fulllineitems}

\index{muscl (su2.Flow\_Numerical\_Method attribute)@\spxentry{muscl}\spxextra{su2.Flow\_Numerical\_Method attribute}}

\begin{fulllineitems}
\phantomsection\label{\detokenize{modules:su2.Flow_Numerical_Method.muscl}}
\pysigstartsignatures
\pysigline{\sphinxbfcode{\sphinxupquote{muscl}}}
\pysigstopsignatures
\sphinxAtStartPar
{[}str{]} MUSCL activation boolean (Default = YES)

\end{fulllineitems}

\index{time (su2.Flow\_Numerical\_Method attribute)@\spxentry{time}\spxextra{su2.Flow\_Numerical\_Method attribute}}

\begin{fulllineitems}
\phantomsection\label{\detokenize{modules:su2.Flow_Numerical_Method.time}}
\pysigstartsignatures
\pysigline{\sphinxbfcode{\sphinxupquote{time}}}
\pysigstopsignatures
\sphinxAtStartPar
{[}str{]} Time discretization (Default = EULER\_EXPLICIT)

\end{fulllineitems}


\end{fulllineitems}

\index{Solver\_Convergence (class in su2)@\spxentry{Solver\_Convergence}\spxextra{class in su2}}

\begin{fulllineitems}
\phantomsection\label{\detokenize{modules:su2.Solver_Convergence}}
\pysigstartsignatures
\pysigline{\sphinxbfcode{\sphinxupquote{class\DUrole{w}{  }}}\sphinxcode{\sphinxupquote{su2.}}\sphinxbfcode{\sphinxupquote{Solver\_Convergence}}}
\pysigstopsignatures
\sphinxAtStartPar
Class Solver convergence

\sphinxAtStartPar
A class to store the convergence criteria
The class in the su2.py file is hardcoded to work with SU2.
\index{cauchy\_elems (su2.Solver\_Convergence attribute)@\spxentry{cauchy\_elems}\spxextra{su2.Solver\_Convergence attribute}}

\begin{fulllineitems}
\phantomsection\label{\detokenize{modules:su2.Solver_Convergence.cauchy_elems}}
\pysigstartsignatures
\pysigline{\sphinxbfcode{\sphinxupquote{cauchy\_elems}}}
\pysigstopsignatures
\sphinxAtStartPar
{[}str{]} Number of elements to be used in the Cauchy convergence window

\end{fulllineitems}

\index{cauchy\_eps (su2.Solver\_Convergence attribute)@\spxentry{cauchy\_eps}\spxextra{su2.Solver\_Convergence attribute}}

\begin{fulllineitems}
\phantomsection\label{\detokenize{modules:su2.Solver_Convergence.cauchy_eps}}
\pysigstartsignatures
\pysigline{\sphinxbfcode{\sphinxupquote{cauchy\_eps}}}
\pysigstopsignatures
\sphinxAtStartPar
{[}str{]} Residual for convergence using the Cauchy Window

\end{fulllineitems}

\index{field (su2.Solver\_Convergence attribute)@\spxentry{field}\spxextra{su2.Solver\_Convergence attribute}}

\begin{fulllineitems}
\phantomsection\label{\detokenize{modules:su2.Solver_Convergence.field}}
\pysigstartsignatures
\pysigline{\sphinxbfcode{\sphinxupquote{field}}}
\pysigstopsignatures
\sphinxAtStartPar
{[}str{]} Fields to look for convergence

\end{fulllineitems}

\index{res\_min (su2.Solver\_Convergence attribute)@\spxentry{res\_min}\spxextra{su2.Solver\_Convergence attribute}}

\begin{fulllineitems}
\phantomsection\label{\detokenize{modules:su2.Solver_Convergence.res_min}}
\pysigstartsignatures
\pysigline{\sphinxbfcode{\sphinxupquote{res\_min}}}
\pysigstopsignatures
\sphinxAtStartPar
{[}str{]} Minimum residual for convergence

\end{fulllineitems}

\index{start\_iter (su2.Solver\_Convergence attribute)@\spxentry{start\_iter}\spxextra{su2.Solver\_Convergence attribute}}

\begin{fulllineitems}
\phantomsection\label{\detokenize{modules:su2.Solver_Convergence.start_iter}}
\pysigstartsignatures
\pysigline{\sphinxbfcode{\sphinxupquote{start\_iter}}}
\pysigstopsignatures
\sphinxAtStartPar
{[}str{]} Start iteration for the convergence assessment

\end{fulllineitems}


\end{fulllineitems}

\index{Solver\_Input\_Output (class in su2)@\spxentry{Solver\_Input\_Output}\spxextra{class in su2}}

\begin{fulllineitems}
\phantomsection\label{\detokenize{modules:su2.Solver_Input_Output}}
\pysigstartsignatures
\pysiglinewithargsret{\sphinxbfcode{\sphinxupquote{class\DUrole{w}{  }}}\sphinxcode{\sphinxupquote{su2.}}\sphinxbfcode{\sphinxupquote{Solver\_Input\_Output}}}{\emph{\DUrole{n}{it}}, \emph{\DUrole{n}{iteration}}, \emph{\DUrole{n}{output\_folder}}, \emph{\DUrole{n}{cluster\_tag}}}{}
\pysigstopsignatures
\sphinxAtStartPar
Class Solver Input Output

\sphinxAtStartPar
A class to store the IO information.
The class in the su2.py file is hardcoded to work with SU2.
\index{mesh\_filename (su2.Solver\_Input\_Output attribute)@\spxentry{mesh\_filename}\spxextra{su2.Solver\_Input\_Output attribute}}

\begin{fulllineitems}
\phantomsection\label{\detokenize{modules:su2.Solver_Input_Output.mesh_filename}}
\pysigstartsignatures
\pysigline{\sphinxbfcode{\sphinxupquote{mesh\_filename}}}
\pysigstopsignatures
\sphinxAtStartPar
{[}str{]} Name of the mesh to be used in the simulation

\end{fulllineitems}

\index{mesh\_format (su2.Solver\_Input\_Output attribute)@\spxentry{mesh\_format}\spxextra{su2.Solver\_Input\_Output attribute}}

\begin{fulllineitems}
\phantomsection\label{\detokenize{modules:su2.Solver_Input_Output.mesh_format}}
\pysigstartsignatures
\pysigline{\sphinxbfcode{\sphinxupquote{mesh\_format}}}
\pysigstopsignatures
\sphinxAtStartPar
{[}str{]} Mesh format (Default = SU2)

\end{fulllineitems}

\index{output\_files (su2.Solver\_Input\_Output attribute)@\spxentry{output\_files}\spxextra{su2.Solver\_Input\_Output attribute}}

\begin{fulllineitems}
\phantomsection\label{\detokenize{modules:su2.Solver_Input_Output.output_files}}
\pysigstartsignatures
\pysigline{\sphinxbfcode{\sphinxupquote{output\_files}}}
\pysigstopsignatures
\sphinxAtStartPar
{[}str{]} Generated output files

\end{fulllineitems}

\index{output\_freq (su2.Solver\_Input\_Output attribute)@\spxentry{output\_freq}\spxextra{su2.Solver\_Input\_Output attribute}}

\begin{fulllineitems}
\phantomsection\label{\detokenize{modules:su2.Solver_Input_Output.output_freq}}
\pysigstartsignatures
\pysigline{\sphinxbfcode{\sphinxupquote{output\_freq}}}
\pysigstopsignatures
\sphinxAtStartPar
{[}str{]} Frequency for the output file generation

\end{fulllineitems}

\index{output\_surf (su2.Solver\_Input\_Output attribute)@\spxentry{output\_surf}\spxextra{su2.Solver\_Input\_Output attribute}}

\begin{fulllineitems}
\phantomsection\label{\detokenize{modules:su2.Solver_Input_Output.output_surf}}
\pysigstartsignatures
\pysigline{\sphinxbfcode{\sphinxupquote{output\_surf}}}
\pysigstopsignatures
\sphinxAtStartPar
{[}str{]} Name of the surface solution filename to write the simulation data

\end{fulllineitems}

\index{output\_vol (su2.Solver\_Input\_Output attribute)@\spxentry{output\_vol}\spxextra{su2.Solver\_Input\_Output attribute}}

\begin{fulllineitems}
\phantomsection\label{\detokenize{modules:su2.Solver_Input_Output.output_vol}}
\pysigstartsignatures
\pysigline{\sphinxbfcode{\sphinxupquote{output\_vol}}}
\pysigstopsignatures
\sphinxAtStartPar
{[}str{]} Name of the volume solution filename to write the simulation data

\end{fulllineitems}

\index{screen (su2.Solver\_Input\_Output attribute)@\spxentry{screen}\spxextra{su2.Solver\_Input\_Output attribute}}

\begin{fulllineitems}
\phantomsection\label{\detokenize{modules:su2.Solver_Input_Output.screen}}
\pysigstartsignatures
\pysigline{\sphinxbfcode{\sphinxupquote{screen}}}
\pysigstopsignatures
\sphinxAtStartPar
{[}str{]} Screen output

\end{fulllineitems}

\index{solution\_input (su2.Solver\_Input\_Output attribute)@\spxentry{solution\_input}\spxextra{su2.Solver\_Input\_Output attribute}}

\begin{fulllineitems}
\phantomsection\label{\detokenize{modules:su2.Solver_Input_Output.solution_input}}
\pysigstartsignatures
\pysigline{\sphinxbfcode{\sphinxupquote{solution\_input}}}
\pysigstopsignatures
\sphinxAtStartPar
{[}str{]} Solution filename to read

\end{fulllineitems}

\index{solution\_output (su2.Solver\_Input\_Output attribute)@\spxentry{solution\_output}\spxextra{su2.Solver\_Input\_Output attribute}}

\begin{fulllineitems}
\phantomsection\label{\detokenize{modules:su2.Solver_Input_Output.solution_output}}
\pysigstartsignatures
\pysigline{\sphinxbfcode{\sphinxupquote{solution\_output}}}
\pysigstopsignatures
\sphinxAtStartPar
{[}str{]} Solution filename to write

\end{fulllineitems}

\index{tabular\_format (su2.Solver\_Input\_Output attribute)@\spxentry{tabular\_format}\spxextra{su2.Solver\_Input\_Output attribute}}

\begin{fulllineitems}
\phantomsection\label{\detokenize{modules:su2.Solver_Input_Output.tabular_format}}
\pysigstartsignatures
\pysigline{\sphinxbfcode{\sphinxupquote{tabular\_format}}}
\pysigstopsignatures
\sphinxAtStartPar
{[}str{]} Solution format

\end{fulllineitems}


\end{fulllineitems}

\index{SU2\_Config (class in su2)@\spxentry{SU2\_Config}\spxextra{class in su2}}

\begin{fulllineitems}
\phantomsection\label{\detokenize{modules:su2.SU2_Config}}
\pysigstartsignatures
\pysiglinewithargsret{\sphinxbfcode{\sphinxupquote{class\DUrole{w}{  }}}\sphinxcode{\sphinxupquote{su2.}}\sphinxbfcode{\sphinxupquote{SU2\_Config}}}{\emph{\DUrole{n}{freestream}}, \emph{\DUrole{n}{assembly}}, \emph{\DUrole{n}{restart}}, \emph{\DUrole{n}{it}}, \emph{\DUrole{n}{iteration}}, \emph{\DUrole{n}{su2}}, \emph{\DUrole{n}{options}}, \emph{\DUrole{n}{cluster\_tag}}}{}
\pysigstopsignatures
\sphinxAtStartPar
Class SU2 Configuration

\sphinxAtStartPar
A class to store all the information required write the SU2 configuration file
The class in the su2.py file is hardcoded to work with SU2.
\index{bc (su2.SU2\_Config attribute)@\spxentry{bc}\spxextra{su2.SU2\_Config attribute}}

\begin{fulllineitems}
\phantomsection\label{\detokenize{modules:su2.SU2_Config.bc}}
\pysigstartsignatures
\pysigline{\sphinxbfcode{\sphinxupquote{bc}}}
\pysigstopsignatures
\sphinxAtStartPar
{[}Solver\_BC{]} Object of class Solver\_BC

\end{fulllineitems}

\index{convergence (su2.SU2\_Config attribute)@\spxentry{convergence}\spxextra{su2.SU2\_Config attribute}}

\begin{fulllineitems}
\phantomsection\label{\detokenize{modules:su2.SU2_Config.convergence}}
\pysigstartsignatures
\pysigline{\sphinxbfcode{\sphinxupquote{convergence}}}
\pysigstopsignatures
\sphinxAtStartPar
{[}Solver\_Convergence{]} Object of class Solver\_Convergence

\end{fulllineitems}

\index{flow (su2.SU2\_Config attribute)@\spxentry{flow}\spxextra{su2.SU2\_Config attribute}}

\begin{fulllineitems}
\phantomsection\label{\detokenize{modules:su2.SU2_Config.flow}}
\pysigstartsignatures
\pysigline{\sphinxbfcode{\sphinxupquote{flow}}}
\pysigstopsignatures
\sphinxAtStartPar
{[}Flow\_Numerical\_Method{]} Object of class Flow\_Numerical\_Method

\end{fulllineitems}

\index{free\_cond (su2.SU2\_Config attribute)@\spxentry{free\_cond}\spxextra{su2.SU2\_Config attribute}}

\begin{fulllineitems}
\phantomsection\label{\detokenize{modules:su2.SU2_Config.free_cond}}
\pysigstartsignatures
\pysigline{\sphinxbfcode{\sphinxupquote{free\_cond}}}
\pysigstopsignatures
\sphinxAtStartPar
{[}Solver\_Freestream\_Conditions{]} Object of class Solver\_Freestream\_Conditions

\end{fulllineitems}

\index{inout (su2.SU2\_Config attribute)@\spxentry{inout}\spxextra{su2.SU2\_Config attribute}}

\begin{fulllineitems}
\phantomsection\label{\detokenize{modules:su2.SU2_Config.inout}}
\pysigstartsignatures
\pysigline{\sphinxbfcode{\sphinxupquote{inout}}}
\pysigstopsignatures
\sphinxAtStartPar
{[}Solver\_Input\_Output{]} Object of class Solver\_Input\_Output

\end{fulllineitems}

\index{name (su2.SU2\_Config attribute)@\spxentry{name}\spxextra{su2.SU2\_Config attribute}}

\begin{fulllineitems}
\phantomsection\label{\detokenize{modules:su2.SU2_Config.name}}
\pysigstartsignatures
\pysigline{\sphinxbfcode{\sphinxupquote{name}}}
\pysigstopsignatures
\sphinxAtStartPar
{[}str{]} Name of the configuration file

\end{fulllineitems}

\index{num (su2.SU2\_Config attribute)@\spxentry{num}\spxextra{su2.SU2\_Config attribute}}

\begin{fulllineitems}
\phantomsection\label{\detokenize{modules:su2.SU2_Config.num}}
\pysigstartsignatures
\pysigline{\sphinxbfcode{\sphinxupquote{num}}}
\pysigstopsignatures
\sphinxAtStartPar
{[}Solver\_Numerical\_Method{]} Object of class Solver\_Numerical\_Method

\end{fulllineitems}

\index{ref (su2.SU2\_Config attribute)@\spxentry{ref}\spxextra{su2.SU2\_Config attribute}}

\begin{fulllineitems}
\phantomsection\label{\detokenize{modules:su2.SU2_Config.ref}}
\pysigstartsignatures
\pysigline{\sphinxbfcode{\sphinxupquote{ref}}}
\pysigstopsignatures
\sphinxAtStartPar
{[}Solver\_Reference\_Value{]} Object of class Solver\_Reference\_Value

\end{fulllineitems}

\index{solver (su2.SU2\_Config attribute)@\spxentry{solver}\spxextra{su2.SU2\_Config attribute}}

\begin{fulllineitems}
\phantomsection\label{\detokenize{modules:su2.SU2_Config.solver}}
\pysigstartsignatures
\pysigline{\sphinxbfcode{\sphinxupquote{solver}}}
\pysigstopsignatures
\sphinxAtStartPar
{[}Solver{]} Object of class Solver

\end{fulllineitems}


\end{fulllineitems}

\index{write\_SU2\_config() (in module su2)@\spxentry{write\_SU2\_config()}\spxextra{in module su2}}

\begin{fulllineitems}
\phantomsection\label{\detokenize{modules:su2.write_SU2_config}}
\pysigstartsignatures
\pysiglinewithargsret{\sphinxcode{\sphinxupquote{su2.}}\sphinxbfcode{\sphinxupquote{write\_SU2\_config}}}{\emph{\DUrole{n}{freestream}}, \emph{\DUrole{n}{assembly}}, \emph{\DUrole{n}{restart}}, \emph{\DUrole{n}{it}}, \emph{\DUrole{n}{iteration}}, \emph{\DUrole{n}{su2}}, \emph{\DUrole{n}{options}}, \emph{\DUrole{n}{cluster\_tag}}, \emph{\DUrole{n}{input\_grid}}, \emph{\DUrole{n}{output\_grid}\DUrole{o}{=}\DUrole{default_value}{\textquotesingle{}\textquotesingle{}}}, \emph{\DUrole{n}{interpolation}\DUrole{o}{=}\DUrole{default_value}{False}}, \emph{\DUrole{n}{bloom}\DUrole{o}{=}\DUrole{default_value}{False}}, \emph{\DUrole{n}{interp\_to\_BL}\DUrole{o}{=}\DUrole{default_value}{False}}}{}
\pysigstopsignatures
\sphinxAtStartPar
Write the SU2 configuration file

\sphinxAtStartPar
Generates a configuration file to run a SU2 CFD simulation according to the position of the object and the user\sphinxhyphen{}defined parameters.
\begin{quote}\begin{description}
\sphinxlineitem{Parameters}\begin{itemize}
\item {} 
\sphinxAtStartPar
\sphinxstyleliteralstrong{\sphinxupquote{freestream}} ({\hyperref[\detokenize{modules:configuration.Freestream}]{\sphinxcrossref{\sphinxstyleliteralemphasis{\sphinxupquote{Freestream}}}}}) \textendash{} Object of class Freestream

\item {} 
\sphinxAtStartPar
\sphinxstyleliteralstrong{\sphinxupquote{assembly}} ({\hyperref[\detokenize{modules:assembly.Assembly_list}]{\sphinxcrossref{\sphinxstyleliteralemphasis{\sphinxupquote{Assembly\_list}}}}}) \textendash{} Object of class Assembly\_list

\item {} 
\sphinxAtStartPar
\sphinxstyleliteralstrong{\sphinxupquote{restart}} (\sphinxstyleliteralemphasis{\sphinxupquote{bool}}) \textendash{} Boolean value to indicate if CFD simulation is restarting from previous solution

\item {} 
\sphinxAtStartPar
\sphinxstyleliteralstrong{\sphinxupquote{it}} (\sphinxstyleliteralemphasis{\sphinxupquote{int}}) \textendash{} Value of adaptive iteration

\item {} 
\sphinxAtStartPar
\sphinxstyleliteralstrong{\sphinxupquote{iteration}} (\sphinxstyleliteralemphasis{\sphinxupquote{int}}) \textendash{} Value of time iteration

\item {} 
\sphinxAtStartPar
\sphinxstyleliteralstrong{\sphinxupquote{su2}} (\sphinxstyleliteralemphasis{\sphinxupquote{CFD}}) \textendash{} Object of class CFD

\item {} 
\sphinxAtStartPar
\sphinxstyleliteralstrong{\sphinxupquote{options}} ({\hyperref[\detokenize{modules:configuration.Options}]{\sphinxcrossref{\sphinxstyleliteralemphasis{\sphinxupquote{Options}}}}}) \textendash{} Object of class Options

\item {} 
\sphinxAtStartPar
\sphinxstyleliteralstrong{\sphinxupquote{cluster\_tag}} (\sphinxstyleliteralemphasis{\sphinxupquote{int}}) \textendash{} Value of the cluster tag number for simulation parallelization

\item {} 
\sphinxAtStartPar
\sphinxstyleliteralstrong{\sphinxupquote{input\_grid}} (\sphinxstyleliteralemphasis{\sphinxupquote{str}}) \textendash{} Name of the input mesh file

\item {} 
\sphinxAtStartPar
\sphinxstyleliteralstrong{\sphinxupquote{output\_grid}} (\sphinxstyleliteralemphasis{\sphinxupquote{str}}) \textendash{} Name of the output file

\end{itemize}

\end{description}\end{quote}

\end{fulllineitems}

\index{retrieve\_index() (in module su2)@\spxentry{retrieve\_index()}\spxextra{in module su2}}

\begin{fulllineitems}
\phantomsection\label{\detokenize{modules:su2.retrieve_index}}
\pysigstartsignatures
\pysiglinewithargsret{\sphinxcode{\sphinxupquote{su2.}}\sphinxbfcode{\sphinxupquote{retrieve\_index}}}{\emph{\DUrole{n}{SU2\_type}}}{}
\pysigstopsignatures
\sphinxAtStartPar
Retrieve index to retrieve solution fields

\sphinxAtStartPar
Returns the index to read the correct fields in the solution file, according to the user\sphinxhyphen{}specified solver
\begin{quote}\begin{description}
\sphinxlineitem{Parameters}
\sphinxAtStartPar
\sphinxstyleliteralstrong{\sphinxupquote{SU2\_type}} (\sphinxstyleliteralemphasis{\sphinxupquote{str}}) \textendash{} Solver used in the CFD simulation

\sphinxlineitem{Returns}
\sphinxAtStartPar
\sphinxstylestrong{index} \textendash{} Array of index with the position of solution fields of interest

\sphinxlineitem{Return type}
\sphinxAtStartPar
np.array()

\end{description}\end{quote}

\end{fulllineitems}

\index{read\_vtk\_from\_su2\_v2() (in module su2)@\spxentry{read\_vtk\_from\_su2\_v2()}\spxextra{in module su2}}

\begin{fulllineitems}
\phantomsection\label{\detokenize{modules:su2.read_vtk_from_su2_v2}}
\pysigstartsignatures
\pysiglinewithargsret{\sphinxcode{\sphinxupquote{su2.}}\sphinxbfcode{\sphinxupquote{read\_vtk\_from\_su2\_v2}}}{\emph{\DUrole{n}{filename}}, \emph{\DUrole{n}{assembly\_coords}}, \emph{\DUrole{n}{idx\_inv}}, \emph{\DUrole{n}{options}}}{}
\pysigstopsignatures
\sphinxAtStartPar
Read the VTK file solution

\sphinxAtStartPar
Reads and retrieves the solution stored in the VTK file format
\begin{quote}\begin{description}
\sphinxlineitem{Parameters}\begin{itemize}
\item {} 
\sphinxAtStartPar
\sphinxstyleliteralstrong{\sphinxupquote{filename}} (\sphinxstyleliteralemphasis{\sphinxupquote{str}}) \textendash{} Name and location of the VTK solution file

\item {} 
\sphinxAtStartPar
\sphinxstyleliteralstrong{\sphinxupquote{assembly\_coords}} (\sphinxstyleliteralemphasis{\sphinxupquote{np.array}}\sphinxstyleliteralemphasis{\sphinxupquote{(}}\sphinxstyleliteralemphasis{\sphinxupquote{)}}) \textendash{} Coordinates of the mesh nodes

\item {} 
\sphinxAtStartPar
\sphinxstyleliteralstrong{\sphinxupquote{idx\_inv}} (\sphinxstyleliteralemphasis{\sphinxupquote{np.array}}) \textendash{} Sort indexing such that the VTK retrieved solution corresponds to the stored mesh nodes positioning

\item {} 
\sphinxAtStartPar
\sphinxstyleliteralstrong{\sphinxupquote{options}} ({\hyperref[\detokenize{modules:configuration.Options}]{\sphinxcrossref{\sphinxstyleliteralemphasis{\sphinxupquote{Options}}}}}) \textendash{} Object of class Options

\end{itemize}

\sphinxlineitem{Returns}
\sphinxAtStartPar
\sphinxstylestrong{aerothermo} \textendash{} object of class Aerothermo

\sphinxlineitem{Return type}
\sphinxAtStartPar
{\hyperref[\detokenize{modules:configuration.Aerothermo}]{\sphinxcrossref{Aerothermo}}}

\end{description}\end{quote}

\end{fulllineitems}

\index{split\_aerothermo() (in module su2)@\spxentry{split\_aerothermo()}\spxextra{in module su2}}

\begin{fulllineitems}
\phantomsection\label{\detokenize{modules:su2.split_aerothermo}}
\pysigstartsignatures
\pysiglinewithargsret{\sphinxcode{\sphinxupquote{su2.}}\sphinxbfcode{\sphinxupquote{split\_aerothermo}}}{\emph{\DUrole{n}{total\_aerothermo}}, \emph{\DUrole{n}{assembly}}}{}
\pysigstopsignatures
\sphinxAtStartPar
Split the solution into the different assemblies used in the CFD simulation
\begin{quote}\begin{description}
\sphinxlineitem{Parameters}\begin{itemize}
\item {} 
\sphinxAtStartPar
\sphinxstyleliteralstrong{\sphinxupquote{total\_aerothermo}} ({\hyperref[\detokenize{modules:configuration.Aerothermo}]{\sphinxcrossref{\sphinxstyleliteralemphasis{\sphinxupquote{Aerothermo}}}}}) \textendash{} Object of class Aerothermo

\item {} 
\sphinxAtStartPar
\sphinxstyleliteralstrong{\sphinxupquote{assembly}} (\sphinxstyleliteralemphasis{\sphinxupquote{List\_Assembly}}) \textendash{} Object of class List\_Assembly

\end{itemize}

\end{description}\end{quote}

\end{fulllineitems}

\index{run\_SU2() (in module su2)@\spxentry{run\_SU2()}\spxextra{in module su2}}

\begin{fulllineitems}
\phantomsection\label{\detokenize{modules:su2.run_SU2}}
\pysigstartsignatures
\pysiglinewithargsret{\sphinxcode{\sphinxupquote{su2.}}\sphinxbfcode{\sphinxupquote{run\_SU2}}}{\emph{\DUrole{n}{n}}, \emph{\DUrole{n}{options}}}{}
\pysigstopsignatures
\sphinxAtStartPar
Calls the SU2 executable and run the simulation
\begin{quote}\begin{description}
\sphinxlineitem{Parameters}\begin{itemize}
\item {} 
\sphinxAtStartPar
\sphinxstyleliteralstrong{\sphinxupquote{n}} (\sphinxstyleliteralemphasis{\sphinxupquote{int}}) \textendash{} Number of cores

\item {} 
\sphinxAtStartPar
\sphinxstyleliteralstrong{\sphinxupquote{options}} ({\hyperref[\detokenize{modules:configuration.Options}]{\sphinxcrossref{\sphinxstyleliteralemphasis{\sphinxupquote{Options}}}}}) \textendash{} Object of class Options

\end{itemize}

\end{description}\end{quote}

\end{fulllineitems}

\index{generate\_BL() (in module su2)@\spxentry{generate\_BL()}\spxextra{in module su2}}

\begin{fulllineitems}
\phantomsection\label{\detokenize{modules:su2.generate_BL}}
\pysigstartsignatures
\pysiglinewithargsret{\sphinxcode{\sphinxupquote{su2.}}\sphinxbfcode{\sphinxupquote{generate\_BL}}}{\emph{\DUrole{n}{assembly}}, \emph{\DUrole{n}{options}}, \emph{\DUrole{n}{it}}, \emph{\DUrole{n}{cluster\_tag}}}{}
\pysigstopsignatures
\sphinxAtStartPar
Generates a Boundary Layer
\begin{quote}\begin{description}
\sphinxlineitem{Parameters}\begin{itemize}
\item {} 
\sphinxAtStartPar
\sphinxstyleliteralstrong{\sphinxupquote{assembly}} (\sphinxstyleliteralemphasis{\sphinxupquote{List\_Assembly}}) \textendash{} Object of class List\_Assembly

\item {} 
\sphinxAtStartPar
\sphinxstyleliteralstrong{\sphinxupquote{options}} ({\hyperref[\detokenize{modules:configuration.Options}]{\sphinxcrossref{\sphinxstyleliteralemphasis{\sphinxupquote{Options}}}}}) \textendash{} Object of class Options

\item {} 
\sphinxAtStartPar
\sphinxstyleliteralstrong{\sphinxupquote{it}} (\sphinxstyleliteralemphasis{\sphinxupquote{int}}) \textendash{} Value of adaptive iteration

\item {} 
\sphinxAtStartPar
\sphinxstyleliteralstrong{\sphinxupquote{cluster\_tag}} (\sphinxstyleliteralemphasis{\sphinxupquote{int}}) \textendash{} Value of Cluster tag

\end{itemize}

\end{description}\end{quote}

\end{fulllineitems}

\index{adapt\_mesh() (in module su2)@\spxentry{adapt\_mesh()}\spxextra{in module su2}}

\begin{fulllineitems}
\phantomsection\label{\detokenize{modules:su2.adapt_mesh}}
\pysigstartsignatures
\pysiglinewithargsret{\sphinxcode{\sphinxupquote{su2.}}\sphinxbfcode{\sphinxupquote{adapt\_mesh}}}{\emph{\DUrole{n}{assembly}}, \emph{\DUrole{n}{options}}, \emph{\DUrole{n}{it}}, \emph{\DUrole{n}{cluster\_tag}}}{}
\pysigstopsignatures
\sphinxAtStartPar
Anisotropically adapts the mesh
\begin{quote}\begin{description}
\sphinxlineitem{Parameters}\begin{itemize}
\item {} 
\sphinxAtStartPar
\sphinxstyleliteralstrong{\sphinxupquote{assembly}} (\sphinxstyleliteralemphasis{\sphinxupquote{List\_Assembly}}) \textendash{} Object of class List\_Assembly

\item {} 
\sphinxAtStartPar
\sphinxstyleliteralstrong{\sphinxupquote{options}} ({\hyperref[\detokenize{modules:configuration.Options}]{\sphinxcrossref{\sphinxstyleliteralemphasis{\sphinxupquote{Options}}}}}) \textendash{} Object of class Options

\item {} 
\sphinxAtStartPar
\sphinxstyleliteralstrong{\sphinxupquote{it}} (\sphinxstyleliteralemphasis{\sphinxupquote{int}}) \textendash{} Value of adaptive iteration

\item {} 
\sphinxAtStartPar
\sphinxstyleliteralstrong{\sphinxupquote{cluster\_tag}} (\sphinxstyleliteralemphasis{\sphinxupquote{int}}) \textendash{} Value of Cluster tag

\end{itemize}

\end{description}\end{quote}

\end{fulllineitems}

\index{compute\_cfd\_aerothermo() (in module su2)@\spxentry{compute\_cfd\_aerothermo()}\spxextra{in module su2}}

\begin{fulllineitems}
\phantomsection\label{\detokenize{modules:su2.compute_cfd_aerothermo}}
\pysigstartsignatures
\pysiglinewithargsret{\sphinxcode{\sphinxupquote{su2.}}\sphinxbfcode{\sphinxupquote{compute\_cfd\_aerothermo}}}{\emph{\DUrole{n}{assembly\_list}}, \emph{\DUrole{n}{options}}, \emph{\DUrole{n}{cluster\_tag}\DUrole{o}{=}\DUrole{default_value}{0}}}{}
\pysigstopsignatures
\sphinxAtStartPar
Compute the aerothermodynamic properties using the CFD software
\begin{quote}\begin{description}
\sphinxlineitem{Parameters}\begin{itemize}
\item {} 
\sphinxAtStartPar
\sphinxstyleliteralstrong{\sphinxupquote{assembly\_list}} (\sphinxstyleliteralemphasis{\sphinxupquote{List\_Assembly}}) \textendash{} Object of class List\_Assembly

\item {} 
\sphinxAtStartPar
\sphinxstyleliteralstrong{\sphinxupquote{options}} ({\hyperref[\detokenize{modules:configuration.Options}]{\sphinxcrossref{\sphinxstyleliteralemphasis{\sphinxupquote{Options}}}}}) \textendash{} Object of class Options

\item {} 
\sphinxAtStartPar
\sphinxstyleliteralstrong{\sphinxupquote{cluster\_tag}} (\sphinxstyleliteralemphasis{\sphinxupquote{int}}) \textendash{} Value of Cluster tag

\end{itemize}

\end{description}\end{quote}

\end{fulllineitems}



\section{Multi\sphinxhyphen{}fidelity switch}
\label{\detokenize{modules:multi-fidelity-switch}}\index{compute\_aerothermo() (in module switch)@\spxentry{compute\_aerothermo()}\spxextra{in module switch}}

\begin{fulllineitems}
\phantomsection\label{\detokenize{modules:switch.compute_aerothermo}}
\pysigstartsignatures
\pysiglinewithargsret{\sphinxcode{\sphinxupquote{switch.}}\sphinxbfcode{\sphinxupquote{compute\_aerothermo}}}{\emph{\DUrole{n}{titan}}, \emph{\DUrole{n}{options}}}{}
\pysigstopsignatures
\sphinxAtStartPar
Aerothermo computation using a multi\sphinxhyphen{}fidelity approach (i.e. can use both low\sphinxhyphen{} and high\sphinxhyphen{}fidelity methodology)

\sphinxAtStartPar
The function uses the Billig formula to assess the shock envelope criteria, used to determine wether to use low\sphinxhyphen{} or high\sphinxhyphen{}fidelity methods
\begin{quote}\begin{description}
\sphinxlineitem{Parameters}\begin{itemize}
\item {} 
\sphinxAtStartPar
\sphinxstyleliteralstrong{\sphinxupquote{titan}} (\sphinxstyleliteralemphasis{\sphinxupquote{List\_Assembly}}) \textendash{} Object of class List\_Assembly

\item {} 
\sphinxAtStartPar
\sphinxstyleliteralstrong{\sphinxupquote{options}} ({\hyperref[\detokenize{modules:configuration.Options}]{\sphinxcrossref{\sphinxstyleliteralemphasis{\sphinxupquote{Options}}}}}) \textendash{} Object of class Options

\end{itemize}

\end{description}\end{quote}

\end{fulllineitems}

\index{compute\_billig() (in module switch)@\spxentry{compute\_billig()}\spxextra{in module switch}}

\begin{fulllineitems}
\phantomsection\label{\detokenize{modules:switch.compute_billig}}
\pysigstartsignatures
\pysiglinewithargsret{\sphinxcode{\sphinxupquote{switch.}}\sphinxbfcode{\sphinxupquote{compute\_billig}}}{\emph{\DUrole{n}{M}}, \emph{\DUrole{n}{theta}}, \emph{\DUrole{n}{center}}, \emph{\DUrole{n}{sphere\_radius}}, \emph{\DUrole{n}{index\_assembly}}, \emph{\DUrole{n}{assembly}}, \emph{\DUrole{n}{list\_assembly}}, \emph{\DUrole{n}{index\_object}}, \emph{\DUrole{n}{i}}, \emph{\DUrole{n}{true\_assembly}}}{}
\pysigstopsignatures
\sphinxAtStartPar
Computation of the shock envelope using the Billing formula

\sphinxAtStartPar
if the object is inside the shock envelope generated by an upstream body, the framework will use the high\sphinxhyphen{}fidelity methodology to compute the aero
thermodynamics. Else, it will use low\sphinxhyphen{}fidelity methodology
\begin{quote}\begin{description}
\sphinxlineitem{Parameters}\begin{itemize}
\item {} 
\sphinxAtStartPar
\sphinxstyleliteralstrong{\sphinxupquote{M}} (\sphinxstyleliteralemphasis{\sphinxupquote{float}}) \textendash{} Freestream Mach number

\item {} 
\sphinxAtStartPar
\sphinxstyleliteralstrong{\sphinxupquote{theta}} (\sphinxstyleliteralemphasis{\sphinxupquote{float}}) \textendash{} Shockwave inclination angle

\item {} 
\sphinxAtStartPar
\sphinxstyleliteralstrong{\sphinxupquote{center}} (\sphinxstyleliteralemphasis{\sphinxupquote{np.array}}\sphinxstyleliteralemphasis{\sphinxupquote{(}}\sphinxstyleliteralemphasis{\sphinxupquote{)}}) \textendash{} Coordinates of the sphere center

\item {} 
\sphinxAtStartPar
\sphinxstyleliteralstrong{\sphinxupquote{sphere\_radius}} (\sphinxstyleliteralemphasis{\sphinxupquote{float}}) \textendash{} Radius of the sphere

\item {} 
\sphinxAtStartPar
\sphinxstyleliteralstrong{\sphinxupquote{index\_assembly}} (\sphinxstyleliteralemphasis{\sphinxupquote{int}}) \textendash{} Index of the assembly producing the shockwave

\item {} 
\sphinxAtStartPar
\sphinxstyleliteralstrong{\sphinxupquote{assembly}} (\sphinxstyleliteralemphasis{\sphinxupquote{List\_Assembly}}) \textendash{} Object of List\_Assembly

\item {} 
\sphinxAtStartPar
\sphinxstyleliteralstrong{\sphinxupquote{list\_assembly}} (\sphinxstyleliteralemphasis{\sphinxupquote{np.array}}\sphinxstyleliteralemphasis{\sphinxupquote{(}}\sphinxstyleliteralemphasis{\sphinxupquote{)}}) \textendash{} Index of the remaining assemblies to check if they are inside or outside the shock envelope

\end{itemize}

\sphinxlineitem{Returns}
\sphinxAtStartPar
\sphinxstylestrong{computational\_domain\_bodies} \textendash{} List of bodies inside the shock envelope

\sphinxlineitem{Return type}
\sphinxAtStartPar
List

\end{description}\end{quote}

\end{fulllineitems}


\sphinxstepscope


\chapter{Example}
\label{\detokenize{example:example}}\label{\detokenize{example::doc}}
\sphinxAtStartPar
Examples of configuration files can be found in the folder TITAN/Examples and can be run by the user.



\renewcommand{\indexname}{Index}
\printindex
\end{document}